%! mpprop.tex $ this file belongs to the Molcas repository $*/
%! mpprop.tex

\section{\program{mpprop}}
\label{UG:sec:mpprop}
\index{Program!MpProp@\program{MpProp}}\index{MpProp@\program{MpProp}}

\subsection{Description}
\label{UG:sec:mpprop_description}
%%Description:
%%+The MpProp program generates a distributed multipole expansion of the charge density
%%+of a molecule and atom distributed polarizabilities.

The \program{MPPROP} is a general distributed multipole expansion, and a first order polarizabilty analyzis program.
It will use the one electron integrals to generate the distribution. The order of the distributed multipole expansion is defined by
\program{SEWARD}. In order to generate distributed multipoles of higher order than 2. One has to use the Keyword
MULTipoles in \program{SEWARD}. \program{SEWARD} also needs the Keyword NEMO to arrange the integrals in correct order.

\subsection{Dependencies}
\label{UG:sec:mpprop_dependencies}
The \program{MPPROP} program requires the one-{}electron integral file
\file{ONEINT} and the communications file \file{RUNFILE},
which contains among others the
multipole integrals produced by \program{SEWARD}.

\subsection{Files}
\label{UG:sec:mpprop_files}
\index{Files!MPPROP}\index{MPPROP!Files}

Below is a list of the files that are used/created by the program
\program{MPPROP}.

\subsubsection{Input files}
\begin{filelist}
%---
\item[ONEINT]
One{}-electron integral file generated by the program {\prgmfont SEWARD}.
%---
\item[RUNFILE]
File for communication of auxiliary information generated by the different programs
e.g. {\prgmfont SEWARD}.
%---
\item[INPORB]
\file{SCFORB} or \file{RASORB} file containing the orbitals of a previous
\program{SCF} run or a \program{RASSCF} run , which are used now as vectors in the \program{MPPROP} run.
%---
\end{filelist}

\subsubsection{Output files}

\begin{filelist}
%---
\item[MPPROP]
The distributed multipole expansion.
%---
\end{filelist}

\subsection{Input}
\label{UG:sec:mpprop_input}
\index{Input!MPPROP}\index{MPPROP!Input}

Below follows a description of the input to \program{MPPROP}. The keywords
are always significant to four characters, but in order to make the
input more transparent, it is recommended to use the full keywords.
The \program{MPPROP} program section of the \molcas\ input starts with the
program:

\namelist{\&MPPROP}

There are no compulsory keywords.

\subsubsection{Optional general keywords}
\begin{keywordlist}
%---
%%Keyword: BONDs <basic>
%%+Use this Keyword to define bond between atoms.
%%+This Keyword should be followed by a line of atomlabels
%%+separated by a space. The following line can define
%%+another bond. This Keyword should be ended by a END statement
%%+in the last line. The example below means that O1 will bond to H1 and H2.
%%+It does not mean that H1 is bonded to H2.
\item[BONDs]
Use this Keyword to define bond between atoms.
This Keyword should be followed by a line of atomlabels
separated by a space. The following line can define
another bond. This Keyword should be ended by a END statement
in the last line. The example below means that O1 will bond to H1 and H2.
It does not mean that H1 is bonded to H2.
\begin{inputlisting}
BOND
O1 H1 H2
END
\end{inputlisting}

%---
%%Keyword: TITLe <basic>
%%+This Keyword specifies the title of the molecule. This will be
%%+recognized by the Nemo package. And you are requested to use
%%+this Keyword. It is defined in the program as a Character*80
\item[TITLe]
This Keyword specifies the title of the molecule. This will be
recognized by the Nemo package. And you are requested to use
this Keyword. It is defined in the program as a Character*80

%%Keyword: LUMOrb <basic>
%%+This Keyword tells MPPROP to use an INPORB file for
%%+the one electron densities.
\item[LUMOrb]
This Keyword tells \program{MPPROP} to use an INPORB file for
the one electron densities.

%%Keyword: TYPE <basic>
%%+This is to specify the typen of the atom.
%%+Where the first number is the atomnumber m counted as in SEWARD.
%%+The second number is the type of the atom n.
\item[TYPE]
This is to specify the typen of the atom.
Where the first number is the atomnumber m, counted in the order it was defined in \program{SEWARD}.
The second number is the type of the atom n.
\begin{inputlisting}
TYPE
m n
\end{inputlisting}

%%Keyword: POLArizability <basic>
%%+This specifies if the polarizability should be calculated or not.
%%+0 Means no polarizability should be calculated.
%%+1 (Default) Means polarizability should be calculated.
\item[POLArizability]
This specifies if the polarizability should be calculated or not.
\begin{inputlisting}
POLArizability
m
\end{inputlisting}

\begin{itemize}
\itemsep 9pt plus 3pt minus 3pt
\item
{\bf m=0}
\index{MPPROP!m=0}
Means no polarizability should be calculated.
\item
{\bf m=1}
\index{MPPROP!m=1}
(Default) Means polarizability should be calculated.
% 2 !not implemented! the polarizability according to the new distribution
\end{itemize}

%%Keyword: NONEarestAtom <basic>
%%+The program is written in the way that multipoles should be moved
%%+to the nearest atom if the nearest atom is closer than any of the
%%+bonding atoms. Note that the move will be to atoms and not nearest bond.
%%+This can be implemented if requested
\item[NONEarestAtom]
The program is written in the way that multipoles should be moved
to the nearest atom if the nearest atom is closer than any of the
bonding atoms.Note that the move will be to atoms and not nearest bond.

%%Keyword: ALLCenter <basic>
%%+This Keyword means that all centers are considered for the distributed multipole expansion.
\item[ALLCenter]
This Keywords means that all centers are considered for the distributed multipole expansion.

\end{keywordlist}

\subsubsection{Limitations}
The limitations on the order of the multipole expansion is defined by {\prgmfont SEWARD}.
While the polarizability can only be calculated directly in the program for an scf wavefunction.
And it is limited to first order polarizability

\subsubsection{Input examples}

First we have the bare minimum of input. This will work well for all systems.

\begin{inputlisting}
 &MPPROP
\end{inputlisting}

The next example is a bit more elaborate and show how to use
a few of the keywords. The system is formic-acid.

\begin{inputlisting}
 &Gateway
Title
Fa
NEMO
Basis set
C.ANO-L...3s2p1d.
C1      2.15211991525414     -3.97152266198745      4.15134452433510
End of Basis set

Basis set
O.ANO-L...4s3p1d.
O2     3.99101917304681     -2.23465022227817      3.72611355598476
O3     2.36712399248396     -5.81178517731397      5.48680572323840
End of Basis set

Basis set
H.ANO-L...3s1p.
H4      0.43787447048429     -3.44210745229883      3.08410918233085
End of Basis set

Basis set
H.ANO-L...3s2p.
H5     5.46574083366162     -2.78397269852552      4.70186773165853
End of Basis set
 &Seward

& Scf
Title
Formic-acid
Occupied
  12

 &MPPROP
Title
Formic-acid
lumorb
POLArizability
1
BONDs
C1 O2 O3 H4
O2 H5
End
TYPE
2 1
3 2
4 1
5 2
End
\end{inputlisting}


