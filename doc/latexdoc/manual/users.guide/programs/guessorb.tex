% guessorb.tex $ this file belongs to the Molcas repository $

%%==============================================================================
%%
%%==============================================================================
\section{\program{guessorb}}
\label{UG:sec:guessorb}
\index{Program!Guessorb@\program{Guessorb}}\index{Guessorb@\program{Guessorb}}

%%------------------------------------------------------------------------------
%%
%%------------------------------------------------------------------------------
\subsection{Description}
\label{UG:sec:guessorb_description}
%%Description:
%%+The GUESSORB program generates a start guess for orbitals.
%%+These orbitals can be used as input for all wavefunction code.

The \program{GUESSORB} program generates a start guess for orbitals.
The file \file{GSSORB} is created containing these orbitals.
They are also put to the runfile and their presence is automatically
detected by the programs \program{SCF} and \program{RASSCF}
if needed.

%%------------------------------------------------------------------------------
%%
%%------------------------------------------------------------------------------
\subsection{Dependencies}
\label{UG:sec:guessorb_dependencies}
The \program{GUESSORB} program requires that the one electron
file \file{ONEINT} as well as the communication file
\file{RUNFILE} exist. These are generated by the program \program{SEWARD}

%%------------------------------------------------------------------------------
%%
%%------------------------------------------------------------------------------
\subsection{Files}
\label{UG:sec:guessorb_files}
\index{Files!Guessorb}\index{Guessorb!Files}

Below is a list of the files that are used/created by the program
\program{GUESSORB}.

\subsubsection{Input files}
\program{GUESSORB} will use the following input
files: \file{ONEINT}, \file{RUNFILE}
(for more information see~\ref{UG:sec:files_list}).


\subsubsection{Output files}

\begin{filelist}
%---
\item[GSSORB]
\program{GUESSORB} orbital output file.
Contains a start guess for orbitals.
%---
\item[RUNFILE]
Communication file for subsequent programs.
%---
\item[MD\_GSS]
Molden input file for molecular orbital analysis.
%---
\end{filelist}

%%------------------------------------------------------------------------------
%%
%%------------------------------------------------------------------------------
\subsection{Input}
\label{UG:sec:guessorb_input}
\index{Input!Guessorb}\index{Guessorb!Input}

Below follows a description of the input to \program{GUESSORB}.

\subsubsection{Keywords}

\begin{keywordlist}
%---
\item[PRMO]
This keyword will make \program{Guessorb} print the orbitals that are
generated. On the next line an integer is to be specified that control
how much output you get, see below. On the same line you can optionally specify
a floating point number that control how many orbitals are printed.
Only orbitals with orbital energy less than this number will be printed,
default is 5.0au.
\begin{list}{}{}
\item 1 --- Only occupation numbers and orbital energies are printed.
\item 2 --- As for 1 but with an additional sorted list of orbital energies.
\item 3 --- As for 2 but with orbitals printed in compact format.
\item 4 --- As for 3 but orbitals are printed in full format.
\end{list}
%---
\item[PRPOpulation]
This keyword will print a Mulliken population analysis based on the
assumptions guessorb make with regards to populating orbitals.
%---
\item[STHR]
This keyword controls how many orbitals will be deleted.
On the next line you specify a threshold that have the default $1\times 10^{-5}$.
The overlap matrix is diagonalized and only eigenvectors
with eigenvalues larger that this threshold will be used,
the other will be deleted.
This removes near liner dependence.
%---
\item[TTHR]
This keyword controls how many orbitals will be deleted.
On the next line you specify a threshold that have the default $1\times 10^6$.
The kinetic energy matrix is diagonalized in the space
of virtual orbitals and only orbitals with energies below
this threshold is used, the other will be deleted.
This removes degrees of freedom describing core correlation.
%---
\item[GAPThr]
This keyword controls how guessorb attempt to populate
the orbitals.
On the next line a threshold is specified that have
the default 0.01.
Using this threshold guessorb will make a closed shell
configuration if it can find a HOMO/LUMO gap that is larger
than the specified threshold.
If that can not be done, guessorb will place a number of orbitals
in an active space in such a way that the gap between the
three spaces (inactive, active and secondary) will be
larger than the threshold.
%---
\item[END of input]
%---
\end{keywordlist}

\subsubsection{Input examples}

In this example \program{Guessorb} is used to produce
a Mulliken population based on assumptions that are made for
population of orbitals.

\begin{inputlisting}
 &GUESSORB
PrPopulation
\end{inputlisting}

In this example \program{Guessorb} is used to construct
an active space for \program{RASSCF} if there are
near degeneracies around the fermi level.
The orbital file that is produced can be fed directly
into \program{RASSCF} without specifying the active space.

\begin{inputlisting}
 &GUESSORB
GapThr =  0.5

 &RASSCF
LumOrb
\end{inputlisting}
