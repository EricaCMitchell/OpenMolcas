% mula.tex $ this file belongs to the Molcas repository $
\section{\program{Mula}}
\label{sec:mula}
\index{Program!Mula@\program{Mula}}\index{Mula@\program{Mula}}
%%Description:
%%+This program computes intensities of vibrational
%%+transitions between electronic states.

The \program{MULA} calculates intensities of vibrational
transitions between electronic states.

\subsection{Dependencies}
\label{sec:mula_dependencies}
\index{MULA!Dependencies}\index{Dependencies!MULA}
The \program{MULA} program may need one or more UNSYM files produced
by the \program{MCLR} program, depending on input options.

\subsection{Files}
\label{sec:mula_files}
\index{MULA!Files}\index{Files!MULA}

\subsubsection{Input files}
\begin{filelist}
%------
\item[UNSYM]
Output file from the \program{MCLR} program
%------
\end{filelist}

\subsubsection{Output files}

\begin{filelist}
%------
\item[plot.intensity]
Contains data for plotting an artificial spectrum.

\end{filelist}

\subsection{Input}
\label{sec:mula_input}
\index{MULA!Input}\index{Input!MULA}
The input for \program{MULA} begins after the program name:
\begin{inputlisting}
 &MULA
\end{inputlisting}

There are no compulsory keyword.

\subsubsection{Keywords}
\index{MULA!Keywords}\index{Keywords!MULA}

\begin{keywordlist}
%---
\item[TITLe]
%%Keyword:TITLe <basic>
%%+A single title line follows.
Followed by a single line, the title of the calculation.
%---
\item[FORCe]
%%Keyword:FORCe <basic>
%%+A force field will be given as input (or read from file).
A force field will be given as input (or read from file), defining two
oscillators for which individual vibrational levels and transition
data will be computed.
%---
\item[ATOMs]
%%Keyword:ATOMs <basic> <endable>
%%+Followed by one line with an atom label for each individual atom
%%+in the molecule. A label consists of element name followed by a
%%+numeric label, optionally followed by a nuclear mass.0
Followed by one line for each individual atom in the molecule.
On each line is the label of the atom, consisting of an element symbol
followed by a number. After the label, separated by one or more blanks,
one can optionally give a mass number; else, a standard mass taken from
the file data/atomic.data.
After these lines is one single line with the keyword "END of atoms".
%---
\item[INTErnal]
%%Keyword:INTErnal <basic> <endable>
%%+Followed by lines of the form e.g. 'BOND C11 Br3', i.e. coordinate type
%%+and atom labels, Other choices are 'ANGLE a b c', 'TORSION a b c d'
%%+and 'OUTOFPL a b c d', where a--d are atom labels.
Specification of which internal coordinates that are to be used in the
calculation. Each subsequent line has the form '\mbox{BOND {\em a} {\em b}}'
or '\mbox{ANGLE {\em a} {\em b} {\em c}}' or
or '\mbox{TORSION {\em a} {\em b} {\em c} {\em d}}' or
or '\mbox{OUTOFPL {\em a} {\em b} {\em c} {\em d}}', for bond distances,
valence angles, torsions (e.g. dihedral angles), and out-of-plane angles.
Here, {\em a}\dots{\em d} stand for atom labels.
After these lines follows one line with the keyword "END of internal".
%---
\item[MODEs]
%%Keyword:MODEs <basic> <endable>
%%+Selection of modes to be used in the intensity calculation.
Selection of modes to be used in the intensity calculation. This is
followed by a list of numbers, enumerating the vibrational modes to use.
The modes are numbered sequentially in order of vibrational frequency.
After this list follows one line with the keyword "END of modes".
%---
\item[MXLEvels]
%%Keyword:MXLEvels <basic>
%%+Followed by one line with max excitation level in the two states.
Followed by one line with
the maximum number of excitations in each of the two states.
%---
\item[VARIational]
%%Keyword:VARIational <basic>
%%+Make a variational calculation, nor harmonic approximation.
If this keyword is included, a variational calculation will be made,
instead of using the default double harmonic approximation.
%---
\item[TRANsitions]
%%Keyword:TRANsitions <basic>
%%+Followed by the word FIRST, then a line with a list of
%%+the number of phonons to be distributed among the modes,
%%+for the first state, then similarly for second state.
Indicates the excitations to be printed in the output.
Followed by the word FIRST on one line, then a list of numbers which
are the number of phonons -- the excitation level -- to be distributed
among the modes, defining the vibrational states of the first
potential function (force field). Then similarly, after a line with
the word SECOND, a list of excitation levels for the second state.
%---
\item[ENERgies]
%%Keyword:ENERgies <basic>
%%+The electronic T_0 energies of the two states, each value
%%+followed by "eV" or "au".
The electronic $T_0$ energies of the two states, each value is followed by
either "eV" or "au".
%---
%%Keyword:GEOMetry <basic>
%%+Geometry input follows. Next line is FILE, CARTESIAN, or INTERNAL.
%%+Followed by FIRST, then coordinates, then SECOND, then coordinates.
%%+Format: See User's Guide.
\item[GEOMetry]
Geometry input. Followed by keywords FILE, CARTESIAN, or INTERNAL.
If FILE, the geometry input is taken from UNSYM1 and UNSYM2.
If CARTESIAN or INTERNAL, two sections follow, one headed by a line
with the word FIRST, the other with the word SECOND. For the CARTESIAN
case, the following lines list the atoms and coordinates. On each line
is an atom label, and the three coordinates ($x,y,z$). For the INTERNAL
case, each line defines an internal coordinate in the same way as for
keyword INTERNAL, and the value.
%---
%%Keyword:MXORder <basic>
%%+Next line is 0 for constant transition dipol, 1 for linear function.
\item[MXORder]
Maximum order of transition dipole expansion. Next line is 0, if the
transition dipole is constant, 1 if it is a linear function, etc.
%---
\item[OSCStr]
%%Keyword:OSCStr <basic>
%%+Print oscillator strengths rather than intensities.
If this keyword is included, the oscillator strength, instead of the
intensity, of the transitions will calculated.
%---
\item[BROAdplot]
%%Keyword:BROAdplot <basic>
%%+Enter life time (sec) to be used for lifetime broadening of
%%+artificial spectrum.
Gives the peaks in the spectrum plot an artificial halfwidth. The default
lifetime is $130\cdot10^{-15}$ s but this can be changed with keyword
LIFEtime followd by the value.
%---
\item[NANOmeters]
%%Keyword:NANOmeters <basic>
%%+If this keyword is included, the plot file will be in nanometers.
%%+Default is in eV.
If this keyword is included, the plot file will be in nanometers.
Default is in eV.
%---
\item[CM-1]
%%Keyword:CM-1 <basic>
%%+If this keyword is included, the plot file will be in cm-1.
%%+Default is in eV.
If this keyword is included, the plot file will be in
cm\textsuperscript{-1}. Default is in eV.
%---
\item[PLOT]
%%Keyword:PLOT <basic>
%%+Enter the limits (in eV, cm-1, or in nm) for the plot file.
Enter the limits (in eV, cm\textsuperscript{-1}, or in nm) for the plot file.
%---
\item[VIBWrite]
%%Keyword:VIBWrite <basic>
%%+Print vibrational levels in the output.
If this keyword is included, the vibrational levels of the two states will
be printed in the output.
%---
\item[VIBPlot]
%%Keyword:VIBPlot <basic>
%%+Generate files plot.modes1 and plot.modes2 picturing normal modes.
Two files, plot.modes1 and plot.modes2, will be generated, with pictures of
the normal vibrational modes of the two electronic states.
%---
\item[HUGElog]
%%Keyword:HUGElog <basic>
%%+Much more detailed output.
This keyword will give a much more detailed output file.
%---
%\item[EXPANSION]
%This keyword indicates that the calculation will be aborted after
%the calculation of the expansion point.
%---
\item[SCALe]
%%Keyword:SCALe <basic>
%%+Enter scale factors that will multiply the Hessians.
Scales the Hessians, by multiplying with the scale factors following this keyword.
%---
\item[DIPOles]
%%Keyword:DIPOles <basic>
%%+Transition dipole data follows. A single line with x,y,z components,
%%+if MAXORDER=0. Else additional lines with gradient values.
Transition dipole data. If MXORDER=0 (see above), there follows a single line
with $x,y,z$ components of the transition dipole moment. If MXORDER=1 there
are an additional line for each cartesian coordinate of each atom, with the
derivative of the transition dipole moment w.r.t. that nuclear coordinate.
%---
\item[NONLinear]
%%Keyword:NONLinear <advanced>
%%+Specifies non-linear variable substitutions in definition of potential functions.
Specifies non-linear variable substitutions to be used in the definition of
potential surfaces.
%---
\item[POLYnomial]
%%Keyword:POLYnomial <advanced>
%%+Specifies which polynomial terms that are used in modeling potential functions.
Gives the different terms to be included in the fit of the polynomial
to the energy data.
%---
\item[DATA]
%%Keyword:DATA <basic>
%%+Grid data follows. See manual for format.
Potential energy surface data.

\end{keywordlist}


\subsubsection{Input example}

\begin{inputlisting}
 &MULA

Title
 Water molecule

Atoms
 O1
 H2
 H3
End Atoms

Internal Coordinates
 Bond  O1 H2
 Bond  O1 H3
 Angle H3 O1 H2
End Internal Coordinates

MxLevels
  0  3

Energies
 First
  0.0 eV
 Second
  3.78 eV

Geometry
 Cartesian
 First
  O1     0.0000000000      0.0000000000     -0.5000000000
  H2     1.6000000000      0.0000000000      1.1000000000
  H3    -1.6000000000      0.0000000000      1.1000000000
 End
 Second
  O1     0.0000000000      0.0000000000     -0.4500000000
  H2     1.7000000000      0.0000000000      1.0000000000
  H3    -1.7000000000      0.0000000000      1.0000000000
 End

ForceField
 First state
 Internal
  0.55 0.07 0.01
  0.07 0.55 0.01
  0.01 0.01 0.35
 Second state
 Internal
  0.50 0.03 0.01
  0.03 0.50 0.01
  0.01 0.01 0.25

DIPOles
  0.20 0.20 1.20

BroadPlot
LifeTime
 10.0E-15

NANO
PlotWindow
 260 305

End of input

\end{inputlisting}

\begin{inputlisting}
 &MULA

TITLe
 Benzene

ATOMs
  C1
  C2
  C3
  C4
  C5
  C6
  H1
  H2
  H3
  H4
  H5
  H6
End of Atoms

GEOMetry
 file

INTERNAL COORDINATES
 Bond    C1 C3
 Bond    C3 C5
 Bond    C5 C2
 Bond    C2 C6
 Bond    C6 C4
 Bond    C1 H1
 Bond    C2 H2
 Bond    C3 H3
 Bond    C4 H4
 Bond    C5 H5
 Bond    C6 H6
 Angle   C1 C3 C5
 Angle   C3 C5 C2
 Angle   C5 C2 C6
 Angle   C2 C6 C4
 Angle   H1 C1 C4
 Angle   H2 C2 C5
 Angle   H3 C3 C1
 Angle   H4 C4 C6
 Angle   H5 C5 C3
 Angle   H6 C6 C2
 Torsion C1 C3 C5 C2
 Torsion C3 C5 C2 C6
 Torsion C5 C2 C6 C4
 Torsion H1 C1 C4 C6
 Torsion H2 C2 C5 C3
 Torsion H3 C3 C1 C4
 Torsion H4 C4 C6 C2
 Torsion H5 C5 C3 C1
 Torsion H6 C6 C2 C5
END INTERNAL COORDINATES

VIBPLOT
 cyclic 4 1

ENERGIES
 First
  0.0 eV
 Second
  4.51 eV

MODES
 14 30 5 6 26 27 22 23 16 17 1 2 9 10
END

MXLE - MAXIMUM LEVEL of excitation (ground state - excited state)
  2 2

MXOR - MAXIMUM ORDER in transition dipole.
  1

OscStr

Transitions
 First
  0
 Second
  0 1 2

FORCEFIELD
 First
   file
 Second
   file

DIPOLES
 file

\end{inputlisting}
