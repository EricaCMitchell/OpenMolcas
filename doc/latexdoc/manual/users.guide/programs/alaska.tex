% alaska.tex $ this file belongs to the Molcas repository $

\section{\program{alaska}}
\label{UG:sec:alaska}
\index{Program!Alaska@\program{alaska}}\index{Alaska@\program{alaska}}
%%%<MODULE NAME="ALASKA">
%%Description:
%%%<HELP>
%%+This program computes the first derivatives of the one- and
%%+two-electron integrals with respect to the nuclear positions.
%%+The derivatives are not stored on files, but contracted
%%+immediately with the one- and two-electron densities to form the
%%+molecular gradients.
%%%</HELP>

This module is automatically invoked by the \program{Slapaf} module.
This is the preferred mode of operation! In connection with numerical gradients
it will ensure that the rotational and translational invariance is fully
utilized in order to reduce the number of used displacements.

The \program{Alaska}  module compute analytic or for numerical gradients requests the execution of
an alternative module.
The \program{Alaska} module figures out
the method automatically. Analytic methods are implemented for the HF, MBPT2, KS-DFT, and
RASSCF and SA-CASSCF method. Numerical methods are implemented for SCF, KS-DFT, RASSCF,
MBPT2, CCSDT, the CASPT2 and MS-CASPT2 methods, including the use of the Cholesky
decomposition for the methods were that has been implemented.

Both analytic and numerical procedures are parallelized.

For SA-CASSCF gradient the \program{Alaska} module will automatically
start up the \program{MCLR} module if required.

\subsection{Analytic gradients}
Gradients of the energy with respect to nuclear coordinates can be computed for
any type of wave function as long as an effective first order density matrix, an
effective Fock matrix, and an effective second order density matrix is provided.
The term effective is related to that
these matrices in the case of non-variational parameters in the wave function
(e.q. CI, MP2, CASPT2, etc.) are modified to include contributions from
the associated Lagrange
multipliers. The gradient expression apart from these modifications is
the same for any wave function type. \program{ALASKA}
is the gradient program, which will generate
the necessary integral derivatives and combine them with the matrices
mentioned in the text above.

\subsection{Description}
\label{UG:sec:alaska_description}
\program{ALASKA} is written such that gradients can be
computed for any kind of basis function that \program{SEWARD} will accept.

\program{ALASKA} is able to compute the following integral derivatives:

\begin{itemize}
\item
overlap integrals,
\item
kinetic energy integrals,
\item
nuclear attraction integrals (point charges or finite nuclei),
\item
electron repulsion integrals,
\item
external electric field integrals,
\item
ECP and PP integrals,
\item
reaction field integrals,
\item
and Pauli repulsion integrals.
\end{itemize}


\program{ALASKA} employs
two different integration schemes
to generate the
one- and two-electron integral derivatives.
The nuclear attraction and electron repulsion
integrals are evaluated by a modified Rys-Gauss quadrature \cite{Alaska}.
All other integral
derivatives are evaluated with the Hermite-Gauss quadrature. The same
restriction of the basis sets applies as to \program{SEWARD}.
None of the integral derivatives are written to disk but rather combined
immediately with the corresponding matrix from the wave function.

At present the following limitations are built into \program{ALASKA}:

% limitations.tex $ this file belongs to the Molcas repository $*/
\bigskip\centerline{
\begin{tabular}{|lr|} \hline
{Max number of unique basis functions:}&{2000} \\
{Max number of symmetry independent centers:}\hfill&{500} \\
{Highest angular momentum:}&{15} \\
{Highest symmetry point group:}&{$D_{2h}$} \\ \hline
\end{tabular}}\bigskip


\subsection{Numerical gradients}
The module is parallelized over the displacements, which in case of large jobs gives a linear
speed up compared to a serial execution, although in order to obtain this it is important to
choose the number of nodes such that the number of contributing perturbations is a multiple of
the number of nodes. For a given molecule the number of perturbations equals the number of atoms
times 6 (a perturbation with plus and minus delta for each of the three axes). Symmetry can of
course reduce this number.


\subsection{Dependencies}
\label{UG:sec:alaska_dependencies}
\program{ALASKA} depends on the density and Fock matrices generated by
\program{SCF} or \program{RASSCF}. In addition it needs the basis set
specification defined in \program{SEWARD}.
The dependencies of the numerical part of the module is the union
of the dependencies of the \program{SEWARD},
                           \program{SCF},
                           \program{RASSCF},
                           \program{MBPT2},
                           \program{MOTRA},
                           \program{CCSDT}, and
                           \program{CASPT2}
modules.
All these dependencies, however, are totally transparent to the user.


\subsection{Files}
\label{UG:sec:alaska_files}
\subsubsection{Input files}
Apart from the standard input unit \program{ALASKA}
will use the following input
files: \file{RYSRW}, \file{ABDATA}, \file{ONEINT}, \file{RUNFILE}
(for more information see~\ref{UG:sec:files_list}).


The files of the \program{SEWARD},
                           \program{SCF},
                           \program{RASSCF},
                           \program{MBPT2},
                           \program{MOTRA},
                           \program{CCSDT}, and
                           \program{CASPT2}
modules are needed for the numerical procedure.



\subsubsection{Output files}
In addition to the standard output unit \program{ALASKA} will generate the following
files.

\begin{filelist}
\item[RUNFILE]
The \file{runfile} is updated with information needed by the \program{SLAPAF}
relaxation program.
\program{ALASKA} will write the molecular Cartesian gradients on this file.
\item[ALASKA.INPUT]
File with the latest input processed by \program{ALASKA}.
\end{filelist}

\subsection{Input}
\label{UG:sec:alaska_input}
Below follows a description of the input to \program{ALASKA}.
Note that input options are related to the analytic gradient procedure if
not otherwise noted!

In addition to the keywords and the comment lines the input may contain blank
lines. The input is always preceded by the program name:
\begin{inputlisting}
 &ALASKA
\end{inputlisting}

Optional keywords for analytical gradients
\begin{keywordlist}
%---
\item[TEST]
%%%<KEYWORD MODULE="ALASKA" NAME="TEST" APPEAR="Test" KIND="SINGLE" LEVEL="BASIC">
%%Keyword: Test <basic>
%%%<HELP>
%%+With this keyword the program will process only the input.
%%+It is a debugging aid to help you check your input.
%%%</HELP></KEYWORD>
With this keyword the program will process only the input.
It is a debugging aid to help you check your input.
%---
\item[NAC]
%%%<KEYWORD MODULE="ALASKA" NAME="NAC" APPEAR="Nonadiabatic coupling" KIND="INTS" SIZE="2" LEVEL="BASIC" MIN_VALUE="1">
%%Keyword: NAC <basic>
%%%<HELP>
%%+Requests a calculation of the nonadiabatic coupling vector between the
%%+two specified roots in a SA-CASSCF calculation.
Requests a calculation of the nonadiabatic coupling vector between the
two specified roots in a SA-CASSCF calculation. If the roots are $i$,
$j$, the vector computed will be $\langle\Psi_j|\nabla\Psi_i\rangle$.
%%%</HELP></KEYWORD>
%---
\item[NOCSF]
%%%<KEYWORD MODULE="ALASKA" NAME="NOCSF" APPEAR="Do not compute CSF contribution" KIND="SINGLE" LEVEL="ADVANCED" REQUIRE="NAC">
%%Keyword: NOCSF <advanced>
%%%<HELP>
%%+In a NAC calculation, neglects the so-called CSF contribution.
In a NAC calculation, neglects the so-called CSF contribution.
Note that this contribution is responsible for the translational and
rotational non-invariance, and it has been suggested that not including
it may give more physical results in dynamics simulations \cite{Fatehi2012}.
%%%</HELP></KEYWORD>
%---
\item[ONEOnly]
%%%<KEYWORD MODULE="ALASKA" NAME="ONEONLY" APPEAR="One-electron integrals only" KIND="SINGLE" LEVEL="ADVANCED">
%%Keyword: Oneonly <advanced>
%%%<HELP>
%%+Compute only the nuclear repulsion and one-electron integrals
%%+contribution to the gradient. The default is to compute all
%%+contributions to the molecular gradient.
Compute only the nuclear repulsion and one-electron integrals
contribution to the gradient. The default is to compute all
contributions to the molecular gradient.
%%%</HELP></KEYWORD>
%---
\item[CUTOff]
%%%<KEYWORD MODULE="ALASKA" NAME="CUTOFF" APPEAR="Gradient threshold" KIND="REAL" MIN_VALUE="0.0" DEFAULT_VALUE="1.0D-7" LEVEL="BASIC">
%%Keyword: Cutoff <advanced>
%%%<HELP>
%%+Specify the threshold for ignoring contributions to the molecular gradient.
%%+The prescreening
%%+is based on the 2nd order density matrix and the radial
%%+overlap contribution to the integral derivatives.
%%%</HELP>
%%+The default is 1.0d-7.
Threshold for ignoring contributions to the molecular gradient
follows on the next line. The default is 1.0d-{}7. The prescreening
is based on the 2nd order density matrix and the radial
overlap contribution to the integral derivatives.
%%%</KEYWORD>
%---
\item[OFEMbedding]
%%%<GROUP MODULE="ALASKA" NAME="OFEMBED" APPEAR="Orbital-free embedding" KIND="BOX" WINDOW="POPUP" LEVEL="ADVANCED">
%%Keyword: OFEM <advanced>
%%%<HELP>
%%+Orbital-Free Embedding gradient calculation, available only in combination with Cholesky or RI integral representation.
%%+The runfile of the environment subsystem renamed AUXRFIL is required.
%%%</HELP>
%%%<KEYWORD MODULE="ALASKA" NAME="OFEM" APPEAR="Functionals" KIND="STRING" LEVEL="ADVANCED">
%%%<HELP>
%%+The keyword OFEM requires the specification of two functionals in the form fun1/fun2
%%+(see the manual for available functionals)
%%%</HELP></KEYWORD>
Performs a Orbital-Free Embedding gradient calculation, available only in combination with Cholesky or RI integral representation.
The runfile of the environment subsystem renamed AUXRFIL is required.
An example of input for the keyword \keyword{OFEM} is the following:
\begin{inputlisting}

OFEMbedding
 ldtf/pbe
dFMD
 1.0

\end{inputlisting}
(see the OPTIONAL keyword :kword:`DFMD` below).
The keyword \keyword{OFEM} requires the specification of two functionals in the form fun1/fun2, where fun1 is the functional
used for the Kinetic Energy (available functionals: Thomas-Fermi, with acronym LDTF, and the NDSD functional), and where
fun2 is the xc-functional (LDA, LDA5, PBE and BLYP available at the moment).
%---
%%%<KEYWORD MODULE="ALASKA" NAME="DFMD" APPEAR="Fraction of correlation" KIND="REAL" LEVEL="ADVANCED" DEFAULT_VALUE="0.0">
%%Keyword: DFMD <advanced>
%%%<HELP>
%%+In combination with OFEM, specifies the fraction of correlation potential to be added to the OFE potential
%%+(zero for KSDFT and one for HF).
%%%</HELP></KEYWORD>
%%%</GROUP>
\item[DFMD]
In combination with \keyword{OFEM}, specifies the fraction of correlation potential to be added to the OFE potential (zero for
KSDFT and one for HF).
%---
%\item[NOINvariance]
%No utilization of the rotational and translational invariance
%of the energy. This is the default.
%---
%\item[EQUIvalence]
%This option is used to indicate that some of the gradients have
%the same magnitude and only one has to be computed. This line
%is followed by a line with
%\verb+nGroup+
%being the number of different
%groups that are equivalent. Then on
%\verb+nGroup+
%subsequent lines follow:\\
%\verb+nElem,(index(iElem), iElem = 1, nElem)+\\
%where \verb+nElem+ is the
%number of equivalent displacements and index is the index of
%such a displacement. This option will disable the automatic
%utilization of the translational and rotational energy.
%---
%\item[SELEction]
%This option will allow the user to exclude some symmetrical
%displacements from the list of gradients to compute. This card
%is followed by a line specifying the number of gradients which
%will be computed. A second additional line contains all indices
%of those symmetrical displacements for which we will compute
%gradients. This option will disable the automatic utilization
%of the translational and rotational invariance of the energy.
%The \keyword{Selection} option can be used together with the
%\keyword{Equivalence} option, however,
%for this to work the \keyword{Selection} option has to  be specified first.
%---
%\item[2DOPrescreening]
%This option will activate prescreening based on the 2nd order
%density matrix only. The default prescreening method is the 2DI
%approach which is based on the 2nd order density matrix and
%bounded estimates of the integral gradient.
%---
%\item[2DIPrescreening]
%---
%\item[PRINt]
%---
%\item[NOTRiangular]
%---
%%%<GROUP MODULE="ALASKA" NAME="PRINT" APPEAR="Print options" KIND="BOX" WINDOW="POPUP" LEVEL="BASIC">
%%%<SELECT MODULE="ALASKA" NAME="PRINTF" APPEAR="Print format" CONTAINS="POLD,PNEW">
\item[POLD]
%%%<KEYWORD MODULE="ALASKA" NAME="POLD" APPEAR="Old format" KIND="SINGLE" LEVEL="BASIC"
%%% MEMBER="PRINT" EXCLUSIVE="PNEW">
%%Keyword: POLD <basic>
%%%<HELP>
%%+The gradient is printed in the old format. Note: by default gradient
%%+is not printed any longer.
The gradient is printed in the old format. Note: by default gradient
is not printed any longer.
%%%</HELP></KEYWORD>
%---
\item[PNEW]
%%%<KEYWORD MODULE="ALASKA" NAME="PNEW" APPEAR="New format" KIND="SINGLE" LEVEL="BASIC"
%%%  MEMBER="PRINT" EXCLUSIVE="POLD">
%%Keyword: PNEW <basic>
%%%<HELP>
%%+The gradient is printed in the new human-readable format.
The gradient is printed in the new human-readable format.
%%%</HELP></KEYWORD>
%%%</SELECT>
%---
\item[VERBose]
%%%<KEYWORD MODULE="ALASKA" NAME="VERBOSE" APPEAR="Verbose printout" KIND="SINGLE"
%%%  MEMBER="PRINT" LEVEL="BASIC">
%%Keyword: Verbose <basic>
%%%<HELP>
%%+The output will be a bit more verbose.
The output will be a bit more verbose.
%%%</HELP></KEYWORD>
%---
%%%<KEYWORD MODULE="ALASKA" NAME="SHOW" APPEAR="Show contributions" KIND="SINGLE"
%%%  MEMBER="PRINT" LEVEL="BASIC">
\item[SHOW gradient contributions]
%%Keyword: Show <basic>
%%%<HELP>
%%+The gradient contributions will be printed.
The gradient contributions will be printed.
%%%</HELP></KEYWORD>
%%%</GROUP>
%---
\end{keywordlist}

Optional keywords for numerical gradients
\begin{keywordlist}
%---
\item[NUMErical]
%%%<GROUP MODULE="ALASKA" NAME="NUMGRAD" APPEAR="Numerical gradient" KIND="BOX" WINDOW="POPUP" LEVEL="BASIC">
%%%<KEYWORD MODULE="ALASKA" NAME="NUMERICAL" APPEAR="Force numerical gradient" KIND="SINGLE" MEMBER="NUMGRAD" LEVEL="BASIC">
%%Keyword: Numerical <basic>
%%%<HELP>
%%+Forces the use of numerical gradients even if analytical ones
%%+are implemented. The default is to use analytical gradients whenever
%%+possible.
Forces the use of numerical gradients even if analytical ones
are implemented. The default is to use analytical gradients whenever
possible.
%%%</HELP></KEYWORD>
%---
\item[ROOT]
%%%<KEYWORD MODULE="ALASKA" NAME="ROOT" APPEAR="Root selection" KIND="INT" MEMBER="NUMGRAD" LEVEL="ADVANCED" MIN_VALUE="1">
%%Keyword: Root <advanced>
%%%<HELP>
%%+Specifies which root to compute the gradient the geometry for, if there is more than
%%+one root to choose from. In a RASSCF optimization, the default is to
%%+compute the gradient for the same root as is relaxed. In a MS-CASPT2 calculation, the
%%+default is to compute it for root 1.
Specifies which root to compute the gradient the geometry for, if there is more than
one root to choose from. In a RASSCF optimization, the default is to
compute the gradient for the same root as is relaxed. In a MS-CASPT2 calculation, the
default is to compute it for root 1. It can be used to override the default
root in an analytical calculation too.
%%%</HELP></KEYWORD>
%---
\item[DELTa]
%%%<KEYWORD MODULE="ALASKA" NAME="DELTA" APPEAR="Displacement value" KIND="REAL" MEMBER="NUMGRAD" LEVEL="ADVANCED" DEFAULT_VALUE="0.01" MIN_VALUE="0.0">
%%Keyword: Delta <advanced>
%%%<HELP>
%%+For use with numerical gradients only!
%%+The displacement for a given center is chosen as the distance to the nearest
%%+neighbor scaled by a factor. This factor can be set through the DELTa
%%+keyword. The default is 0.01.
For use with numerical gradients only!
The displacement for a given center is chosen as the distance to the nearest
neighbor, scaled by a factor. This factor can be set through the \keyword{DELTa}
keyword. The default value is $0.01$.
%%%</HELP></KEYWORD>
%---
%%%<KEYWORD MODULE="ALASKA" NAME="KEEP" APPEAR="Keep old gradient" KIND="SINGLE" LEVEL="ADVANCED">
%%Keyword: KeepOldGradient <advanced>
%%%<HELP>
%%+Keep the existing gradient for constrained coordinates when doing numerical differentiation.
\item[KEEPOldGradient]
When computing numerical gradients with constraints, the gradient of the constrained degrees
of freedom is normally set to zero. If this keyword is specified, the existing value of the gradient
(probably computed analytically with a different method) is maintained instead.
This is used, in combination with \keyword{NGEXclude} in \program{Gateway} (or ``phantom''
constraints), to set up composite gradients \cite{Stenrup2015}.
%%%</HELP></KEYWORD>
%%%</GROUP>
%---
\end{keywordlist}

The following is an example of an input which will work for
almost all practical cases. Note that it is very rarely that you need to run
this program explicitly. It is usually controlled by the program
\program{Slapaf}.

\begin{inputlisting}
 &Alaska
\end{inputlisting}
%%%</MODULE>
