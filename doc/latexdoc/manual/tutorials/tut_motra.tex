% tut_motra.tex $ this file belongs to the Molcas repository $*/
\section{MOTRA --- An Integral Transformation Program}
\label{TUT:sec:motra}
\index{Program!MOTRA}\index{MOTRA}
\index{Integrals!Integral transformation}
Integrals saved by the \program{SEWARD} module
are stored in the Atomic Orbital (AO) basis.  Some programs have their own
procedures to transform the integrals into the Molecular Orbital (MO) basis.
The \molcas\ \program{MOTRA} module performs this task for
Configuration Interaction (CI), Coupled- and Modified Coupled-Pair (CPF and
MCPF, respectively) and Coupled-Cluster (CC) calculations.

The sample input below contains the \program{motra} input
information for our continuing water calculation.  We firstly specify that the
\program{RASSCF} module interface file will be the source of the
orbitals using the keyword \keyword{JOBIph}.  The keyword
\keyword{FROZen} is used to specify the number of orbitals  in each
symmetry which will not be correlated in
subsequent calculations.  This can also be performed in the corresponding
\program{MRCI}, \program{CPF} or CC programs
but is more efficient to freeze them here.  
Virtual orbitals can be deleted using the \keyword{DELEte} keyword.

\begin{inputlisting}
 &MOTRA
JobIph
Frozen= 1 0 0 0
\end{inputlisting}

\subsection{\program{motra} Output}

The \program{motra} section of the output is short and self
explanatory.  The integral files produced by \program{SEWARD}, \file{ONEINT} 
and \file{ORDINT}, are used as input by the
\program{MOTRA} module which produces the transformed symbolic files
\file{TRAONE} and \file{TRAINT}, respectively. In our case, the files
are called \file{water.TraOne} and \file{water.TraInt}, respectively.
 

The \program{motra} module also requires input orbitals.
If the \keyword{LUMOrb} keyword is specified the orbitals are taken
from the \file{INPORB} file which can be any formated orbital
file such as \file{water.ScfOrb} or \file{water.RasOrb}.  The 
\keyword{JOBIph} keyword causes the \program{MOTRA} module to
read the required orbitals from the \file{JOBIPH} file.

\subsection{MOTRA - Basic and Most Common Keywords}
\begin{keywordlist}
\item[FROZEN] By symmetry: non-correlated orbitals (default: core)
\item[RFPErt] Previous reaction field introduced as a perturbation
\item[LUMORB] Input orbital file as ASCII (INPORB)
\item[JOBIPH] Input orbital file as binary (JOBOLD)
\item[]

%--
\end{keywordlist}

