% mclr.tex $ this file belongs to the Molcas repository $*/
\section{MCLR --- A Program for Linear Response Calculations}
\label{TUT:sec:mclr} 
\index{MCLR}\index{Program!MCLR}

\program{MCLR} computes response calculations on single and multiconfigurational
SCF wave functions. One of the basic uses of \program{MCKINLEY} and \program{MCLR}
is to compute analytical hessians (vibrational frequencies, IR intensities, etc).
\program{MCLR} can also calculate the Lagrangian multipliers for
a MCSCF state included in a state average optimization and construct the effective
densities required for analytical gradients of such a state.
The use of keyword \keyword{RLXRoot} in the \program{RASSCF} program is required.
In both cases the explicit request of executing the \program{MCLR} module is not
required and will be automatic.
We postpone further
discussion about \program{MCLR} to section~\ref{TUT:sec:structure}.

It follows an example of how to optimize an excited state from a previous
State-Average (SA) CASSCF calculation. 

%%%To_extract{/doc/samples/tutorials/MCLR.acrolein.input}
\begin{inputlisting}
 &GATEWAY
Title= acrolein minimum optimization in excited state 2
Coord=$MOLCAS/Coord/Acrolein.xyz
*$
Basis= sto-3g
Group=NoSym
>>> Do while
 &SEWARD
 &RASSCF
Title= acrolein
Spin= 1; nActEl= 6 0 0; Inactive= 12; Ras2= 5
CiRoot= 3 3 1
Rlxroot= 2
 &SLAPAF
>>> EndDo
\end{inputlisting}
%%%To_extract

%$
The root selected for optimization has been selected here with the keyword
\keyword{Rlxroot} in \program{RASSCF}, but it is also possible to select it
with keyword \keyword{SALA} in \program{MCLR}.

Now if follows an example as how to compute the analytical hessian for the lowest
state of each symmetry in a CASSCF calculation (SCF, DFT, and RASSCF analytical
hessians are also available).

%%%To_extract{/doc/samples/tutorials/MCLR.benzoquinone.input}
\begin{inputlisting}
 &GATEWAY
Title=p-benzoquinone anion. Casscf optimized geometry.
Coord = $MOLCAS/Coord/benzoquinone.xyz
Basis= sto-3g
Group= X Y Z
 &SEWARD
 &RASSCF
TITLE=p-benzoquinone anion. 2B3u state.
SYMMETRY=2; SPIN=2; NACTEL=9 0 0
INACTIVE=8  0  5  0  7  0  4  0
RAS2    =0  3  0  1  0  3  0  1

 &MCKINLEY; Perturbation=Hessian

\end{inputlisting}
%%%To_extract

The \program{MCLR} is automatically called after \program{MCKINLEY}
and it is not needed in the input.

\subsection{MCLR program - Basic and Most Common Keywords}
\begin{keywordlist}
\item[SALA] Root to relax in geometry optimizations
\item[ITER] Number of iterations
\item[]

%--
\end{keywordlist}

