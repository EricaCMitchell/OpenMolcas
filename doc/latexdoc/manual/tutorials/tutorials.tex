% tutorials.tex $ this file belongs to the Molcas repository $*/

\index{Tutorials}
%! tutorials.tex $ this file belongs to the Molcas repository $

\chapter{Problem Based Tutorials}

\index{Problem Based Tutorials}

\section{Electronic Energy at Fixed Nuclear Geometry}

The \molcas\ \molcasversion\ suite of Quantum Chemical programs is modular in
design, and a desired calculation is achieved by executing a list of
\molcas\ program modules in succession, occasionally manipulating
the program information files. If the information files from a previous
calculation are saved, then a subsequent calculation need not recompute
them.  This is dependent on the correct information being preserved in
the information files for the subsequent calculations. Each module has keywords 
to specify the functions to be carried out, and many modules rely on the
specification of keywords in previous modules.

In the present examples the calculations will be designed by preparing
a single file in which the input for the different programs is presented
sequentially. The initial problem will be to compute an electronic energy
at a fixed geometry of the nuclei, and this will be performed using different
methods and thus requiring different \molcas\ program modules.

First, the proper \molcas\ environment has to be set up which requires that 
following variables must be properly defined, for instance:

\begin{inputlisting}
export MOLCAS=/home/molcas/molcas.\molcasversion
export Project=CH4
export WorkDir=/home/user/tmp
\end{inputlisting}

If not defined, \molcas\ provides default values for the above environment variables:
\begin{itemize}
\item The {MOLCAS} variable will be set to the latest implemented version of the code.

This variable is set directly in the \molcas\ home directory 

\item Project and WorkDir have the default values None and \$PWD, respectively.

It is very important that the molcas driver, called by command \command{molcas},
and built during the installation of the code, is included in the \$PATH.
\end{itemize}

The first run involves a calculation of the SCF energy of the methane
(CH$_4$) molecule. Three programs should be used: \program{GATEWAY} to specify 
information about the system, \program{SEWARD} to compute
and store the one- and two-electron integrals, and \program{SCF} to obtain
the Hartree-Fock SCF wave function and energy.  

The three \molcas\ programs to 
be used leads to three major entries in the input file: \program{GATEWAY}, \program{SEWARD}, and \program{SCF}.
The \program{GATEWAY} program contains the nuclear geometry in cartesian
coordinates and the label for the one-electron basis set.
The keyword \keyword{coord} allows automatic insertion of \program{GATEWAY} input from a standard
file containing the cartesian coordinates in Angstrom which can be generated by
programs like \program {LUSCUS} or \program{MOLDEN}). 
No symmetry is being considered so the keyword \keyword{group=C1} is used to force the program not 
to look for symmetry in the CH$_4$ molecule, and ,thus, input for \program{SEWARD} is not required. 
In closed-shell cases, like CH$_4$, input for \program{SCF} is not required. All the input
files discussed here can be found at {$\$MOLCAS/doc/samples/problem\_based\_tutorials$}, including the file
\file{SCF.energy.CH4} described below.

%%%To_extract{/doc/samples/problem_based_tutorials/SCF.energy.CH4.input}
\begin{inputlisting}
*SCF energy for CH4 at a fixed nuclear geometry.
*File: SCF.energy.CH4
*
&GATEWAY
 Title = CH4 molecule
 coord  = CH4.xyz 
 basis = STO-3G 
 group = C1

&SEWARD                                                                                                                                                                             
&SCF
\end{inputlisting}
%%%To_extract

where the content of the \file{CH4.xyz} file is: 

%%%To_extract{/doc/samples/problem_based_tutorials/CH4.xyz}
\begin{inputlisting}
5
distorted CH4 coordinates in Angstroms 
C    0.000000     0.000000     0.000000
H    0.000000     0.000000     1.050000
H    1.037090     0.000000    -0.366667
H   -0.542115    -0.938971    -0.383333
H   -0.565685     0.979796    -0.400000
\end{inputlisting}
%%%To_extract

To run \molcas\ , simply execute the command

\begin{inputlisting}
molcas SCF.energy.CH4.input > SCF.energy.CH4.log 2 > SCF.energy.CH4.err
\end{inputlisting}
where the main output is stored in file \file{SCF.energy.CH4.log} 

or

\begin{inputlisting}
molcas -f SCF.energy.CH4.input
\end{inputlisting}
where the main output is stored in \file{SCF.energy.CH4.log}, and the default error file in \file{SCF.energy.CH4.err}. 

The most relevant information is contained in the output file, where the \program{GATEWAY} program 
information describing the nuclear geometry, molecular symmetry, and the data 
regarding the one-electron basis sets and the calculation of one- and 
two-electron integrals, as described in section~\ref{TUT:sec:seward}. Next,
comes the output of program \program{SCF} with information of the electronic
energy, wave function, and the Hartree-Fock (HF) molecular orbitals 
(see section~\ref{TUT:sec:scf}).

Files containing intermediate information, integrals, orbitals, etc, will be 
kept in the {\$WorkDir} directory for further use. For instance, files
\file{\$Project.OneInt} and \file{\$Project.OrdInt} contain the one- and
two-electron integrals stored in binary format. File \file{\$Project.ScfOrb}
stores the HF molecular orbitals in ASCII format, and 
\file{\$Project.RunFile} is a communication file between programs. All these 
files can be used later for more advanced calculations avoiding a 
repeat of certain calculations.

There are graphical utilities that can be used for the analysis of the
results. By default, \molcas\ generates files which can be read with the 
\program{MOLDEN} program and are found in the {\$WorkDir} including the file{CH4.scf.molden}. 
This file contains information about molecular geometry and molecular orbitals, and requires the use if \textit{Density Mode} in \program{MOLDEN}.
However, \molcas\ has its own graphical tool, program \program{LUSCUS}, which is a viewer based on openGL and allows the visualization of 
molecular geometries, orbitals, densities, and density differences. For 
example, a graphical display of the CH$_4$ molecule can be obtained from a standard coordinate file by the following command:

\begin{inputlisting}
luscus CH4.xyz 
\end{inputlisting}

In order to obtain the information for displaying molecular orbitals and densities,
it is necessary to run the \molcas\ program called \program{GRID\_IT}:

%%%To_extract{/doc/samples/problem_based_tutorials/SCF.energy_grid.CH4.input}

\begin{inputlisting}
*SCF energy for CH4 at a fixed nuclear geometry plus a grid for visualization.
*File: SCF.energy_grid.CH4
*
&GATEWAY
 Title = CH4 molecule
 coord = CH4.xyz 
 basis = STO-3G 
 Group = C1

&SEWARD; &SCF                                                                                                                                                                            

&GRID_IT 
 All
\end{inputlisting}
%%%To_extract

Now, execcute the \molcas\ program:

\begin{inputlisting}
molcas SCF.energy_grid.CH4.input -f 
\end{inputlisting}

In the {\$WorkDir} and {\$PWD} directories a new file is generated, \file{CH4.lus} which
contains the information required by the \program{GRID\_IT} input. The file can 
be visualized by \program{LUSCUS} (Open source program, which can be downloaded and 
installed to your Linux, Windows, or MacOS workstation or laptop). By typing the command:

\begin{inputlisting}
luscus CH4.lus
\end{inputlisting}

a window will be opened displaying the molecule and its charge density. By proper
selection of options with the mouse buttons, the shape and size of several molecular orbitals
can be visualized.

\program{GRID\_IT} can also be run separately, if an orbital file is specified in
the input, and the {\$WorkDir} directory is available.

More information can be found in section \ref{UG:sec:gridit}.

As an alternative to running a specific project, the short script provided below can be placed
in the directory {\$MOLCAS/doc/samples/problem\_based\_tutorials} with the name \file{project.sh}.
Simply execute the shell script, \command{project.sh \$Project}, where {\$Project} is the {MOLCAS} input,
and output files, error files, and a {\$WorkDir} directory called {\$Project.work} will be obtained.

%%%To_extract{/doc/samples/problem_based_tutorials/project.sh}
\begin{inputlisting}
#!/bin/bash
                                                                                                                                                                            
export MOLCAS=$PWD
export MOLCAS_DISK=2000
export MOLCAS_MEM=64
export MOLCAS_PRINT=3
                                                                                                                                                                            
export Project=$1
export HomeDir=$MOLCAS/doc/samples/problem_based_tutorials
export WorkDir=$HomeDir/$Project.work
mkdir $WorkDir 2>/dev/null
molcas $HomeDir/$1 >$HomeDir/$Project.log 2>$HomeDir/$Project.err
exit
\end{inputlisting}
%%%To_extract

In order to run a Kohn-Sham density functional calculation, \molcas\ uses the 
same \program{SCF} module, and, therefore, the only change needed are the specification 
of the DFT option and required functional (e.g. B3LYP) in the \program{SCF} input:

%%%To_extract{/doc/samples/problem_based_tutorials/DFT.energy.CH4.input}
\begin{inputlisting}
*DFT energy for CH4 at a fixed nuclear geometry plus a grid for visualization.
*File: DFT.energy.CH4
*
&GATEWAY
 Title = CH4 molecule
 coord = CH4.xyz 
 basis = STO-3G 
 group = C1
&SEWARD
&SCF 
 KSDFT = B3LYP
&GRID_IT 
 All
\end{inputlisting}
%%%To_extract

Similar graphical files can be found in \$WorkDir and \$PWD.

The next step is to obtain  the second-order M{\o}ller--Plesset perturbation (MP2)
energy for methane at the same molecular geometry using the same one-electron
basis set. Program \program{MBPT2} is now used, and it is possible to take 
advantage of having previously computed the proper integrals with \program{SEWARD}
and the reference closed-shell HF wave function with the \program{SCF} program.
In such cases, it is possible to keep the same definitions as before and simply prepare a file 
containing the \program{MBPT2} input and run it using the \command{molcas}
command.

The proper intermediate file will be already in \$WorkDir.
On the other hand, one has to start from scratch, all required inputs should
be placed sequentially in the \file{MP2.energy.CH4} file.
If the decision is to start the project from the beginning,  it is probably a good idea to remove
the entire {\$WorkDir} directory, unless it is known for certain the exact nature of the files contained in this directory.


%%%To_extract{/doc/samples/problem_based_tutorials/MP2.energy.CH4.input}
\begin{inputlisting}
*MP2 energy for CH4 at a fixed nuclear geometry.
*File: MP2.energy.CH4
*
&GATEWAY
 Title = CH4 molecule
 coord = CH4.xyz 
 basis = STO-3G 
 group = C1
&SEWARD 
&SCF
&MBPT2 
 Frozen = 1
\end{inputlisting}
%%%To_extract

In addition to the calculation of a HF wave function, an MP2 calculation has been performed with 
a frozen deepest orbital, the carbon 1s, of CH$_4$. Information about the output
of the \program{MBPT2} program can be found on section~\ref{TUT:sec:mbpt2}.

The \program{SCF} program works by default with closed-shell systems with an
even number of electrons at the Restricted Hartee-Fock (RHF) level. If, 
instead there is a need to use the Unrestricted Hartree Fock (UHF) method, this can be schieved by invoking the
keyword \keyword{UHF}. This is possible for both even and odd electron systems. 
For instance, in a system with an odd number of electrons such as the CH$_3$ radical, with the 
following Cartesian coordinates

%%%To_extract{/doc/samples/problem_based_tutorials/CH3.xyz}
\begin{inputlisting}
4
CH3 coordinates in Angstrom 
C    0.000000     0.000000     0.000000
H    0.000000     0.000000     1.050000
H    1.037090     0.000000    -0.366667
H   -0.542115    -0.938971    -0.383333
\end{inputlisting}
%%%To_extract

the input to run an open-shell UHF calculation is easily obtained:

%%%To_extract{/doc/samples/problem_based_tutorials/SCF.energy_UHF.CH3.input}
\begin{inputlisting}
*SCF/UHF energy for CH3 at a fixed nuclear geometry
*File: SCF.energy_UHF.CH3
*
&GATEWAY
 Title = CH3 molecule
 coord = CH3.xyz 
 basis = STO-3G 
 group = C1
&SEWARD
&SCF 
 UHF
\end{inputlisting}
%%%To_extract

If the system is charged, this must be indicated in the 
\program{SCF} input, for example,  by computing the cation of the CH$_4$ molecule 
at the UHF level:

%%%To_extract{/doc/samples/problem_based_tutorials/SCF.energy_UHF.CH4plus.input}
\begin{inputlisting}
*SCF/UHF energy for CH4+ at a fixed nuclear geometry
*File: SCF.energy_UHF.CH4plus
*
&GATEWAY
 Title = CH4+ molecule
 coord = CH4.xyz 
 basis = STO-3G 
 group = c1
&SEWARD
&SCF
 UHF
 Charge = +1
\end{inputlisting}
%%%To_extract

The Kohn-Sham DFT calculation can be also run using the UHF algorithm:

%%%To_extract{/doc/samples/problem_based_tutorials/DFT.energy.CH4plus.input}
\begin{inputlisting}
*DFT/UHF energy for CH4+ at a fixed nuclear geometry
*File: DFT.energy.CH4plus
*
&GATEWAY
 Title = CH4+ molecule
 coord = CH4.xyz 
 basis = STO-3G 
 group = C1
&SEWARD
&SCF 
 KSDFT = B3LYP
 UHF
 Charge = +1
\end{inputlisting}
%%%To_extract

For the UHF and UHF/DFT methods it is also possible to specify 
$\alpha$ and $\beta$ orbital occupations in two ways. 
\begin{enumerate}
\item First, the keyword \keyword{ZSPIn} can be invoked in the \program{SCF} program, which represents the 
difference between the number of $\alpha$ and $\beta$ electrons. 

For example, setting the keyword to 2 forces the program to converge to a result with two more $\alpha$ than $\beta$ electrons.

%%%To_extract{/doc/samples/problem_based_tutorials/DFT.energy_zspin.CH4.input}
\begin{inputlisting}
*DFT/UHF energy for different electronic occupation in CH4 at a fixed nuclear geometry
*File: DFT.energy_zspin.CH4
*
&GATEWAY
 Title = CH4 molecule 
 coord = CH4.xyz 
 basis = STO-3G 
 group = c1
&SEWARD
&SCF
 Title = CH4 molecule zspin 2
 UHF; ZSPIN =  2
 KSDFT =  B3LYP
\end{inputlisting}
%%%To_extract

The final occupations in the output will show six $\alpha$ and four $\beta$ orbitals.

\item Alternatively, instead of \keyword{ZSPIn}, it is possible to specify 
occupation numbers with keyword \keyword{Occupation} at the beginning of the SCF calculation. 

This requires an additional input line containing the occupied $\alpha$ orbitals (e.g. 6 in this case), and a second line 
with the $\beta$ orbitals (e.g. 4 in this case). Sometimes, SCF convergence may be improved by using this option.
\end{enumerate}

Different sets of methods use other \molcas\ modules. For example, to perform a Complete
Active Space (CAS) SCF calculation, the \program{RASSCF} program has to be used. This
module requires starting trial orbitals, which can be obtained from a previous SCF
calculation or, automatically, from the \program{SEWARD} program which provides trial orbitals by
using a model Fock operator. 

Recommended keywords are 
\begin{itemize}
\item \keyword{Nactel} defines the total number of active 
electrons, holes in Ras1, and particles in Ras3, respectively.  The last two values 
are only for RASSCF-type calculations. 
\item \keyword{Inactive} indicates the number of inactive orbitals where the occupation is always 2 in the CASSCF reference, and 
\item \keyword{Ras2} defines the number of active orbitals. 

By default, the wave function for the lowest state corresponds to the symmetry with spin multiplicity of 1.
Most of the input may not be necessary, if one has prepared and linked an INPORB file with the different orbital types defined by 
a program like \program{LUSCUS}.
\end{itemize}

%%%To_extract{/doc/samples/problem_based_tutorials/CASSCF.energy.CH4.input}
\begin{inputlisting}
*CASSCF energy for CH4 at a fixed nuclear geometry
*File: CASSCF.energy.CH4
*
&GATEWAY
 coord = CH4.xyz
 basis = STO-3G
 group = C1
&SEWARD
&RASSCF
 Title = CH4 molecule
 Spin = 1; Nactel = 8 0 0; Inactive = 1; Ras2 = 8
&GRID_IT
 All
\end{inputlisting}
%%%To_extract

In this case, the lowest singlet state (i.e. the ground dstate) is computed, since this is a 
closed-shell situation with an active space of eight electrons in eight orbitals and 
with an inactive C 1s orbital, the lowest orbital of the CH$_4$ molecule. This is a CASSCF example in which all the valence 
orbitals and electrons (C 2s, C 2p and 4 x H 1s) are included 
in the active space and allows complete dissociation into
atoms. If this is not the goal, then the three almost degenerate 
highest energy occupied orbitals and the corresponding antibonding unoccupied orbitalsmust be active, leading to 
a 6 in 6 active space.

Using the CASSCF wave function as a reference, it is possible to perform a second-order 
perturbative, CASPT2, correction to the electronic energy by employing the 
\program{CASPT2} program. If all previously calculated files are retained in the 
\$WorkDir directory, in particular, integral files (\file{CH4.OneInt},\file{CH4.OrdInt}), 
the CASSCF wave function information file (\file{CH4.JobIph}), and communication file \file{CH4.RunFile}), it will not be 
necessary to re-run programs \program{SEWARD}, and \program{RASSCF}. In this case
case, it is enough to prepare a file containing input only for the \program{CASPT2} program followed be execution.
Here, however, for the sake of completness, input to all \molcas\ moddules is provided:

%%%To_extract{/doc/samples/problem_based_tutorials/CASPT2.energy.CH4.input}
\begin{inputlisting}
*CASPT2 energy for CH4 at a fixed nuclear geometry
*File: CASPT2.energy.CH4
*
&GATEWAY
 coord = CH4.xyz; basis = STO-3G; group = C1
&SEWARD
&RASSCF
   LumOrb
   Spin = 1; Nactel = 8 0 0; Inactive = 1; Ras2 = 8
&CASPT2
 Multistate = 1 1
\end{inputlisting}
%%%To_extract

In most of casesi, the Hartree-Fock orbitals will be a better choice as starting orbitals. 
In that case, the \program{RASSCF} input has to include keyword \keyword{LumOrb} to read 
from any external source of orbitals other than those generated by the \program{SEWARD} program. 
By modifying input to the \program{SCF} program, it is possible to generate 
alternative trial orbitals for the \program{RASSCF} program. Since a new set of trial orbitals is used,
the input to the \program{RASSCF} program is also changed. Now, the number of
active orbitals, as well as the number of active electrons, are 6. 

The two lowest orbitals (\keyword{Inactive} 2) are excluded from the active space
and one other orbital is placed in the secondary space.
If the previous (8,8) full valence space was used,
the \program{CASPT2} program would not be able to include more electronic correlation energy,
considering that the calculation involves a minimal basis set.
The input for the \program{CASPT2} program includes a frozen C 1s orbital, the lowest orbital
in the CH$_4$ molecule.

The charge and multiplicity of our wave function can be changed by computing the
CH$_4^+$ cation with the same methods. The \program{RASSCF} program defines
the character of the problem by specifying the number of electrons, the spin multiplicity, and the spatial
symmetry. In the example below, there is one less electron giving rise to doublet multiplicity:

%%%To_extract{/doc/samples/problem_based_tutorials/CASSCF.energy.CH4plus.input}
\begin{inputlisting}
*CASSCF energy for CH4+ at a fixed nuclear geometry
*File: CASSCF.energy.CH4plus
*
&GATEWAY
 Title = CH4+ molecule 
 coord = CH4.xyz; basis = STO-3G; Group = C1
&SEWARD
&RASSCF 
   Spin = 2; Nactel = 7 0 0; Inactive = 1; Ras2 = 8
\end{inputlisting}
%%%To_extract

No further modification is needed to the \program{CASPT2} input:

%%%To_extract{/doc/samples/problem_based_tutorials/CASPT2.energy.CH4plus.input}
\begin{inputlisting}
*CASPT2 energy for CH4+ at a fixed nuclear geometry
*File: CASPT2.energy.CH4plus
*
&GATEWAY
 coord = CH4.xyz; basis = STO-3G; group = C1
&SEWARD
&RASSCF
   Title = CH4+ molecule
   Spin = 2; Nactel = 1 0 0; Inactive = 4; Ras2 = 1
&CASPT2
\end{inputlisting}
%%%To_extract

A somewhat more sophisticated calculation can be performed at the
Restricted Active Space (RAS) SCF level. In such a situation, the level of excitation
in the CI expansion can be controlled by restricting the number of holes
and particles present in certain orbitals. 

%%%To_extract{/doc/samples/problem_based_tutorials/RASSCF.energy.CH4.input}
\begin{inputlisting}
*RASSCF energy for CH4 at a fixed nuclear geometry
*File: RASSCF.energy.CH4
*
&GATEWAY
 coord = CH4.xyz; basis = STO-3G; group = C1
&SEWARD
&RASSCF
   Title = CH4 molecule
   Spin = 1; Nactel = 8 1 1
   Inactive = 1; Ras1 = 1; Ras2 = 6; Ras3 = 1
\end{inputlisting}
%%%To_extract

In particular, the previous calculation includes one orbital within the Ras1
space and one orbital within the Ras3 space. One hole (single excitation) at
maximum is allowed from Ras1 to Ras2 or Ras3, while a maximum of one particle
is allowed in Ras3, derived from either Ras1 or Ras2. Within Ras2, all types
of orbital occupations are allowed. The RASSCF wave functions can be used 
as reference for multiconfigurational perturbation theory (RASPT2), but
this approach has not been as extensively tested as CASPT2, and, so experience is
still somewhat limited.

\molcas\ also has the possibility of computing electronic energies at 
different CI levels by using the \program{MRCI} program. The input provided below involves
a Singles and Doubles Configuration Interaction  (SDCI) calculation on the CH$_4$ molecule.
To set up the calculations, program \program{MOTRA} which transforms
the integrals to molecular basis, and program \program{GUGA} which computes the
coupling coefficients, must be run before the \program{MRCI} program.
In \program{MOTRA} the reference orbitals are specifiedi, and those employed 
here are from an HF \program{SCF} calculation including frozen orbitals. In \program{GUGA}
the reference for the CI calculation is described by the number of correlated electrons,
the spatial and spin symmetry, the inactive orbitals always occupation 2 in
the reference space, and the type of CI expansion. 

%%%To_extract{/doc/samples/problem_based_tutorials/SDCI.energy.CH4.input}
\begin{inputlisting}
*SDCI energy for CH4 at a fixed nuclear geometry
*File: SDCI.energy.CH4
*
&GATEWAY
 coord = CH4.xyz; basis = STO-3G; group = c1
&SEWARD
&SCF
&MOTRA
 Lumorb
 Frozen= 1
&GUGA 
 Electrons = 8
 Spin = 1
 Inactive= 4
 Active= 0
 Ciall= 1
&MRCI 
 SDCI
\end{inputlisting}
%%%To_extract

To use reference orbitals from a previous CASSCF calculation, the
\program{RASSCF} program will have to be run before the \program{MOTRA}
module. Also, if the spatial or spin symmetry are changed for the CI
calculation, the modifications will be introduced in the input to \program{GUGA} program.
Many alternatives are possible for performing an MRCI calculation as shown in the next example below,
in which the reference space to perform the CI is multiconfigurational:

%%%To_extract{/doc/samples/problem_based_tutorials/MRCI.energy.CH4.input}
\begin{inputlisting}
*MRCI energy for CH4 at a fixed nuclear geometry
*File: MRCI.energy.CH4
*
&GATEWAY
 Title = CH4 molecule
 coord = CH4.xyz; basis = STO-3G; group = c1
&SEWARD
&SCF
&RASSCF
 LumOrb
 Spin= 1; Nactel= 6 0 0; Inactive= 2; Ras2= 6
&MOTRA 
 Lumorb
 Frozen= 1
&GUGA
 Electrons= 8
 Spin= 1
 Inactive= 2
 Active= 3
 Ciall= 1
&MRCI
 SDCI
\end{inputlisting}
%%%To_extract

The \program{MRCI} program also allows the calculation of electronic energies using the
ACPF method. Another \molcas\ program, \program{CPF}, offers the possibility to 
use the CPF, MCPF, and ACPF methods with a single reference function. The 
required input is quite similar to that for the \program{MRCI} program:

%%%To_extract{/doc/samples/problem_based_tutorials/CPF.energy.CH4.input}
\begin{inputlisting}
*CPF energy for CH4 at a fixed nuclear geometry
*File: CPF.energy.CH4
*
&GATEWAY
 Title= CH4 molecule
 coord = CH4.xyz; basis = STO-3G; group = c1
&SEWARD
&SCF 
&MOTRA 
 Lumorb
 Frozen= 1
&GUGA
 Electrons= 8
 Spin = 1
 Inactive = 4
 Active = 0
 Ciall= 1
&CPF
 CPF
End Of Input
\end{inputlisting}
%%%To_extract

Finally, \molcas\ can also perform closed- and open-shell coupled cluster
calculations at the CCSD and CCSD(T) levels. These calculations are controlled by
the \program{CCSDT} program, whose main requirement is that the reference 
function has to be generated with the \program{RASSCF} program. The following input is 
required to obtain a CCSD(T) energy for the CH$_4$ molecule:

%%%To_extract{/doc/samples/problem_based_tutorials/CCSDT.energy.CH4.input}
\begin{inputlisting}
*CCSDT energy for CH4 at a fixed nuclear geometry
*File: CCSDT.energy.CH4
*
&GATEWAY
 Title= CH4 molecule
 coord = CH4.xyz; basis = STO-3G; group = c1
&SEWARD
&RASSCF
 Spin= 1; Nactel= 0 0 0; Inactive= 5; Ras2= 0
 OutOrbitals
 Canonical
&MOTRA
 JobIph
 Frozen= 1
&CCSDT
 CCT
\end{inputlisting}
%%%To_extract

Since this is a closed-shell calculation, the \program{RASSCF} input 
computes a simple RHF wave function with zero active electrons and orbitals using 
keywords \keyword{OutOrbitals} and \keyword{Canonical}. The \program{MOTRA} must
include the keyword \keyword{JobIph} to extract the wave function information
from file \file{JOBIPH} which is automatically generated by \program{RASSCF}. Finally,
the keywork \keyword{CCT} in program \program{CCSDT} leads to the calculation of the
CCSD(T) energy using the default algorithms.

The \program{CCSDT} program in \molcas\ is especially suited to compute open-shell
cases. The input required to obtain the electronic energy of the CH$_4^+$ cation
with the CCSD(T) method is:


%%%To_extract{/doc/samples/problem_based_tutorials/CCSDT.energy.CH4plus.input}
\begin{inputlisting}
*CCSDT energy for CH4+ at a fixed nuclear geometry
*File: CCSDT.energy.CH4plus
*
&GATEWAY
 Title= CH4+ molecule
 coord = CH4.xyz; basis = STO-3G; group = c1
&SEWARD
&RASSCF
 Spin= 2; Nactel= 1 0 0; Inactive= 4; Ras2= 1
 OutOrbitals
 Canonical
&MOTRA
 JobIph
 Frozen= 1
&CCSDT
 CCT
\end{inputlisting}
%%%To_extract

where the \program{RASSCF} program generated the proper Restricted Open-Shell 
Hartree-Fock (ROHF) reference. Different levels of spin adaptation are also available.

If solvent effects are desired, \molcas\ includes two
models: Kirkwood and PCM. Adding a solvent effect to a ground state at HF, DFT, or CASSCF levels,
simply requires the inclusion of the keyword \keyword{RF-input} within the input for the \program{SEWARD} 
which calculates a self-consistend reaction field.

%%%To_extract{/doc/samples/problem_based_tutorials/DFT.energy_solvent.CH4.input}
\begin{inputlisting}
*DFT energy for CH4 in water at a fixed nuclear geometry
*File: DFT.energy_solvent.CH4
*
&GATEWAY
 Title= CH4 molecule
 coord = CH4.xyz; basis = STO-3G; group = c1
 RF-input
   PCM-model; solvent= water
 End of RF-input
&SEWARD
&SCF
KSDFT= B3LYP
\end{inputlisting}
%%%To_extract

Other programs such as \program{CASPT2}, \program{RASSI}, and \program{MOTRA} require that
the reaction field is included as a perturbation with keyword \keyword{RFPErturbation}.
In the next example the correction is added at both the CASSCF and CASPT2 levels.

%%%To_extract{/doc/samples/problem_based_tutorials/CASPT2.energy_solvent.CH4.input}
\begin{inputlisting}
*CASPT2 energy for CH4 in acetone at a fixed nuclear geometry
*File: CASPT2.energy_solvent.CH4
*
&GATEWAY
 Title= CH4 molecule
 coord = CH4.xyz; basis = STO-3G; group = c1
  RF-input
   PCM-model; solvent= acetone; AAre= 0.2
  End of RF-input
&SEWARD
&RASSCF
  Spin= 1; Nactel= 6 0 0; Inactive= 2; Ras2= 6
&CASPT2
 Frozen= 1
 Multistate= 1 1
 RFPert
\end{inputlisting}
%%%To_extract

%)
Notice that the tesserae of the average area in the PCM model (keyword
has been changed to the value required for acetone by the keyword \keyword{Aare},
while the default is 0.4 \AA$^2$ for water 
\ifmanual
(see section~\ref{UG:sec:rfield}).
More detailed examples can be found in section~\ref{TUT:sec:cavity}. 
\fi

\section{Optimizing geometries}
%: minima, transition states, crossings, and minimum energy paths

It is now useful to explore potential energy surfaces (PES) and optimize the molecular geometry for
specific points along the PES. Different cases are discussed including a way to obtain the optimal geometry
in a minimum energy search, to obtain a transition-state structure connecting different regions of 
the PES, to find the crossing between two PES where the energy becomes degenerate, or to map
the minimum steepest-descent energy path (MEP) from an initial point to the final 
a minimum energy geometry as the PES progresses in a downward manner. 

All these types of searches can be performed either by fully optimizing all 
degrees of freedom of the system or by introducing certain restrictions. \molcas\ \molcasversion\ can perform
geometry optimizations at the SCF (RHF and UHF), DFT (RHF and UHF based), CASSCF (CASSCF and RASSCF) levels of theory, 
where efficient analytical gradients are available and at the CASPT2 and other correlated levels where numerical
gradients are used.

Geometry optimizations require many cycles, in which the electronic energy is estimated at a specific
level of calculation followed by calculation of the gradient of the energy with respect to the geometric
degrees of freedom (DOF). With this information at hand, the program must decide if the molecule is 
already at the final required geometry (i.e. gradient $\sim$ 0 for all
DOF) indicating a minimum in the PES or if the geometry must be modified 
and continue the cycle. The input file should,
therefore, be built in a way that allows a loop over the different programs. 

The general input commands \command{Do while} and \command{Enddo} control the loop 
and program input is inserted within these commands. Instructions for the number of maximum iterations allowed and the type of output required can also be added.
\ifmanual
(see section~\ref{UG:sec:sysvar})
\fi
%The commands \command{Set output file}, which prints output for each iterations and
%in the \$WorkDir directory with the file name Structure.\$iteration.output, and 
%\command{Set maxiter 100}, which sets maximum iterations to one hundred.

All examples previously discussed, use \keyword{COORD} keyword, but it also possible
to use \textit{native format}, where symmetry unique atoms are specified (\keyword{SYMMETRY}) 
and provide generators to construct all atoms in the molecule.

The selected example describes geometry optimization of the water molecule at the SCF RHF level
of calculation:

%%%To_extract{/doc/samples/problem_based_tutorials/Water_distorted.xyz}
\begin{inputlisting}
3
 coordinates for water molecule NOT in equilibrium 
O 0.000000  0.000000  0.000000 
H 0.758602  0.000000  0.504284 
H 0.758602  0.000000 -0.504284 
\end{inputlisting}
%%%To_extract


%%%To_extract{/doc/samples/problem_based_tutorials/SCF.minimum_optimization.H2O.input}
\begin{inputlisting}
*SCF minimum energy optimization for H2O
*File: SCF.minimum_optimization.H2O
*
&GATEWAY 
 Title= H2O minimum optimization
 coord=Water_distorted.xyz
 basis=ANO-S-MB
 group=C1

>>> Do while
 &SEWARD ;&SCF; &SLAPAF
>>> EndDo
\end{inputlisting}
%%%To_extract

The sequence of programs employed includes \program{GATEWAY} which is external to the loop, followed by
\program{SEWARD}, \program{SCF}, and \program{SLAPAF}. \program{SEWARD} 
computes the integrals,  \program{SCF} program computes the RHF energy and wave 
function, and  \program{SLAPAF} will control the calculation of gradients and
estimate if the calculation has already finished or needs to proceed to a new
nuclear geometry for the next iteration. Automatically, a file named 
\file{\$Project.geo.molden} will be generated in \$WorkDir containing all the 
geometric steps contained in the optimization process. \program{MOLDEN} or \program{LUSCUS} can 
then read this file to display the individual molecular geometries which form the optimization cycle.
 
Using another reference wave function can be simply performed by changing the sequence of 
programs. For instance, we can perform an UHF calculation of the H$_2$O$^+$ 
cation:
  
%%%To_extract{/doc/samples/problem_based_tutorials/UHF.minimum_optimization.H2Oplus.input}
\begin{inputlisting}
*UHF minimum energy optimization for H2O+
*File: UHF.minimum_optimization.H2Oplus
*
&GATEWAY 
 Title= H2O minimum optimization
 coord=Water_distorted.xyz
 basis=ANO-S-MB
 group=C1
>> Do while

 &SEWARD 
 &SCF; Title="H2O minimum optimization"; UHF; Charge=1
 &SLAPAF 

>> EndDo
\end{inputlisting}
%%%To_extract

The same procedure can be followed if we pretend to perform a DFT geometry optimization:

%%%To_extract{/doc/samples/problem_based_tutorials/DFT.minimum_optimization.H2O.input}
\begin{inputlisting}
*DFT minimum energy optimization for H2O
*File: DFT.minimum_optimization.H2O
*
&GATEWAY 
 Title= H2O minimum optimization
 coord=Water_distorted.xyz
 basis=ANO-S-MB
 group=C1

>>> Export MOLCAS_MAXITER=100
>>> Do while

 &SEWARD
 &SCF ; Title="H2O minimum optimization"; KSDFT=B3LYP
 &SLAPAF &END

>>> EndDo
\end{inputlisting}
%%%To_extract

Once an energy minimum is found based on the calculation of gradients, it is necessary to
ensure that the geometry really is a minimum energy point. This can be only 
accomplished by computing second derivatives of the energy (i.e. the Hessian). 
\molcas\ can compute analytical Hessians for SCF and single state
CASSCF wave functions. For other methods,  numerical procedures can be used 
to compute the Hessian. Once the Hessian is computed, vibrational
frequencies are calculated, and Statistical Mechanics is used to obtain thermodynamic 
properties. At a true energy minimum, there will be 3N-6 real frequencies 
Program \program{MCKINLEY} computes second derivatives 
of a predefined (SCF or CASSCF) wave function, while \program{MCLR} performs 
the vibrational and statistical analyses. \molcas\ simply requires input for 
the \program{MCKINLEY} program to perform the entire calculation by using keywords
\keyword{Perturbation} and \keyword{Hessian}, while program \program{MCLR} will be 
called automatically but requires no input. 
The full set of calculationsi is included below first a geometry optimization followed by the 
calculation of a Hessian.

%%%To_extract{/doc/samples/problem_based_tutorials/SCF.minimization_plus_Hessian.H2O.input}
\begin{inputlisting}
*SCF minimum energy optimization plus hessian of the water molecule
*File: SCF.minimization_plus_hessian.H2O
*
&GATEWAY 
 Title= H2O minimum optimization
 coord=Water_distorted.xyz
 basis=ANO-S-MB
 group=C1

>>> Export MOLCAS_MAXITER=100
>>> Do while

 &SEWARD
 &SCF; Title="H2O minimum optimization"
 &SLAPAF

>>> EndDo

&MCKINLEY
\end{inputlisting}
%%%To_extract

Note that \program{MCKINLEY} input above is placed after \command{EndDo}, and, therefore,
is external to the looping scheme. Once the geometry optimization at the desired level of theory has finished, the 
Hessian will be computed at the final geometry.
In general, any calculation performed using a \$WorkDir directory where a 
previous geometry optimization has taken place will use the last geomtry calculated 
from that optimization as the input geometry even if \program{SEWARD} input is 
present. To avoid that, the only solution is to remove the communication file 
\file{RUNFILE} where the geometry is stored. Note also, that the frequencies are 
computed in a cartesian basis, and that three translational and three rotational 
frequencies which should be very close to zero are included in the output file.
This is not the case when numerical gradients and Hessians are used. 
In particular, for water at its minimum energy structure three (3N-6) 
real vibrational frequencies. By default, in \$WorkDir a file \file{\$Project.freq.molden}
is generated containing the vibrational frequencies and modes, which can be visualized by \program{MOLDEN}.

A new level of theory, CASSCF, is introduced here which is especially suited for 
geometry optimizations of excited states discussed in the next chapter.
A geometry optimization is performed to illustrate a broader range of possibilities including 
the imposition of a geometric restrain that the HOH angle in water should be constrained to 120$^o$
during the optimization.
This means that only the O-H bond distances be optimized in this partial minimization. 
The restriction is indicated
in in \program{GATEWAY}
by invoking the keyword \keyword{Constraints} and ending with the keyword \keyword{End of Constraints}. 
The names of variables corresponding to geometrical variables in either internal or Cartesian coordinates
that are to be constrained are placed between these two keywords. 
\ifmanual
(see nomenclature in 
section~\ref{UG:sec:definition_of_internal_coordinates})
\fi
In the case of H$_2$O, the H1-O-H2 angle is fixed at 120$^o$, so a variable,
$a$, is first defined with the keywork \keyword(Angle), which relates it to the H1-O1-H2 angle, followed by the second keyword, \keyword{Value}, 
where the variable $a$ is specified as 120$^o$. 
It is not required that the initial geometry is 120$^o$, only that the final result for the calculation
will become 120$^o$.

Note that the \program{RASSCF} program requires initial trial orbitals, and those  
which are automatically generated by \program{SEWARD} are used.  The resulting CASSCF 
wave function includes all valence orbitals and electrons.

%%%To_extract{/doc/samples/problem_based_tutorials/CASSCF.minimum_optimization_restricted.H2O.input}
\begin{inputlisting}
*CASSCF minimum energy optimization of the water molecule with geometrical restrictions
*File: CASSCF.minimum_optimization_restricted.H2O
&GATEWAY 
 Title= H2O minimum optimization
 coord=Water_distorted.xyz
 basis=ANO-S-MB
 group=C1
Constraint
   a = Angle H2 O1 H3
  Value
   a = 90. degree
End of Constraints

>>> Do while
   
 &SEWARD 
 &RASSCF; nActEl=8 0 0; Inactive=1; Ras2=6
 &SLAPAF 

>>> EndDo
\end{inputlisting}
%%%To_extract

Other more flexible ways to impose geometric restrictions involve the specification of which internal
coordinates should remain fixed and which should change. In the next example, 
the bond lengths are forced to remain fixed at their initial distance (here 0.91 \AA), while the
bond angle, having an initial of 81$^\circ$, is optimized.

%%%To_extract{/doc/samples/problem_based_tutorials/DFT.minimum_optimization_restricted.H2O.input}
\begin{inputlisting}
*DFT minimum energy optimization of the angle in the water molecule at fixed bond lengths
*File: DFT.minimum_optimization_restricted.H2O
*
&GATEWAY 
 Title= H2O minimum optimization
 coord=Water_distorted.xyz
 basis=ANO-S-MB
 group=C1

>>> EXPORT MOLCAS_MAXITER=100
>>> Do while
                                                                                                                                                                            
 &SEWARD; &SCF; Title="H2O restricted minimum"; KSDFT=B3LYP
 &SLAPAF 
  Internal Coordinates
     b1 = Bond O1 H2
     b2 = Bond O1 H3
     a1 = Angle H2 O1 H3
  Vary
     a1
  Fix
     b1
     b2
  End of Internal

>>> EndDo
\end{inputlisting}
%%%To_extract

In the final output, the two O-H bond lengths remain at the initia values, while the H1-O1=H2 angle is optimized
to a final angle of 112$^o$.

The next step entails the computation of a transition state, a structure connecting different regions of
the potential energy hypersurface, and is a maximum for only one degree of
freedom. The most common saddle points have order one, that is, they are maxima for one of
one displacement and minima for the others. The simplest way to search for a 
transition state in \molcas\ is to add the keyword \keyword{TS} to the 
\program{SLAPAF} input. Keyword \keyword{PRFC} is suggested in order to verify 
the nature of the transition structure. Searching for transition states is, 
however, not an easy task. An illustration of the input required for transition state optimization for water at the DFT level 
is given below:

%%%To_extract{/doc/samples/problem_based_tutorials/Water_TS.xyz}
\begin{inputlisting}
3
water in Transition state in bohr
O1             0.750000        0.000000        0.000000 
H2             1.350000        0.000000        1.550000 
H3             1.350000        0.000000       -1.550000 
\end{inputlisting}
%%%To_extract


%%%To_extract{/doc/samples/problem_based_tutorials/DFT.transition_state.H2O.input}
\begin{inputlisting}
*DFT transition state optimization of the water molecule 
*File: DFT.transition_state.H2O
*
&Gateway
 Coord=Water_TS.xyz
 Basis=ANO-S-VDZ
 Group=C1
>>> Do while

 &SEWARD
 &SCF; Title="H2O TS optimization"; KSDFT=B3LYP
 &SLAPAF ; ITER=20 ; TS

>>> EndDo
\end{inputlisting}
%%%To_extract

The initial coordinates were chosen in units of Bohr, to illustrare that this is the
default case.  The optimal geometry for ground state of water is a structure with C$_{2v}$ symmetry. 
A transition state has been found with a linear HOH angle of 180$^o$. 
In many cases, there may be a clue along the energy pathway for a chemical reaction about the nature of the transition state structure, 
which typically represents an intermediate conformation between reactants and products. 
If this turns out to be the case, it is possible to help the optimization process 
proceed toward an informed guess, by invoking the keyword \keyword{FindTS} in \program{SLAPAF}.
\keyword{FindTS} must to be accompanied with a definition of constrained geometric definitions.
\program{SLAPAF} will guide the optimization of the transition state towards a region in
which the restriction is fulfilled. Once there, the restriction will be released
and a free search of the transition state will be performed. This technique is 
frequently quite effective and makes it possible to find difficult transition 
states or reduce the number of required iterations. Here, an example is provided, in 
which the initial geometry of water is clearly bent, and a trial restraint is imposed
such that the angle for the transition state should be near 180$^o$. The 
final transition state will, however, be obtained without any type of geometrical restriction.

%%%To_extract{/doc/samples/problem_based_tutorials/DFT.transition_state_restricted.H2O.input}
\begin{inputlisting}
*DFT transition state optimization of the water molecule with geometrical restrictions
*File: DFT.transition_state_restricted.H2O
*
&Gateway
 Coord=Water_TS.xyz
 Basis=ANO-S-VDZ
 Group=C1
 Constraints
   a = Angle H2 O1 H3
 Value
   a = 180.0 degree
 End of Constraints

>>> Do while

 &SEWARD
 &SCF; Title="H2O TS optimization"; KSDFT=B3LYP
 &SLAPAF ;FindTS

>>> EndDo
\end{inputlisting}
%%%To_extract

The \program{CASPT2} geometry optimizations are somewhat different because \program{ALASKA}
is not suited to compute \program{CASPT2} analytical gradients. Therefore the \program{ALASKA}
program is automatically substituted by program \program{NUMERICAL\_GRADIENT}, which will take care
of performing numerical gradients. From the user point of view the only requirement is to place
the \program{CASPT2} input after the \program{RASSCF} input.
The CASSCF wave function has of course to be generated in each step before 
performing CASPT2. To compute a numerical gradient can be quite time consuming, 
although it is a task that can be nicely parallelized. In a double-sided 
gradient algorithm like here a total of 6N-12+1 CASPT2 calculations are performed 
each pass of the optimization, where N is the number of atoms.

%%%To_extract{/doc/samples/problem_based_tutorials/CASPT2.minimum_optimization.H2O.input}
\begin{inputlisting}
*CASPT2 minimum energy optimization for water
*File: CASPT2.minimum_optimization.H2O
*
&GATEWAY 
 coord=Water_distorted.xyz
 basis=ANO-S-MB
 group=C1

>>> Do while

 &SEWARD
 &RASSCF; Title="H2O restricted minimum"; nActEl=8 0 0; Inactive=1; Ras2=6
 &CASPT2; Frozen=1 
 &SLAPAF 

>>> EndDo
\end{inputlisting}
%%%To_extract

The use of spatial symmetry makes the calculations more efficient, although
they may again complicate the preparation of input files. We can repeat the previous \program{CASPT2}
optimization by restricting the molecule to work in the C$_{2v}$ point group, which, by the way,
is the proper symmetry for water in the ground state. The \program{GATEWAY} program (as no symmetry
has been specified) will identify and work with the highest available point group,
C$_{2v}$. Here the molecule is placed with YZ as the molecular plane. By adding
keyword \keyword{Symmetry} containing as elements of symmetry the YZ (symbol X) and YX (symbol Z),
the point group is totally defined and the molecule properly generated. From that point the
calculations will be restricted to use symmetry restrictions. For instance, the molecular
orbitals will be classified in the four elements of symmetry of the group, a$_1$, b$_1$, b$_2$,
and a$_2$, and most of the programs will require to define the selection of the orbitals in
the proper order. The order of the symmetry labels is determined by \program{SEWARD} and must
be checked before proceeding, because from that point the elements of symmetry will be known
by their order in \program{SEWARD}: a$_1$, b$_1$, b$_2$, and a$_2$, for instance, will be
symmetries 1, 2, 3, and 4, respectively. \program{SCF} does not require to specify the
class of orbitals and it can be used as a learning tool.


%%%To_extract{/doc/samples/problem_based_tutorials/CASPT2.minimum_optimization_C2v.H2O.input}
\begin{inputlisting}
*CASPT2 minimum energy optimization for water in C2v
*File: CASPT2.minimum_optimization_C2v.H2O
*
 &GATEWAY
Title= H2O caspt2 minimum optimization
Symmetry= X Z
Basis set
O.ANO-S...2s1p.
O        0.000000  0.000000  0.000000 Angstrom
End of basis
Basis set
H.ANO-S...1s.
H1       0.000000  0.758602  0.504284 Angstrom
End of basis

>>> EXPORT MOLCAS_MAXITER=100
>>> Do while

 &SEWARD
 &RASSCF; nActEl=8 0 0; Inactive=1 0 0 0; Ras2=3 1 2 0
 &CASPT2; Frozen=1 0 0 0
 &SLAPAF &END

>>> EndDo
\end{inputlisting}
%%%To_extract

Thanks to symmetry restrictions the number of iterations within \program{NUMERICAL\_GRADIENT}
has been reduced to five instead of seven, because many of the deformations 
are redundant within the C$_{2v}$ symmetry. Also, symmetry considerations are 
important when defining geometrical restrictions 
\ifmanual
(see sections~\ref{UG:sec:definition_of_internal_coordinates}
and \ref{TUT:sec:optim}).
\else
(see online manual).
\fi

\section{Computing excited states}

The calculation of electronic excited states is typically a multiconfigurational problem, and
therefore it should preferably be treated with multiconfigurational methods such as CASSCF and
CASPT2. We can start this section by computing the low-lying electronic states of the
acrolein molecule at the CASSCF level and using a minimal 
basis set. The  standard file with cartesian coordinates is:

%%%To_extract{/doc/samples/problem_based_tutorials/acrolein.xyz}
\begin{inputlisting}
 8
Angstrom
 O      -1.808864   -0.137998    0.000000
 C       1.769114    0.136549    0.000000
 C       0.588145   -0.434423    0.000000
 C      -0.695203    0.361447    0.000000
 H      -0.548852    1.455362    0.000000
 H       0.477859   -1.512556    0.000000
 H       2.688665   -0.434186    0.000000
 H       1.880903    1.213924    0.000000
\end{inputlisting}
%%%To_extract

We shall carry out State-Averaged (SA) CASSCF calculations, in which one single 
set of molecular orbitals is used to compute all the states of a given spatial 
and spin symmetry. The obtained density matrix is the average for all states 
included, although each state will have its own set of optimized CI 
coefficients. Different weights can be considered for each of the states, 
but this should not be used except in very special cases by experts. It is 
better to let the CASPT2 method to handle that. The use of a SA-CASSCF 
procedure has an great advantage. For example, all states in a SA-CASSCF 
calculation are orthogonal to each other, which is not necessarily true for
state specific calculations. Here, we shall include five states of singlet 
character the calculation. As no symmetry is invoked all the states belong by 
default to the first symmetry, including the ground state.

%%%To_extract{/doc/samples/problem_based_tutorials/CASSCF.excited.acrolein.input}
\begin{inputlisting}
*CASSCF SA calculation on five singlet excited states in acrolein
*File: CASSCF.excited.acrolein
*
&GATEWAY
  Title= Acrolein molecule
  coord = acrolein.xyz; basis = STO-3G; group = c1
&SEWARD; &SCF
&RASSCF
  LumOrb
  Spin= 1; Nactel= 6 0 0; Inactive= 12; Ras2= 5
  CiRoot= 5 5 1
&GRID_IT
  All
\end{inputlisting}
%%%To_extract


We have used as active all the $\pi$ and $\pi^*$ orbitals, two bonding and
two antibonding $\pi$ orbitals with four electrons and in addition the oxygen 
lone pair ($n$). Keyword \keyword{CiRoot} informs the program that we want to 
compute a total of five states, the ground state and the lowest four excited 
states at the CASSCF level and that all of them should have the same weight in 
the average procedure. Once analyzed we find that the calculation has provided,
in this order, the ground state, two $n\to\pi^*$ states, and two $\pi\to\pi^*$ states.
It is convenient to add the \program{GRID\_IT} input in order to be able to use
the \program{LUSCUS} interface for the analysis of the orbitals and the occupations
in the different electronic states. Such an analysis should always be made in 
order to understand the nature of the different excited states.
In order to get a more detailed analysis of the nature of the obtained states it is
also possible to obtain in a graphical way the charge density differences between
to states, typically the difference between the ground and an excited state. The
following example creates five different density files:

%%%To_extract{/doc/samples/problem_based_tutorials/CASSCF.excited_grid.acrolein.input}
\begin{inputlisting}
*CASSCF SA calculation on five singlet excited states in acrolein
*File: CASSCF.excited_grid.acrolein
*
&GATEWAY
  Title= Acrolein molecule
  coord= acrolein.xyz; basis= STO-3G; group= c1
&SEWARD; &SCF
&RASSCF 
 LumOrb
 Spin= 1; Nactel= 6 0 0; Inactive= 12; Ras2= 5
 CiRoot= 5 5 1
 OutOrbital
 Natural= 5
&GRID_IT
 FILEORB = $Project.RasOrb.1
 NAME = 1; All
&GRID_IT
 FILEORB = $Project.RasOrb.2
 NAME = 2; All
&GRID_IT
 FILEORB = $Project.RasOrb.3
 NAME = 3; All
&GRID_IT
 FILEORB = $Project.RasOrb.4
 NAME = 4; All
&GRID_IT
 FILEORB = $Project.RasOrb.5
 NAME = 5; All
\end{inputlisting}
%%%To_extract

In \program{GRID\_IT} input we have included all orbitals. It is, however,
possible and in general recommended to restrict the calculation to certain
sets of orbitals. How to do this is described in the input manual for
\program{GRID\_IT}. 

Simple math operations can be performed with grids of the same size, 
for example, \program{LUSCUS} can be used to display the difference 
between two densities. 

CASSCF wave functions are typically good enough, but this is not the case for
electronic energies, and the dynamic correlation effects have to be included,
in particular here with the CASPT2 method. The proper input is prepared, again
including \program{SEWARD} and \program{RASSCF} (unnecessary if they were
computed previously), adding a \program{CASPT2} input with the keyword
\keyword{MultiState} set to 5 1 2 3 4 5. The \program{CASPT2} will perform four
consecutive single-state (SS) CASPT2 calculations using the SA-CASSCF roots computed
by the \program{RASSCF} module. At the end, a multi-state CASPT2 calculation
will be added in which the five SS-CASPT2 roots will be allowed to interact.
The final MS-CASPT2 solutions, unlike the previous SS-CASPT2 states, will be
orthogonal. The \keyword{FROZen} keyword is put here as a reminder. By
default the program leaves the core orbitals frozen.


%%%To_extract{/doc/samples/problem_based_tutorials/CASPT2.excited.acrolein.input}
\begin{inputlisting}
*CASPT2 calculation on five singlet excited states in acrolein
*File: CASPT2.excited.acrolein
*
&GATEWAY
 Title= Acrolein molecule
 coord = acrolein.xyz; basis = STO-3G; group= c1
&SEWARD; &SCF
&RASSCF
 Spin= 1; Nactel= 6 0 0; Inactive= 12; Ras2= 5
 CiRoot= 5 5 1
&GRID_IT
 All
&CASPT2
 Multistate= 5 1 2 3 4 5
 Frozen= 4
\end{inputlisting}
%%%To_extract


Apart from energies and state properties it is quite often necessary to compute
state interaction properties such as transition dipole moments, Einstein coefficients,
and many other. This can be achieved with the \program{RASSI} module, a powerful
program which can be used for many purposes 
\ifmanual
(see section~\ref{UG:sec:rassi})
\else
(see online manual)
\fi
. We can
start by simply computing the basic interaction properties

%%%To_extract{/doc/samples/problem_based_tutorials/CASSI.excited.acrolein.input}
\begin{inputlisting}
*RASSI calculation on five singlet excited states in acrolein
*File: RASSI.excited.acrolein
*
&GATEWAY
 Title= Acrolein molecule
 coord = acrolein.xyz; basis = STO-3G; group = c1
&SEWARD; &SCF
&RASSCF
 LumOrb
 Spin= 1; Nactel= 6 0 0; Inactive= 12; Ras2= 5
 CiRoot= 5 5 1
&CASPT2
 Frozen = 4
 MultiState= 5 1 2 3 4 5
                                                                                                                                                                            
>>COPY $Project.JobMix JOB001

&RASSI
 Nr of JobIph
 1 5
 1 2 3 4 5
 EJob
\end{inputlisting}
%%%To_extract

Oscillator strengths for the computed transitions and Einstein coefficients are
compiled at the end of the \program{RASSI} output file. To obtain these values,
however, energy differences have been used which are obtained from the previous
CASSCF calculation. Those energies are not accurate because they do not include
dynamic correlation energy and it is better to substitute them by properly
computed values, such those at the CASPT2 level. This is achieved with the
keyword \keyword{Ejob}. 
\ifmanual
More information is available
in section~\ref{TUT:sec:rassi_thio}.
\fi

Now a more complex case. We want to compute vertical singlet-triplet gaps from
the singlet ground state of acrolein to different, up to five, triplet excited
states. Also, interaction properties are requested. Considering that the spin
multiplicity differs from the ground to the excited states, the spin Hamiltonian
has to be added to our calculations and the \program{RASSI} program takes charge
of that. It is required first, to add in the \program{SEWARD} input the keyword 
\keyword{AMFI}, which introduces the proper integrals required, and to the 
\program{RASSI} input the keyword \keyword{SpinOrbit}. Additionally, as we want
to perform the calculation sequentially and \program{RASSI} will read from
two different wave function calculations, we need to perform specific links
to save the information. The link to the first \program{CASPT2} calculation
will saved in file \file{\$Project.JobMix.S} the data from the \program{CASPT2}
result of the ground state, while the second link before the second \program{CASPT2}
run will do the same for the triplet states. Later, we link these files as
\file{JOB001} and \file{JOB002} to become input files for \program{RASSI}.
In the \program{RASSI} input \keyword{NrofJobIph} will be set to two, meaning
two \file{JobIph} or \file{JobMix} files, the first containing one root (the ground
state) and the second five roots (the triplet states). Finally, we have added 
\keyword{EJob}, which will read the CASPT2 (or MS-CASPT2) energies from the
\file{JobMix} files to be incorporated to the \program{RASSI} results.
The magnitude of properties computed with spin-orbit coupling (SOC) depends
strongly on the energy gap, and this has to be computed at the highest possible
level, such as CASPT2.

%%%To_extract{/doc/samples/problem_based_tutorials/CASPT2.S-T_gap.acrolein.input}
\begin{inputlisting}
*CASPT2/RASSI calculation on singlet-triplet gaps in acrolein
*File: CASPT2.S-T_gap.acrolein
*
&GATEWAY
 Title= Acrolein molecule
 coord = acrolein.xyz; basis = STO-3G; group= c1
&SEWARD 
 AMFI
&SCF
&RASSCF
 Spin= 1; Nactel= 6 0 0; Inactive= 12; Ras2= 5
 CiRoot= 1 1 1
&CASPT2
 Frozen= 4
 MultiState= 1 1
>>COPY $Project.JobMix JOB001
&RASSCF
 LumOrb
 Spin= 3; Nactel= 6 0 0; Inactive= 12; Ras2= 5
 CiRoot= 5 5 1
&CASPT2
 Frozen= 4
 MultiState= 5 1 2 3 4 5
>>COPY $Project.JobMix JOB002
&RASSI 
 Nr of JobIph= 2 1 5; 1; 1 2 3 4 5
 Spin
 EJob
\end{inputlisting}
%%%To_extract

As here with keyword \keyword{AMFI}, 
when using command \command{Coord} to build a \program{SEWARD} input
and we want to introduce other keywords, it is enough if we place them
after the line corresponding to \command{Coord}.
Observe that the nature of the triplet states obtained is in sequence one
$n\pi^*$, two $\pi\pi^*$, and two $n\pi^*$. The \program{RASSI} output is 
somewhat complex to analyze, but it makes tables summarizing oscillator 
strengths and Einstein coefficients, if those are the magnitudes of interest. 
Notice that a table is first done with the spin-free states, while the final 
table include the spin-orbit coupled eigenstates (in the CASPT2 energy order 
here), in which each former triplet state has three components.

In many cases working with symmetry will help us to perform calculations
in quantum chemistry. As it is a more complex and delicate problem we direct
the reader to the examples section in this manual. However, we include here
two inputs that can help the beginners. They are based on trans-1,3-butadiene,
a molecule with a C$_{2h}$ ground state. If we run the next input, the
\program{SEWARD} and \program{SCF} outputs will help us to understand how
orbitals are classified by symmetry, whereas reading the \program{RASSCF} output
the structure of the active space and states will be clarified.

%%%To_extract{/doc/samples/problem_based_tutorials/CASSCF.excited.tButadiene.1Ag.input}
\begin{inputlisting}
*CASSCF SA calculation on 1Ag excited states in tButadiene
*File: CASSCF.excited.tButadiene.1Ag
*
&SEWARD 
  Title= t-Butadiene molecule
  Symmetry= Z XYZ
Basis set
C.STO-3G...
C1   -3.2886930 -1.1650250 0.0000000  Bohr
C2   -0.7508076 -1.1650250 0.0000000  Bohr
End of basis
Basis set
H.STO-3G...
H1   -4.3067080  0.6343050 0.0000000  Bohr
H2   -4.3067080 -2.9643550 0.0000000  Bohr
H3    0.2672040 -2.9643550 0.0000000  Bohr
End of basis
                                                                                                                                                                            
&SCF 
                                                                                                                                                                            
&RASSCF 
 LumOrb
 Title= tButadiene molecule (1Ag states). Symmetry order (ag bg bu au)
 Spin= 1; Symmetry= 1; Nactel= 4 0 0; Inactive= 7 0 6 0; Ras2= 0 2 0 2
 CiRoot= 4 4 1

&GRID_IT 
 All
\end{inputlisting}
%%%To_extract

Using the next input will give information about states of a different symmetry.
Just run it as a simple exercise.

%%%To_extract{/doc/samples/problem_based_tutorials/CASSCF.excited.tButadiene.1Bu.input}
\begin{inputlisting}
*CASSCF SA calculation on 1Bu excited states in tButadiene
*File: CASSCF.excited.tButadiene.1Bu
*
&SEWARD 
 Title= t-Butadiene molecule
 Symmetry= Z XYZ
Basis set
C.STO-3G...
C1   -3.2886930 -1.1650250 0.0000000  Bohr
C2   -0.7508076 -1.1650250 0.0000000  Bohr
End of basis
Basis set
H.STO-3G...
H1   -4.3067080  0.6343050 0.0000000  Bohr
H2   -4.3067080 -2.9643550 0.0000000  Bohr
H3    0.2672040 -2.9643550 0.0000000  Bohr
End of basis
                                                                                                                                                                            
&SCF 
                                                                                                                                                                            
&RASSCF 
 FileOrb= $Project.ScfOrb
 Title= tButadiene molecule (1Bu states). Symmetry order (ag bg bu au)
 Spin= 1; Symmetry= 1; Nactel= 4 0 0; Inactive= 7 0 6 0
 Ras2= 0 2 0 2
 CiRoot= 4 4 1
>COPY $Project.RasOrb $Project.1Ag.RasOrb
>COPY $Project.JobIph JOB001

&GRID_IT 
 Name= $Project.1Ag.lus
 All

&RASSCF 
 FileOrb= $Project.ScfOrb
 Title= tButadiene molecule (1Bu states). Symmetry order (ag bg bu au)
 Spin= 1; Symmetry= 3; Nactel= 4 0 0; Inactive= 7 0 6 0; Ras2= 0 2 0 2
 CiRoot= 2 2 1
>COPY $Project.RasOrb $Project.1Bu.RasOrb
>COPY $Project.JobIph JOB002
                                                                                                                                                                            
&GRID_IT 
 Name= $Project.1Bu.lus
 All

&RASSI 
 NrofJobIph= 2 4 2; 1 2 3 4; 1 2
\end{inputlisting}
%%%To_extract

Structure optimizations can be also performed at the CASSCF, RASSCF or CASPT2
levels. Here we shall optimize the second singlet state in the first (here the
only) symmetry for acrolein at the SA-CASSCF level. It is strongly recommended
to use the State-Average option and avoid single state CASSCF calculations for
excited states. Those states are non-orthogonal with the ground state and
are typically heavily contaminated. The usual set of input commands will be
prepared, with few changes. In the \program{RASSCF} input two states will
be simultaneously computed with equal weight (\keyword{CiRoot} 2 2 1), but,
in order to get accurate gradients for a specific root (not an averaged one),
we have to add \keyword{Rlxroot} and set it to two, which is, among the
computed roots, that we want to optimize. The proper density matrix will be
stored. The \program{MCLR} program optimizes, using a perturbative approach,
the orbitals for the specific root (instead of using averaged orbitals), but
the program is called automatically and no input is needed.

%%%To_extract{/doc/samples/problem_based_tutorials/CASSCF.excited_state_optimization.acrolein.input}
\begin{inputlisting}
*CASSCF excited state optimization in acrolein
*File: CASSCF.excited_state_optimization.acrolein
*
 &GATEWAY
Title= acrolein minimum optimization in excited state 2
Basis set
O.STO-3G...2s1p.
O1       1.608542      -0.142162       3.240198 Angstrom
End of basis
Basis set
C.STO-3G...2s1p.
C1      -0.207776       0.181327      -0.039908 Angstrom
C2       0.089162       0.020199       1.386933 Angstrom
C3       1.314188       0.048017       1.889302 Angstrom
End of basis
Basis set
H.STO-3G...1s.
H1       2.208371       0.215888       1.291927 Angstrom
H2      -0.746966      -0.173522       2.046958 Angstrom
H3      -1.234947       0.213968      -0.371097 Angstrom
H4       0.557285       0.525450      -0.720314 Angstrom
End of basis
>>> Do while

 &SEWARD  
                                                                                                                                                                            
>>> If ( Iter = 1 ) <<<
                                                                                                                                                                            
 &SCF 
Title= acrolein minimum optimization
                                                                                                                                                                            
>>> EndIf <<<

 &RASSCF 
LumOrb
Title= acrolein
Spin= 1; nActEl= 4 0 0; Inactive= 13; Ras2= 4
CiRoot= 2 2 1
Rlxroot= 2
                                                                                                                                                                            
 &SLAPAF 
                                                                                                                                                                            
>>> EndDo
\end{inputlisting}
%%%To_extract

In case of performing a \program{CASPT2} optimization for an excited
state, still the SA-CASSCF approach can be used to generate the reference
wave function, but keyword \keyword{Rlxroot} and the use of the \program{MCLR} program
are not necessary, because \program{CASPT2} takes care of selecting
the proper root (the last one).

A very useful tool recently included in \molcas\ is the possibility to
compute minimum energy paths (MEP), representing steepest descendant minimum
energy reaction paths which are built through a series of geometry optimizations, 
each requiring the minimization of the potential energy on a hyperspherical
cross section of the PES centered on a given reference geometry and characterized 
by a predefined radius. One usually starts the calculation from a high energy reference
geometry, which may correspond to the Franck-Condon (FC) structure on an excited-state PES 
or to a transition structure (TS). Once the first lower energy optimized structure is
converged, this is taken as the new hypersphere center, and the procedure is iterated 
until the bottom of the energy surface is reached. Notice that in the TS case a pair of
steepest descent paths, connecting the TS to the reactant and product structures 
(following the forward and reverse orientation of the direction defined by the transition 
vector) provides the minimum energy path (MEP) for the reaction. As mass-weighted 
coordinates are used by default, the MEP coordinate corresponds to the so-called Intrinsic 
Reaction Coordinates (IRC). We shall compute here the MEP from the FC structure of acrolein
along the PES of the second root in energy at the CASSCF level. It is important to remember 
that the CASSCF order may not be accurate and the states may reverse orders at higher
levels such as CASPT2.

%%%To_extract{/doc/samples/problem_based_tutorials/CASSCF.mep_excited_state.acrolein.input}
\begin{inputlisting}
*CASSCF excited state mep points in acrolein
*File: CASSCF.mep_excited_state.acrolein
*
 &GATEWAY
Title = acrolein mep calculation root 2
Basis set
O.STO-3G...2s1p.
 O1    1.367073     0.000000     3.083333 Angstrom
End of basis
Basis set
C.STO-3G...2s1p.
 C1    0.000000     0.000000     0.000000 Angstrom
 C2    0.000000     0.000000     1.350000 Angstrom
 C3    1.367073     0.000000     1.833333 Angstrom
End of basis
Basis set
H.STO-3G...1s.
 H1    2.051552     0.000000     0.986333 Angstrom
 H2   -0.684479     0.000000     2.197000 Angstrom
 H3   -1.026719     0.000000    -0.363000 Angstrom
 H4    0.513360     0.889165    -0.363000 Angstrom
End of basis

>>> EXPORT MOLCAS_MAXITER=300
>>> Do while

 &SEWARD
>>> If ( Iter = 1 ) <<<
 &SCF 
>>> EndIf <<<

 &RASSCF 
   Title="acrolein mep calculation root 2"; Spin=1
   nActEl=4 0 0; Inactive=13; Ras2=4; CiRoot=2 2 1; Rlxroot=2
 &SLAPAF 
   MEP-search
   MEPStep=0.1

>>> EndDo
\end{inputlisting}
%%%To_extract

As observed, to prepare the input for the MEP is simple, just add the keyword \keyword{MEP-search}
and specify a step size with \keyword{MEPStep}, and the remaining structure equals that of a geometry optimization.
The calculations are time consuming, because each point of the
MEP (four plus the initial one obtained here) is computed through a specific optimization.
A file named \file{\$Project.mep.molden} (read by \program{MOLDEN} )
will be generated in \$WorkDir containing only those points belonging to the MEP.

We shall now show how to perform geometry optimizations under nongeometrical
restrictions, in particular, how to compute hypersurface crossings, which are key structures
in the photophysics of molecules. We shall get those points as minimum energy crossing points in
which the energy of the highest of the two states considered is minimized under the restriction
that the energy difference with the lowest state should equal certain value (typically zero).
Such point can be named a minimum energy crossing point (MECP). If a further restriction is
imposed, like the distance to a specific geometry, and several MECP as computed at varying distances,
it is possible to obtain a crossing seam of points where the energy between the two states is
degenerated. Those degeneracy points are funnels with the highest probability for the energy
to hop between the surfaces in internal conversion or intersystem crossing photophysical processes.
There are different possibilities. A crossing between states of the same spin
multiplicity and spatial symmetry is named a conical intersection. Elements like the nonadiabatic
coupling terms are required to obtain them strictly, and they are not computed presently
by \molcas. If the crossing occurs between states of the same
spin multiplicity and different spatial symmetry or between states of different spin multiplicity,
the crossing is an hyperplane and its only requirement is the energetic degeneracy and the
proper energy minimization.

Here we include an example with the crossing between the lowest singlet (ground) and triplet 
states of acrolein. Notice that two different states are computed, first by using 
\program{RASSCF} to get the wave function and then \program{ALASKA} to get the gradients
of the energy. Nothing new on that, just the information needed in any geometry optimizations.
The \program{GATEWAY} input requires to add as constraint an energy
difference between both states equal to zero. A specific instruction is required after 
calculating the first state. We have to copy the communication file \file{RUNFILE}
(at that point contains the information about the first state) to \file{RUNFILE2}
to provide later \program{SLAPAF} with proper information about both states:

%%%To_extract{/doc/samples/problem_based_tutorials/CASSCF.S-T_crossing.acrolein.input}
\begin{inputlisting}
*CASSCF singlet-triplet crossing in acrolein
*File: CASSCF.S-T_crossing.acrolein
*
 &GATEWAY
Title= Acrolein molecule
Basis set
O.sto-3g....
 O1             1.5686705444       -0.1354553340        3.1977912036  Angstrom
End of basis
Basis set
C.sto-3g....
 C1            -0.1641585340        0.2420235062       -0.0459895824  Angstrom
 C2             0.1137722023       -0.1389623714        1.3481527296  Angstrom
 C3             1.3218729238        0.1965728073        1.9959513294  Angstrom
End of basis
Basis set
H.sto-3g....
 H1             2.0526602523        0.7568282320        1.4351034056  Angstrom
 H2            -0.6138178851       -0.6941171027        1.9113821810  Angstrom
 H3            -0.8171509745        1.0643342316       -0.2648232855  Angstrom
 H4             0.1260134708       -0.4020589690       -0.8535699812  Angstrom
End of basis
Constraints
   a = Ediff
  Value
   a = 0.000
End of Constraints

>>> Do while
                                                                                                                                                                            
 &SEWARD  
                                                                                                                                                                            
>>> IF ( ITER = 1 ) <<<
 &SCF 
>>> ENDIF <<<
                                                                                                                                                                            
 &RASSCF 
   LumOrb
   Spin= 1; Nactel= 4 0 0; Inactive= 13; Ras2= 4
   CiRoot= 1 1; 1
 &ALASKA
>>COPY $WorkDir/$Project.RunFile $WorkDir/RUNFILE2
                                                                                                                                                                            
 &RASSCF 
   LumOrb
   Spin= 3; Nactel= 4 0 0; Inactive= 13; Ras2= 4
   CiRoot= 1 1; 1
 &ALASKA 
 &SLAPAF 
                                                                                                                                                                            
>>> EndDo
\end{inputlisting}
%%%To_extract

%$
Solvent effects can be also applied to excited states, but first the reaction
field in the ground (initial) state has to be computed. This is because solvation in
electronic excited states is a non equilibrium situation in with the electronic
polarization effects (fast part of the reaction field) have to treated apart
(they supposedly change during the excitation process) from the orientational
(slow part) effects. The slow fraction of the reaction field is maintained from
the initial state and therefore a previous calculation is required. 
From the practical point of view the input is simple as illustrated in the next 
example. First, the proper reaction-field
input is included in \program{SEWARD}, then a \program{RASSCF} and \program{CASPT2}
run of the ground state, with keyword \keyword{RFPErt} in \program{CASPT2},
and after that another SA-CASSCF calculation of five roots to get the wave function
of the excited states. Keyword \keyword{NONEequilibrium} tells the program to extract
the slow part of the reaction field from the previous calculation of the ground
state (specifically from the \file{JOBOLD} file, which may be stored for other
calculations) while the fast part is freshly computed. Also, as it is a SA-CASSCF
calculation (if not, this is not required) keyword \keyword{RFRoot} is introduced
to specify for which of the computed roots the reaction field is generated. We have
selected here the fifth root because it has a very large dipole moment, which is
also very different from the ground state dipole moment. If you compare the excitation
energy obtained for the isolated and the solvated system, a the large red shift is 
obtained in the later.

%%%To_extract{/doc/samples/problem_based_tutorials/CASPT2.excited_solvent.acrolein.input}
\begin{inputlisting}
*CASPT2 excited state in water for acrolein
*File: CASPT2.excited_solvent.acrolein
*
&GATEWAY  
  Title= Acrolein molecule
  coord = acrolein.xyz; basis = STO-3G; group= c1
  RF-input
   PCM-model; solvent= water
  End of RF-input
&SEWARD  
&RASSCF 
  Spin= 1; Nactel= 6 0 0; Inactive= 12; Ras2= 5
  CiRoot= 1 1 1
&CASPT2 
  Multistate= 1 1
  RFPert
&RASSCF 
  Spin= 1; Nactel= 6 0 0; Inactive= 12; Ras2= 5
  CiRoot= 5 5 1
  RFRoot= 5
  NONEquilibrium
&CASPT2 
  Multistate= 1 5
  RFPert
\end{inputlisting}
%%%To_extract

A number of simple examples as how to proceed with the most frequent 
quantum chemical problems computed with \molcas\ have been given above. Certainly there are many more
possibilities in \molcas\ \molcasversion\, such as calculation of 3D band
systems in solids at a semiempirical level, obtaining valence-bond structures,
the use of QM/MM methods in combination with a external MM code, the introduction
of external homogeneous or non homogeneous perturbations, generation of atomic
basis sets, application of different localization schemes, analysis of first
order polarizabilities, calculation of vibrational intensities, analysis, generation,
and fitting of potentials, computation of vibro-rotational spectra for diatomic
molecules, introduction of relativistic effects, etc. All those aspects are
explained in the manual and are much more specific. Next section~\ref{TUT:sec:pg-based-tut}
details the basic structure of the inputs, program by program, while easy examples
can also be found. Later, another chapter includes a number of extremely detailed
examples with more elaborated quantum chemical examples, in which also scientific
comments are included. Examples include calculations on high symmetry molecules,
geometry optimizations and Hessians, computing reaction paths, high quality wave
functions, excited states, solvent models, and computation of relativistic effects.



\chapter{Program Based Tutorials}
\label{TUT:sec:pg-based-tut}

The \molcas\ \molcasversion\ suite of Quantum Chemical programs is modular in
design.  The desired calculation is achieved by executing a list of
\molcas\ program modules in succession, while potentially manipulating
the program information files. If the information files from a previous
calculation are saved, then a subsequent calculation need not recompute
them.  This is dependent on the correct information being preserved in
the information files for subsequent calculations.
Each module has keywords to specify the
functions to be carried out, and many modules rely on the
specification of keywords in previous modules.

The following sections describe the use of the \molcas\ modules and
their inter-relationships.  Each module is introduced in the
approximate order for performing a typical calculation.
A complete flowchart for the \molcas\ \molcasversion\ suite of programs follows.

\clearpage
\section{\molcasversion\ Flowchart}
\label{TUT:sec:flow_all}
\begin{figure}[hbt]
\leavevmode
\flowchart{all}
\caption{Flowchart for Module Dependencies in \molcas }
\label{fig:flow_all}
\end{figure}

\section{Environment and EMIL Commands}

The following are basic and most common commands for the \molcas\ environment variables and input language (EMIL):

\begin{variablelist}
\item[MOLCAS] \molcas\ home directory.
\item[MOLCAS\_MEM] Memory definition in Mb. Default 1024.
\item[MOLCAS\_PRINT] Printing level: 2 Normal, 3 Verbose
\item[MOLCAS\_PROJECT] Name used for the project/files.
\item[MOLCAS\_WORKDIR] Scratch directory for intermediate files.
\end{variablelist}
\begin{commandlist}
\item[$>>$Do While] Start of a loop in an input file for geometry optimization with conditional termination.
\item[$>>$Foreach]  Start of a loop in an input file over a number of items.
\item[$>>$EndDo] End of a loop in an input file.
\item[$>>$If ( condition )] Start of If block.
\item[$>>$EndIf] End of If block.
\item[$>>$Label Mark] Setting the label "Mark" in the input.
\item[$>>$Goto Mark] Forward jump to the label "Mark" skipping that part of the input.
\end{commandlist}


% gateway.tex $ this file belongs to the Molcas repository $*/
\section{GATEWAY - Definition of geometry, basis sets, and symmetry}
\label{TUT:sec:gateway}
\index{GATEWAY}\index{Program!GATEWAY}\index{Integrals}

The program \program{GATEWAY} handles the basic molecular parameters in the
calculation. It generates data that are used in all subsequent calculations.
These data are stored in the \file{RUNFILE}. \program{GATEWAY} is the first
program to be executed, if the \variable{\$WorkDir} directory and the \file{RUNFILE} file 
has not already been generated by a previous calculation.

This tutorial is describes how to set up the basic \molcas\ input for the water molecule. 
For a more general description of the input options for \program{GATEWAY}, please refer to the Users Guide. 
%The input for water is given in 
%Figure~\ref{fig:gateway_input}. 
The first line of the input is the program identifier \&GATEWAY.
Then follows the keyword used is \keyword{TITLe} which will also get
printed in the \program{GATEWAY} section of the calculation output. 
The title line is also saved in the integral file and will appear in subsequent programs. 

\index{GATEWAY!Symmetry}\index{Option!Symmetry}
\index{Symmetry!Generators}\index{Symmetry!Point groups}
\index{GATEWAY!Input}

The \keyword{GROUp} keyword is followed by the generators for the C$_{2v}$ 
point group, since the example deals with the water molecule.
The specification of the C$_{2v}$ point group given in
Table~\ref{tab:symmetry_list} is not unique, but, in this tutorial, the 
generators have been input in an order that reproduces the ordering in the
character tables. A complete list of symmetry generator input syntax is given 
in Table~\ref{tab:symmetry_list}.  The symmetry groups available are listed 
with the symmetry generators defining the group. The \molcas\ keywords required 
to specify the symmetry groups are also listed. The last column contains the 
symmetry elements generated by the symmetry generators.

\begin{inputlisting}
 &GATEWAY
Title= Water in C2v symmetry - A Tutorial
Coord = water.xyz
Group =  XY Y
Basis Set = O.ANO-S-MB,H.ANO-S-MB
\end{inputlisting}

\begin{table}[htbp]
\caption{Symmetries available in MOLCAS including generators, MOLCAS keywords 
and symmetry elements.}
\label{tab:symmetry_list}
\begin{tabular}{c|ccc|ccc|cccccccc}
Group &
\multicolumn{3}{c|}{Generators} &
\multicolumn{3}{c|}{\molcas} &
\multicolumn{8}{c}{Elements} \\
         & $g_1$    & $g_2$      & $g_3$ &$g_1$&$g_2$&$g_3$& $E$ &  $g_1$ & $g_2$ & $g_1g_2$ & $g_3$ & $g_1g_3$ & $g_2g_3$ & $g_1g_2g_3$ \\
\hline
$C_1$    &          &            &       &&&& $E$ &        &       &          &      &      &      &     \\
$C_2$    & $C_2$    &            &       &\keyword{xy}&&& $E$ & $C_2$  &       &          &      &      &      &     \\
$C_s$    & $\sigma$ &            &       &\keyword{x}&&& $E$ & $\sigma$ &       &          &      &      &      &     \\
$C_i$    & $i$      &            &       &\keyword{xyz}&&& $E$ & $i$    &       &          &      &      &      &     \\
$C_{2v}$ & $C_2$    & $\sigma_v$ &       &\keyword{xy}&\keyword{y}&& $E$ & $C_2$  & $\sigma_v$ & $\sigma_v'$ &  &  &  & \\
$C_{2h}$ & $C_2$    & $i$        &       &\keyword{xy}&\keyword{xyz}&& $E$ & $C_2$  & $i$ & $\sigma_h$ &       &      &      &     \\
$D_2$    & $C_2^z$  & $C_2^y$    &       &\keyword{xy}&\keyword{xz}&& $E$ & $C_2^z$ & $C_2^y$ & $C_2^x$ &      &      &      &     \\
$D_{2h}$ & $C_2^z$  & $C_2^y$    & $i$   &\keyword{xy}&\keyword{xz}&\keyword{xyz}& $E$ & $C_2^z$ & $C_2^y$ & $C_2^x$ & $i$ & $\sigma^{xy}$ & $\sigma^{xz}$
& $\sigma^{yz}$ \\
\end{tabular}
\end{table}

To reduce the input, the unity operator $E$ is always assumed. The twofold 
rotation about the z-axis, C$_{2}$($z$), and the reflection in the xz-plane,
$\sigma_v$($xz$), are input as XY and Y respectively.  The \molcas\
input can be viewed as symmetry operators that operate on the
Cartesian elements specified.  For example, the reflection in the
xz-plane is specified by the input keyword \keyword{Y} which is the
Cartesian element operated upon by the reflection. 

The input produces the character table in the 
\program{GATEWAY} section of the output shown in 
Figure~\ref{fig:Tut_C2v_output}. Note that $\sigma_v$($yz$) was produced from 
the other two generators.  The last column contains the basis functions of 
each irreducible symmetry representation.  The totally symmetric $a_1$ 
irreducible representation has the $z$ basis function listed which is unchanged 
by any of the symmetry operations.

%$
\label{fig:Tut_C2v_output}
\begin{verbatim}
                             E   C2(z) s(xz) s(yz)
                    a1       1     1     1     1  z
                    b1       1    -1     1    -1  x, xz, Ry
                    a2       1     1    -1    -1  xy, Rz, I
                    b2       1    -1    -1     1  y, yz, Rx
\end{verbatim}

\index{Character table}
\index{GATEWAY!Test}\index{Input!Comment lines}

\index{Units}\index{GATEWAY!Geometry}\index{GATEWAY!Units}
\index{Coordinates!GATEWAY input}

The geometry of the molecule is defined using the keyword \keyword{coord}. On 
the next line,  the name of the xyz file that defines the geometrical 
parameters of the molecule (\file{water.xyz}) is given. 
\begin{enumerate}
\item The first line of the \file{water.xyz} file contains the number of atoms. 
\item The second line is used to indicate the units: \AA ngstr\"om or atomic units. 

The default is to use \AA ngstr\"om. 
\item Then follows one line for each atom containing the name of each atom and its coordinates. 
\end{enumerate}

Basis sets are defined after the keyword \keyword{BASIs sets}. The oxygen
and hydrogen basis set chosen, for this example, are the small Atomic Natural Orbitals 
(ANO) sets.  There are three contractions of the basis included in the input,
which can be toggled in or excluded with an asterisk, according to the desired calculation:
minimal basis, double zeta basis with polarization, or triple zeta basis with polarization.
\ifmanual
\begin{figure}[ht]
\caption{The geometry of the water molecule}
\label{fig:coord}
\end{figure}
{\footnotesize
\begin{verbatim}
 3

O	       .000000        .000000	     .000000  
H	      0.700000        .000000	    0.700000 
H	     -0.700000        .000000	    0.700000
\end{verbatim} }
\fi


\subsection{\program{GATEWAY} Output}

\index{GATEWAY!Output}
%\index{GATEWAY!RTRN option}
\index{GATEWAY!Geometry}

The \program{GATEWAY} output contains the symmetry character table, basis set 
information and input atomic centers. The basis set information lists the 
exponents and contraction coefficients as well as the type of Gaussian functions
(Cartesian, spherical or contaminated) used.  

The internuclear distances and valence bond angles (including dihedral angles) 
are displayed after the basis set information. 
%There is a keyword, 
%\keyword{RTRN}, which is used to increase the threshold for printing of bond 
%lengths, bond angles and dihedral angles from the default of 3.5 au.
Inertia and rigid-rotor analysis is also included in the output along with
the timing information.

A section of the output that is useful for determining the input to
the \molcas\ module \program{SCF} is the symmetry adapted basis
functions which appears near the end of the \program{GATEWAY} portion
of the output.  This is covered in more detail in the \program{SCF}
tutorial.

The most important file produced by the \program{GATEWAY} module is the 
\file{RUNFILE} which in our case is linked to \file{water.RunFile}.  This is 
the general \molcas\ communications file for transferring data between the
various \molcas\ program modules.  Many of the program modules add
data to the \file{RUNFILE} which can be used in still other modules. A new 
\file{RUNFILE} is produced every time \program{GATEWAY} is run. It should finally
be mentioned that for backwards compatibility one can run \program{MOLCAS} 
without invoking \program{GATEWAY}. The corresponding input and output will 
then be handled by the program \program{SEWARD}.

\ifmanual
\subsection{Basis Set Superposition Error (BSSE)}

\index{GATEWAY!BSSE}
\program{GATEWAY} can operates with several coordinate files, which is convenient
for computing BSSE corrections. \keyword{BSSE} followed by a number marks a XYZ
file which should be treated as dummy atoms. The following example demonstrates
this feature:

\begin{inputlisting}
&GATEWAY
coord = ethanol.xyz
coord = water.xyz
bsse  = 1
basis = ANO-S-MB
NOMOVE
&SEWARD; &SCF
&GRID_IT
NAME = water
***************
&GATEWAY
coord = ethanol.xyz
coord = water.xyz
bsse  = 2
basis = ANO-S-MB
NOMOVE
&SEWARD; &SCF
&GRID_IT
NAME = ethanol
**************
&GATEWAY
coord = ethanol.xyz
coord = water.xyz
basis = ANO-S-MB
NOMOVE
&SEWARD; &SCF
&GRID_IT
NAME = akvavit

\end{inputlisting}

Note, that NOMOVE keyword prevents centering of the molecule, so the computed 
grids are identical. An alternative way to compute density difference is to
modify coordinates, and change an element label to X.

\fi

\subsection{\program{GATEWAY} Basic and Most Common Keywords}

\begin{keywordlist}
\item[Coord] File name or inline number of atoms and XYZ coordinates
\item[BASIs Set] Atom\_label.Basis\_label (for example ANO-L-VTZP)
\item[Group] Full (find maximum), NoSym, or symmetry generators
\item[SYMMetry]  Symmetry generators: X, Y, Z, XY, XZ, YZ, XYZ (in native format)
\item[RICD] On-the-fly auxiliary basis sets.
\item[]
\end{keywordlist}

% seward.tex $ this file belongs to the Molcas repository $*/
\section{SEWARD --- An Integral Generation Program}
\label{TUT:sec:seward}
\index{SEWARD}\index{Program!SEWARD}\index{Integrals}

An {\em ab initio} calculation always requires integrals. In the
\molcas\ suite of programs, this function is supplied by the \program{SEWARD} 
module. \program{SEWARD} computes the one- and two-electron integrals for the
molecule and basis set specified in the input to the program \program{GATEWAY},
which should be run before \program{SEWARD}. \program{SEWARD} can also be used 
to perform some property expectation calculations on the isolated molecule.
The module is also used as an input parser for the reaction field and
numerical quadrature parameters.

We commence our tutorial by calculating the integrals for a water molecule. The 
input is given in Figure~\ref{fig:seward_input}.  Each \molcas\ module
identifies input from a file by the name of the module.  In the case of 
\program{SEWARD}, the program starts with the label
\&SEWARD, which is the first statement in the file shown below.

In normal cases no input is required for \program{SEWARD}, so the following
input is optional. The first keyword used is \keyword{TITLe}. Only the first
line of the title is printed in the output. The first title line is also saved 
in the integral file and appears in any subsequent programs that use the 
integrals calculated by \program{SEWARD}.

%\begin{figure}[ht]
%\caption{Sample input requesting the SEWARD module
% to calculate the integrals for water in C$_{2v}$ symmetry.}
\label{fig:seward_input}
%\end{figure}
\begin{inputlisting}
 &SEWARD
Title
 Water - A Tutorial. The integrals of water are calculated using C2v symmetry
\end{inputlisting}

In more complicated cases more input may be needed, to specify certain types of
integrals, that use of Cholesky decomposition techniques (\keyword{CHOLesky} keyword), etc. We refer to the
specific sections of the Users-Guide for more information.
The output from a \program{SEWARD} calculation is small and contains in principle
only a list of the different types of integrals that are computed.

The integrals produced by the \program{SEWARD} module are stored in
two files in the working directory.  They are ascribed the \program{FORTRAN}
names \file{ONEINT} and \file{ORDINT} which are 
automatically symbolically linked by the \molcas\ script to the file
names \variable{\$Project}\file{.OneInt} and 
\variable{\$Project}\file{.OrdInt}, respectively
or more specifically, in our case, \file{water.OneInt} and
\file{water.OrdInt}, respectively. The default name for each
symbolical name is contained in the corresponding program files of the
directory {\$MOLCAS/shell}.  
The \file{ONEINT} file contains the one-electron integrals.
The \file{ORDINT} contains the ordered and packed two-electron integrals.
Both files are used by later \molcas\ program modules.

%\subsection{SEWARD - Basic and Most Common Keywords}
%\begin{keywordlist}
%\item[Cholesky] Use Cholesky decomposition
%\item[AMFI] Atomic mean-field integrals for relativistic calculations.
%Required for spin-coupling. Automatic for ANO-RCC basis sets
%\item[]
%
%--
%\end{keywordlist}


% tut_scf.tex $ this file belongs to the Molcas repository $*/
\section{SCF --- A Self-Consistent Field program and Kohn Sham DFT}
\label{TUT:sec:scf}
\index{SCF}\index{Program!SCF}
The simplest {\em ab initio} calculations possible use the Hartree-Fock
(HF) Self-Consistent Field (SCF) method with the program name \program{SCF} in 
the \molcas\ suite.  It is possible to calculate the HF energy once we have 
calculated the integrals using the \program{SEWARD} module, although \molcas\
can perform a direct SCF calculation in which the two-electron integrals are
not stored on disk. The \molcas\ implementation performs a closed-shell (all 
electrons are paired in orbitals) and open-shell (Unrestricted Hartree-Fock) 
calculation. It is not possible to perform an Restricted Open-shell Hartree-Fock (ROHF)
calculation with the \program{SCF}. This is instead done using the program
\program{RASSCF}. The \program{SCF} program can also be used to perform
calculations using Kohn Sham Density Functional Theory (DFT). 

The \program{SCF} input for a Hartree-Fock calculation of a water
molecule is given in figure~\ref{fig:scf_input}
which continues our calculations on the water molecule.

There are no compulsory keywords following the program name, \&SCF. If no input
is given the program will compute the SCF energy for a neutral molecule with the
orbital occupations giving the lowest energy. Here, we have used the following
input:  the first is \keyword{TITLe}. As 
with the \program{SEWARD} program, the first line following the keyword is 
printed in the output.

\index{SCF!Occupied}\index{SCF!Input}

No other keyword is required for a closed-shell calculation. The program
will find the lowest-energy electron configuration compatible with the
symmetry of the system and will distribute the orbitals accordingly.
In complex cases the procedure may fail and produce a higher-lying configuration.
It is possible to use the keyword \keyword{OCCUpied}
which specifies the number of occupied orbitals in each symmetry grouping
listed in the \program{GATEWAY} output and given in
Figure~\ref{fig:Irreducible}, forcing the method to converge to the specified
configuration. The basis label and type give an
impression of the possible molecular orbitals that will be obtained in
the SCF calculation.  For example, the first basis function in the $a_1$
irreducible representation is an $s$ type on the oxygen indicating the
oxygen 1$s$ orbital. Note, also, that the fourth basis  function is centered on
the hydrogens, has an $s$ type and is symmetric on both hydrogens as
indicated by both hydrogens having a phase of 1, unlike the sixth basis function
which has a phase of 1 on center 2 (input H1) and -1 on center 3
(generated H1).
As an alternative you can use the keyword \keyword{Charge} with parameters 0 and
1 to indicate a neutral molecule and optimization procedure 1 that searches for
the optimal occupation.

\begin{figure}[ht]
\caption{Sample input requesting the SCF module to calculate the ground Hartree-Fock energy for a neutral water molecule in $C_{2v}$ symmetry.}
\label{fig:scf_input}
\end{figure}
\begin{inputlisting}
 &SCF 
Title= Water - A Tutorial. The SCF energy of water is calculated using C2v symmetry
End of Input
\end{inputlisting}

\index{Symmetry!Adapted basis functions}

\begin{figure}[h]
\caption{Symmetry adapted Basis Functions from a GATEWAY output.}
\label{fig:Irreducible}
\end{figure}
{\footnotesize
\begin{verbatim}
           Irreducible representation : a1
           Basis function(s) of irrep: z

 Basis Label        Type   Center Phase
   1   O1           1s        1     1
   2   O1           2s        1     1
   3   O1           2p0       1     1
   4   H1           1s        2     1      3     1

           Irreducible representation : b1
           Basis function(s) of irrep: x, xz, Ry

 Basis Label        Type   Center Phase
   5   O1           2p1+      1     1
   6   H1           1s        2     1      3    -1

           Irreducible representation : b2
           Basis function(s) of irrep: y, yz, Rx

 Basis Label        Type   Center Phase
   7   O1           2p1-      1     1
\end{verbatim}}

We have ten electrons to ascribe to five orbitals to describe a
neutral water molecule in the ground state. Several
techniques exist for correct allocation of electrons.  As a test of
the electron allocation, the energy obtained should be the same with
and without symmetry.
Water is a simple case, more so when
using the minimal basis set.  In this case, the fourth irreducible
representation is not listed in the \program{GATEWAY} output as there
are no basis functions in that representation.  

\index{SCF!Open-shell cases -Unrestricted Kohn Sham DFT}

To do a UHF calculation, the keyword \keyword{UHF} must be specified. 
To force a specific occupation for alpha and beta orbitals
In this keyword \keyword{OCCNumbers} has to be used with two entries, one
for alpha and beta occupied orbital. It is possible to use UHF
together with keyword \keyword{Charge} or \keyword{Aufbau}, in this case 
you have to specify a keyword \keyword{ZSPIN} set to the
difference between alpha and beta electrons.

If you want to do an UHF calculation for a closed shell system, for example,
diatomic molecule with large interatomic distance, you have to specify keyword
\keyword{SCRAMBLE}.

To do the Density Functional Theory calculations, keyword \keyword{KSDFT} followed
in the next line by the name of the available functional as listed in the input
section is compulsory.  Presently following Functional Keywords are available:
BLYP, B3LYP, B3LYP5, HFB, HFS, LDA, LDA5, LSDA, LSDA5, SVWN, SVWN5, TLYP, XPBE,
MO6, MO6/HF, MO6/2X, MO6/L.
The description of functional keywords and the functionals is defined in the 
\ifmanual
section DFT Calculations \ref{UG:sec:dft}.
\else
online manual for the \program{SCF} program.
\fi

The input for KSDFT is given as,
\begin{inputlisting}
KSDFT= B3LYP5
\end{inputlisting}
In the above example B3LYP5 functional will be used in  KSDFT calculations.


\subsection{Running \program{SCF}}

Performing the Hartree-Fock calculation introduces some important
aspects of the transfer of data between the \molcas\ program modules.
The \program{SCF} module uses the integral files computed by
\program{SEWARD}.  It produces a orbital file with the symbolic name
\file{SCFORB} which contains all the MO information. This is then
available for use in subsequent \molcas\ modules.  The
\program{SCF} module also adds information to the \file{RUNFILE}.
Recall that the \program{SEWARD} module produces two integral files
symbolically linked to \file{ONEINT} and \file{ORDINT} and actually
called, in our case, \file{water.OneInt} and \file{water.OrdInt}, 
respectively (this is for non-Cholesky-type calculations only).  
Because the two integral files are present in 
the working directory when the \program{SCF} module is performed, \molcas\ 
automatically links them to the symbolic names.  

If the integral files were not deleted in a previous calculation
the \program{SEWARD} calculation need not be repeated.  Furthermore,
integral files need not be in the working directory if they are linked
by the user to their respective symbolic names.  Integral files,
however, are often very large making it desirable to remove them after the
calculation is complete.  The linking of files to their symbolic names
is useful in other case, such as input orbitals.

\index{SCF!LumOrb}
\index{SCF!Input orbitals}
\index{SCF!Convergence problems}

If nothing else is stated, the \program{SCF} program will use the guess orbitals
produced by \program{SEWARD} as input orbitals with the internal name 
\file{GUESSORB}. If one wants to use any other input orbitals for the 
\program{SCF} program the option \keyword{LUMOrb} must be used. The
corresponding file should be copied to the internal file \file{INPORB}. This
could for example be an orbital file generated by an earlier SCF calculation, 
\file{\$Project.ScfOrb}. Just copy or link the file as \file{INPORB}.

\subsection{\program{SCF} Output}

\index{SCF!Output}

The \program{SCF} output includes the title from the input as well as
the title from the \program{GATEWAY} input because we used the integrals
generated by \program{SEWARD}.  The output also contains the cartesian
coordinates of the molecule and orbital specifications including the
number of frozen, occupied and virtual (secondary) orbitals in each
symmetry.  This is followed by details regarding the \program{SCF} 
algorithm including convergence criteria and iteration limits.  The 
energy convergence information includes the one-electron, two-electron,
and total energies for each iteration.  This is followed by the final
results including the final energy and molecular orbitals for each
symmetry.

The Density Functional Theory Program gives in addition to the above,
details of grids used, convergence criteria, and name of the functional used. 
This is followed by integrated DFT energy which is the functional contribution 
to the total energy and the total energy including the correlation.
This is followed results including the Kohn Sham orbitals for each symmetry.

\index{SCF!Orbitals}\index{SCF!Orbital energies}
\index{Orbital energies}

The molecular orbital (MO) information lists the orbital energy, the
electron occupation and the coefficients of the basis functions
contributing to that MO.\ For a minimal basis set, the basis functions
correspond directly to the atomic orbitals.  Using larger basis sets
means that a combination of the basis functions will be used for each
atomic orbital and more so for the MOs.
The MOs from the first symmetry species are
given in Figure~\ref{fig:water_MOs}.  The first MO has an energy of 
-20.5611~hartree and an occupation of 2.0.  The major
contribution is from the first basis function label `{\tt O1  1s}'
meaning an {\it s} type function centered on the oxygen atom.  The
orbital
energy and the coefficient indicates that it is the MO based largely
on the oxygen 1{\it s} atomic orbital.  

\begin{figure}[h]
\caption{Molecular orbitals from the first symmetry species of a calculation of water using C$_{2v}$ symmetry and a minimal basis set.}
%$
\label{fig:water_MOs}
\end{figure}
{\footnotesize
\begin{verbatim}
          ORBITAL        1         2         3         4
          EneRGY    -20.5611   -1.3467    -.5957     .0000
          Occ. NO.    2.0000    2.0000    2.0000     .0000

        1 O1  1s      1.0000    -.0131    -.0264    -.0797
        2 O1  2s       .0011     .8608    -.4646    -.7760
        3 O1  2p0      .0017     .1392     .7809    -.7749
        4 H1  1s      -.0009     .2330     .4849    1.5386
\end{verbatim}}

The second MO has a major contribution from the second oxygen {\tt 1s}
basis function indicating a mostly oxygen 2{\it s} construction.  
Note that it is the
absolute value of the coefficient that determines it importance.  The
sign is important for determining the orthogonality of its orbitals and
whether the atomic orbitals contributions with overlap constructively
(bonding) or destructively (anti-bonding).  
The former occurs in this MO as indicated by the positive sign on the
oxygen 2{\it s} and the hydrogen 1{\it s} orbitals, showing a bonding
interaction between them.
The latter occurs in the third MO, where the relative sign is reversed.

The third MO has an energy of 
-0.5957~hartree and major contributions from the second oxygen {\tt
1s} basis function, the oxygen {\tt 2p0} basis function and the 
hydrogen {\tt 1s} basis functions which are symmetrically situated on
each hydrogen (see Figure~\ref{fig:Irreducible}).  The mixing of the
oxygen {\tt 2s} and {\tt 2p0} basis functions leads to a hybrid
orbital that points away from the two hydrogens,  to which it is
weakly antibonding.

A similar analysis of the fourth orbital reveals that it is the 
strongly anti-bonding orbital partner to the third MO.  The oxygen {\tt 2p0}
basis function is negative which reverses the overlap characteristics.

The molecular orbital information is followed by a Mulliken charge
analysis by input center and basis function.  This provides a measure
of the electronic charge of each atomic center.

Towards the end of the \program{SCF} section of the \molcas\ output
various properties of the molecule are displayed.  By default the
first (dipole) and second cartesian moments and the quadrupoles are displayed.
%  The
%inclusion of the \keyword{FLDG} keyword (with zero (0))
%with cause the electric field gradients at each atomic center to be calculated
%and displayed.  
%There are several other properties that can be calculated
%in this fashion using the variational \molcas\ programs -- \program{SCF}
%and \program{RASSCF} when producing a CASSCF wave function.

\subsection{SCF - Basic and Most Common Keywords}
\begin{keywordlist}
\item[UHF] Unrestricted Hartee Fock or unrestricted DFT calculation
\item[KSDFt] DFT calculations, with options: BLYP, B3LYP, B3LYP5, HFB, HFS, 
LDA, LDA5, LSDA, LSDA5, SVWN, SVWN5, TLYP, PBE, PBE0
\item[CHARge] Net charge of the system (default zero)
\item[ZSPIn] Difference between $\alpha$ and $\beta$ electrons
\item[Occupied] Specify the orbital occupations per irreps
\item[]
%--
\end{keywordlist}

% mbpt2.tex $ this file belongs to the Molcas repository $*/
\section{MBPT2 --- A Second-Order Many-Body PT RHF Program}
\label{TUT:sec:mbpt2}
\index{MP2}\index{Program!MBPT2}

The \program{MBPT2} program performs second-order Many Body Perturbation
Theory calculations based on a RHF-type of wave function (MP2 method).
The calculation is to some extent defined by the SCF
calculation which must be performed before running the \program{MBPT2}
program. Therefore, there is no difficulty related to the input file
unless an analysis of the correlation energies of specific electron
pairs or contribution from external orbitals wants to be performed.
In this case keywords \keyword{SFROzen} and \keyword{SDELeted} have to
be used as described in 
\ifmanual
section~\ref{UG:sec:mbpt2} 
\else
MBPT2 section
\fi
of the user's guide.


To run the program the \file{ORDINT} integral file(s)
generated by the \program{SEWARD} program and the \file{RUNFILE} file generated
by the \program{SCF} program are needed. The program can be otherwise run in a
direct manner. Therefore the \program{SEWARD} program can be run
with the option \keyword{DIREct} included in its input. Only the \file{ONEINT}
will then be generated and used by the \program{SCF} module. 
The input file used to run an \program{MBPT2} calculation on the ground state 
of the water molecule is displayed in figure~\ref{fig:mbpt2_input}. For large
molecules it is also possible to use the Cholesky decomposition technique to
speed up the calculations. This will be described in another section of the
tutorials.

\begin{inputlisting}
 &MBPT2
Title= MP2 of ground state of C2v Water 
Frozen= 1 0 0 0
\end{inputlisting}
\begin{figure}[h]
\caption{Sample input requested by the MBPT2 module to
calculate the MP2 energy for the ground state of the water in C$_{2v}$ symmetry.}
\label{fig:mbpt2_input}
\end{figure}

The output of \program{MBPT2} is self-explanatory.

%\subsection{MBPT2 - Basic and Most Common Keywords}
%\begin{keywordlist}
%\item[FROZEN] By symmetry: non-correlated orbitals (default: core)
%\item[]
%
%--
%\end{keywordlist}


% tut_rasscf.tex $ this file belongs to the Molcas repository $*/
\section[RASSCF --- A Multi Configurational SCF Program]
        {RASSCF --- A Multi Configurational 
                                Self-Consistent Field Program}
\label{TUT:sec:rasscf}
\index{RASSCF}\index{CASSCF (see RASSCF)}
\index{Program!RASSCF}
One of the central codes in \molcas\ is the \program{RASSCF} program, which
performs multiconfigurational SCF calculations. Both Complete Active Space 
(CASSCF) and Restricted Active Space (RASSCF) SCF calculations can be performed 
with the \program{RASSCF} program module \cite{Roos:92}.  
An open shell Hartree-Fock calculation is not possible with the \program{SCF}
but it can be performed using the \program{RASSCF} module. An input listing for 
a CASSCF calculation of water appears in Figure~\ref{fig:rasscf_input}.
\program{RASSCF} requires orbital information of the system which can be 
obtained in two ways. The \keyword{LUMOrb} indicates that the orbitals should be
taken from a user defined orbital file, which is copied to the internal file
INPORB. If this keyword is not given, the program will look for orbitals on the
runfile in the preference order: \file{RASORB}, \file{SCFORB} and
\file{GUESSORB}

\begin{figure}[htbp]
\caption{Sample input requesting the RASSCF module to calculate the 
eight-electrons-in-six-orbitals CASSCF energy of the second excited triplet 
state in the second symmetry group of a water molecule in C$_{2v}$ symmetry.}
%$
\label{fig:rasscf_input}
\begin{inputlisting}
 &RASSCF 
Title= The CASSCF energy of water is calculated using C2v symmetry. 2 3B2 state.
nActEl= 8 0 0
Inactive= 1 0 0 0; Ras2= 3 2 0 1
Symmetry= 2; Spin= 3
CIRoot= 1 2; 2
LumOrb
\end{inputlisting}
\end{figure}

The \keyword{TITLe} performs the same function as in the previous \molcas\
modules. The keyword \keyword{INACtive} specifies the number of doubly occupied
orbitals in each symmetry that will not be included in the electron excitations 
and thus remain doubly occupied throughout the calculation. A diagram of the 
complete orbital space available in the \program{RASSCF} module is given in 
Figure~\ref{fig:rasscf_space}.

In our calculation, we have placed the oxygen 1$s$ orbital in the inactive 
space using the \keyword{INACtive} keyword. The keyword \keyword{FROZen} can be 
used, for example, on heavy atoms to reduce the Basis Set
Superposition Error (BSSE). The corresponding orbitals will then not be
optimized. The \keyword{RAS2} keyword specifies the number of orbitals in each
symmetry to be included in the electron excitations with all possible 
occupations allowable. Because the \keyword{RAS1} and \keyword{RAS3} spaces are 
zero (not specified in the input in Figure~\ref{fig:rasscf_input}) the 
\program{RASSCF} calculation will produce a CASSCF wave function.  The 
\keyword{RAS2} space is chosen to use all the orbitals available in each 
symmetry (except the oxygen 1$s$ orbital). The keyword \keyword{NACTel} 
specifies the number of active electrons (8), maximum number of holes in the 
Ras1 space (0) and the maximum number of electrons in the Ras3 space (0).  
Using the keywords \keyword{RAS1} and/or \keyword{RAS3} to specify orbitals and 
specifying none zero numbers of holes/electrons will produce a RASSCF wave 
function.We are, therefore, performing an 8in6 CASSCF calculation of
water. 

\begin{table}[htbp]
\begin{center}
\caption{Examples of types of wave functions obtainable using the RAS1 and RAS3 spaces in the RASSCF module.}
\label{tab:RAS1_3}
\begin{tabular}{lccc} \hline
&Number of holes&&Number of electrons \\
Description&in \keyword{RAS1} orbitals&\keyword{RAS2} orbitals&in \keyword{RAS3} orbitals\\ \hline
SD-CI   &2      &0      &2\\
SDT-CI  &3      &0      &3\\
SDTQ-CI &4      &0      &4\\
Multi Reference SD-CI   &2      &$n$    &2\\
Multi Reference SD(T)-CI &3     &$n$    &2\\
\hline
\end{tabular}
\end{center}
\end{table}

\index{Active space}\index{CI}

There are a number of wave function types that can be performed by manipulating
the \keyword{RAS1} and \keyword{RAS3} spaces. Table~\ref{tab:RAS1_3} lists
a number of types obtainable.  The first three are Configuration
Interaction (CI) wave functions of increasing magnitude culminating with a
Single, Double, Triples and Quadruples (SDTQ) CI.  These can become
multi reference if the number of \keyword{RAS2} orbitals is non-zero.
The last type provides some inclusion of the triples excitation by
allowing three holes in the \keyword{RAS1} orbitals but save
computation cost by only allowing double excitations in the \keyword{RAS3}
orbitals.
\index{RASSCF!Symmetry}\index{RASSCF!Spin}\index{RASSCF!CIroot}
\index{RASSCF!Level-shift}\index{RASSCF!Iterations}

\begin{figure}
\scalebox{1.00}{\myincludegraphics{tutorials/rasscf}}
\caption{RASSCF orbital space including keywords and electron occupancy ranges.}
\label{fig:rasscf_space}
\end{figure}
The symmetry of the wave function is specified using the
\keyword{SYMMetry} keyword.  It specifies the number of the symmetry 
subgroup in the calculation.  We have chosen the second symmetry
species, b$_2$, for this calculation.  We have also chosen the triplet
state using the keyword \keyword{SPIN}. The keyword \keyword{CIROot} has been 
used to instruct \program{RASSCF} to find the second excited state in the 
given symmetry and spin. This is achieved by specifying the number of roots,
1, the dimension of the small CI matrix which must be as large as the
highest required root and the number of the required second root.  
Only for averaged calculations \keyword{CIROot} needs an additional line
containing the weight of the selected roots (unless equal weights are used for
all states).

As an alternative to giving inactive and active orbital input we can use the
type index input on the \file{INPORB} and indicate there which type the
different orbitals should belong to: frozen (f), inactive (i), RAS1 (1), RAS2
(2), RAS3 (3), secondary (s), or deleted (d). This approach is very useful when the input
orbitals have been run through \program{LUSCUS}, which is used to select the
different subspaces. \program{LUSCUS} will relabel to orbitals according to the
users instructions and the corresponding orbital file ,\file{GvOrb} can be
linked as the \file{INPORB} in the \program{RASSCF} program without any 
further input.

\index{Convergence problems!In RASSCF}
A level shift was included using the \keyword{LEVShift} keyword 
to improve convergence of the calculation. In this case, the calculation
does not converge without the use of the level shift.  It is advisable to
perform new calculations with a non-zero \keyword{LEVShift} value (the default
value is 0.5). Another possibility is to increase the maximum number of 
iterations for the macro and the super-CI Davidson procedures
from the default values (200,100) using the keyword \keyword{ITERations}. 

Sometimes convergence problems might appear when the wave function is
close to fulfill all the convergence criteria. An infrequent but possible 
divergence might appear in a calculation starting from orbitals of an already 
converged wave function, or in cases where the convergence thresholds
have been decreased below the default values.
Option \keyword{TIGHt} may be useful in those cases. It contains the 
thresholds criteria for the Davidson diagonalization procedure. In situations
such as those described above it is recommended to decrease the first
parameter of \keyword{TIGHt} to a value lower than the default, for instance
1.0d-06.

\subsection{\program{RASSCF} Output}

\index{RASSCF!Output}
\index{RASSCF!CI coefficients}
\index{RASSCF!Configurations}
\index{RASSCF!Natural occupation}

The \program{RASSCF} section of the \molcas\ output contains similar 
information to the \program{SCF}
output. Naturally, the fact that we have requested an excited state is
indicated in the output.  In fact, both the lowest triplet state and the first
excited state or second root are documented including energies.  
For both of these states the CI
configurations with a coefficient greater than 0.05 are printed along 
with the partial electron distribution in the active space.
Figure~\ref{fig:RASSCF_CI} shows the relevant output for the second
root calculated.   There are three configurations with a CI-coefficient 
larger than 0.05 and two with very much larger values.  The number of the
configuration is given in the first column and the CI-coefficient and
weight are given in the last two columns.  The electron occupation of the
orbitals of the first symmetry for each configuration is given under the 
`{\tt 111}' using `{\tt 2}' for a fully occupied orbital and `{\tt u}' 
for a singly occupied orbital containing an electron with an up spin.  
The down spin electrons are represented with a `{\tt d}'. The occupation 
numbers of the active space for each symmetry is given below the contributing 
configurations. It is important to remember that the active orbitals are
not ordered by any type of criterion within the active space.

\begin{figure}[h]
\caption{RASSCF portion of output relating to CI configurations and electron 
occupation of natural orbitals.}
\label{fig:RASSCF_CI}
\end{figure}
{\footnotesize
\begin{verbatim}
      printout of CI-coefficients larger than   .05 for root   2
      energy=    -75.443990
      conf/sym  111 22 4     Coeff  Weight
             3  22u u0 2    .64031  .40999
             4  22u 0u 2    .07674  .00589
            13  2u0 2u 2   -.75133  .56450
            14  2u0 u2 2    .06193  .00384
            19  udu 2u 2    .06489  .00421

      Natural orbitals and occupation numbers for root  2
      sym 1:   1.986957   1.416217    .437262
      sym 2:   1.567238    .594658
      sym 4:   1.997668
\end{verbatim}}

The molecular orbitals are displayed in a similar fashion to the
\program{SCF} section of the output except that the energies of the
active orbitals are not defined and therefore are displayed as zero and
the electron occupancies are those calculated by the \program{RASSCF}
module. In a state average calculation (more than one root calculated),
the MOs will be the natural orbitals corresponding to the state
averaged density matrix (called pseudo-natural orbitals) and the occupation 
numbers will be the corresponding eigenvalues.  Natural orbital occupation 
numbers for each state are printed as shown in Figure~\ref{fig:RASSCF_CI}, but 
the MOs specific to a given state are not shown in the output.  They are,
however, available in the \file{JOBIPH} file.  A number of molecular
properties are also computed for the requested electronic state in a similar 
fashion to the \program{SCF} module. 


\subsection{Storing and Reading \program{RASSCF} Orbitals and Wave Functions}
\label{TUT:sec:rasread}
\index{Program!RASREAD (obsolete)}\index{RASREAD (obsolete)}
\index{Files!JOBIPH}
\index{Files!RASORB}
\index{Convergence problems!In RASSCF}

Part of the information stored in the \program{RASSCF} output file, \file{JOBIPH}, 
for instance the molecular orbitals and occupation numbers can be also found
in an editable file named \file{RASORB}, which is automatically generated by 
\program{RASSCF}. In case more than one root is used the natural orbitals are
also stored in files \file{RASORB.1}, \file{RASORB.2}, etc, up to ten. In such 
cases the file \file{RASORB} contains the averaged orbitals. If more roots 
are used the files can be generated using the \keyword{OUTOrbitals} keyword. 
The type of orbital produced can be either \keyword{AVERaged}, 
\keyword{NATUral}, \keyword{CANOnical} or \keyword{SPIN} (keywords) orbitals.  
The \keyword{OUTOrbitals} keyword, combined with the \keyword{ORBOnly} keyword,
can be used to read the \file{JOBIPH} file and produce 
an orbital file, \file{RASORB}, which can be read by a subsequent
\program{RASSCF} calculation using the same input section.
The formatted \file{RASORB} file is useful to operate on the orbitals in order
to obtain appropriate trial orbitals for a subsequent \program{RASSCF}
calculation. In particular the type index can be changed
directly in the file if the \program{RASSCF} program has converged to a solution 
with wrong orbitals in  the active space. The \program{RASSCF} program
will, however, automatically place the orbital files from the calculation in the
user's home directory under the name \file{\$Project.RasOrb}, etc. In 
calculations with spin different from zero the program will also produce the 
spin orbital files \file{\$Project.SpdOrb1}, etc for each state. These orbitals 
can be used by the program \program{LUSCUS} to produce spin densities. 

\subsection{RASSCF - Basic and Most Common Keywords}
\begin{keywordlist}
\item[SYMMetry] Symmetry of the wave function (according to \program{GATEWAY})
(1 to 8)
\item[SPIN] Spin multiplicity
\item[CHARGE] Molecular charge
\item[NACTel] Three numbers: Total number of active electrons, holes in Ras1, particles in Ras3
\item[INACtive] By symmetry: doubly occupied orbitals
\item[RAS1] By symmetry: Orbitals in space Ras1 (RASSCF)
\item[RAS2] By symmetry: Orbitals in space Ras1 (CASSCF and RASSCF)
\item[RAS3] By symmetry: Orbitals in space Ras1 (RASSCF)
\item[CIROot] Three numbers: number of CI roots, dimension of the CI matrix, relative weights
(typically 1)
\item[LUMORB/FILEORB] use definition of active space from Orbital file
\item[]
%--
\end{keywordlist}


% tut_caspt2.tex $ this file belongs to the Molcas repository $*/
\section{CASPT2 --- A Many Body Perturbation Program} 
\label{TUT:sec:caspt2}
\index{CASPT2}\index{Program!CASPT2}
Dynamic correlation energy of a molecular system can be calculated using
the \program{CASPT2} program module in \molcas.  A \program{CASPT2}
calculation gives a second order perturbation estimate of the full CI energy 
using the CASSCF wave function of the system. 
The program can also perform Multi-State CASPT2 calculations (MS-CASPT2) in 
which different CASPT2 states are coupled using an effective Hamiltonian
computed to second order in perturbation theory. This is necessary in cases
where different CASSCF wave functions are strongly dependent on dynamical
correlation effects. The wave function have to be obtained in a previous 
State-Average CASSCF calculation.

\index{CASPT2!Input}

A sample input is given in Figure~\ref{fig:caspt2_input}. The
\keyword{FROZen} keyword specifies the number of orbitals of each
symmetry which will not be included in the correlation.  We have
chosen the \program{RASSCF} \keyword{INACtive} orbitals to be frozen
for this calculation (the default is to freeze all core orbitals, so the input
is strictly not needed).  The remaining two keywords, \keyword{CONVergence} and 
\keyword{MAXIter}, are included with there default values. The 
\keyword{MULTistate} keyword is included for clarity even if not needed in this single
state calculation. A single line follows indicating the number of 
simultaneously treated CASPT2 roots and the number of the roots in the previous 
SA-CASSCF calculation.

\subsection{\program{CASPT2} Output}

\index{CASPT2!Output}

\ifmanual
In section~\ref{TUT:sec:pt2out} the meaning and significance of most of the
features used and printed by the \program{CASPT2} program are explained in the 
context of an actual example. We suggest a careful reading of that section
because understanding the results of a CASPT2 calculation is important for
the analysis of problems like intruder states, large coefficients, convergence,
etc.
\fi

\begin{figure}[ht]
\caption{Sample input requesting the CASPT2 module to calculate the CASPT2 
energy of a water molecule in C$_{2v}$ symmetry with one frozen orbital.}
\label{fig:caspt2_input}
\end{figure}
\begin{inputlisting}
 &CASPT2
Frozen= 1 0 0 0
Multistate= 1 1
MaxIter= 40
\end{inputlisting}

The output of the \program{CASPT2} program begins with the title
from the input as well as the title from the \program{SEWARD} input.
It also contains the cartesian coordinates of the molecule and the
CASSCF wave function and orbital specifications. This is followed by
details about the type of Fock and H$_0$ operator used and, eventually,
the value of the level-shift parameter employed. It is possible then
to obtain, by input specifications, the quasi-canonical orbitals in
which the wave function will be represented. The following CI vector
and occupation number analysis will be performed using the 
quasi-canonical orbitals.

Two important sections follow. First a detailed report on small energy 
denominators, large components, and large energy contributions which will 
inform about the reliability of the calculation 
\ifmanual
(see section~\ref{TUT:sec:pt2out}) 
\fi
and finally the \program{CASPT2} property section 
including the natural orbitals obtained
as defined in the output and a number of approximated molecular properties.

If the \keyword{MULTistate} option is used, the program will perform one CASPT2 
calculation for each one of the selected roots, and finally the complete 
effective Hamiltonian containing the selected states will be solved to obtain
the final MS-CASPT2 energies and PM-CASSCF wave functions \cite{Finley:98b}.

The \program{CASPT2} module needs the integral files in \$WorkDir and the 
\file{RUNFILE} file from the and the \file{JOBIPH} file from the 
\program{RASSCF} module. The orbitals are saved in the \file{PT2ORB} file.
The new PM-CASSCF wave functions generated in a MS-CASPT2 calculation
is saved in the \file{JOBMIX} file.

\subsection{CASPT2 - Basic and Most Common Keywords}
\begin{keywordlist}
\item[MULTistate] Multi-State CASPT2 calculation: number of roots and roots (Ex. 3 1 2 3)
\item[IMAG] Value for the imaginary shift for the zero order Hamiltonian 
\item[]
%--
\end{keywordlist}


% tut_rassi.tex $ this file belongs to the Molcas repository $*/
\section{RASSI --- A RAS State Interaction Program}
\label{TUT:sec:rassi}
\index{Program!RASSI}\index{RASSI}
\index{Properties!With RASSI}\index{Properties!Expectation values}
\index{Properties!Matrix elements}
\index{Matrix elements}\index{Expectation values}

Program \program{RASSI} (RAS State Interaction) computes matrix elements
of the Hamiltonian and other operators in a wave function basis, which
consists of individually optimized CI expansions from the \program{RASSCF}
program. Also, it solves the Schr\"odinger equation within the space of
these wave functions. There are many possible applications for such type
of calculations. The first important consideration to have into account
is that \program{RASSI} computes the interaction among RASSCF states
expanding the same set of configurations, that is,
having the same active space size and number of electrons.

The \program{RASSI} program is routinely used to compute electronic
transition moments, as it is shown in the Advanced Examples in the
calculation of transition dipole moments for the
excited states of the thiophene molecule using CASSCF-type wave functions. 
By default the program will compute the matrix elements and expectation values 
of all the operators for which \program{SEWARD} has computed the integrals
and has stored them in the \file{ONEINT} file. 

\index{Non-orthogonal states}

RASSCF (or CASSCF) individually optimized states are interacting and
non-orthogonal. It is imperative when the states involved have different
symmetry to transform the states to a common eigenstate basis in such
a way that the wave function remains unchanged. The State Interaction
calculation gives an unambiguous set of non-interacting and orthonormal
eigenstates to the projected Schr\"odinger equation and also the
overlaps between the original RASSCF wave functions and the eigenstates.
The analysis of the original states in terms of RASSI eigenstates is
very useful to identify spurious local minima and also to inspect the
wave functions obtained in different single-root RASSCF calculations,
which can be mixed and be of no help to compare the states.

Finally, the \program{RASSI} program can be applied in situations when
there are two strongly interacting states and there are two very different 
MCSCF solutions. This is a typical situation in transition metal chemistry
when there are many close states associated each one to a configuration
of the transition metal atom. It is also the case when there are two
close quasi-equivalent localized and delocalized solutions. \program{RASSI}
can provide with a single set of orbitals able to represent, for instance,
avoided crossings. \program{RASSI} will produce a
number of files containing the natural orbitals for
each one of the desired eigenstates to be used in subsequent calculations.

\program{RASSI} requires as input files the \file{ONEINT} and \file{ORDINT}
integral files and the \file{JOBIPH} files from the \program{RASSCF} program
containing the states which are going to be computed. The \file{JOBIPH} files
have to be named consecutively as \file{JOB001}, \file{JOB002}, etc.
The input for the \program{RASSI} module has to contain at least
the definition of the number of states available in each of the input
\file{JOBIPH} files. Figure~\ref{fig:rassi_input} lists the input file
for the \program{RASSI} program in a calculation including two \file{JOBIPH} 
files (2 in the first line), the first one including three roots (3 in the first
line) and the second five roots (5 in the first line). Each one of the 
following lines lists the number of these states within each \file{JOBIPH} file.
Also in the input, keyword \keyword{NATOrb} indicates that three files
(named sequentially \file{NAT001}, \file{NAT002}, and \file{NAT003}) will
be created for the three lowest eigenstates.

\index{RASSI!Input}

\begin{figure}[ht]
\caption{Sample input requesting the RASSI module to calculate the matrix 
elements and expectation values for eight interacting RASSCF states}
\label{fig:rassi_input}
\end{figure}
\begin{inputlisting}
 &RASSI
NROFjobiph= 2 3 5; 1 2 3; 1 2 3 4 5
NATOrb= 3
\end{inputlisting}

\subsection{\program{RASSI} Output}

\index{RASSI!Output}

The \program{RASSI} section of the \molcas\ output is basically divided
in three parts. Initially, the program prints the information about the
\file{JOBIPH} files and input file, optionally prints the wave functions,
and checks that all the configuration spaces are the same in all the
input states. In second place \program{RASSI} prints the expectation
values of the one-electron operators, the Hamiltonian matrix, the
overlap matrix, and the matrix elements of the one-electron operators,
all for the basis of input RASSCF states. The third part starts with
the eigenvectors and eigenvalues for the states computed in
the new eigenbasis, as well as the overlap of the computed eigenstates
with the input RASSCF states. After that, the expectation values and
matrix elements of the one-electron operators are repeated on the
basis of the new energy eigenstates. A final section informs about 
the occupation numbers of the natural orbitals computed by 
\program{RASSI}, if any.

In the Advanced Examples a detailed example of how to interpret
the matrix elements output section for the thiophene molecule is
displayed. The rest of the output is self-explanatory. It has to be
remembered that to change the default origins for the one electron
operators (the dipole moment operator uses the nuclear charge
centroid and the higher order operators the center of the nuclear
mass) keyword \keyword{CENTer} in \program{GATEWAY} must be used.
Also, if multipoles higher than order two are required, the
option \keyword{MULTipole} has to be used in \program{GATEWAY}.

The program \program{RASSI} can also be used to compute a spin-orbit Hamiltonian
for the input CASSCF wave functions as defined above. The keyword \keyword{AMFI}
has to be used in \program{SEWARD} to ensure that the corresponding integrals
are available.

\begin{figure}[ht]
\caption{Sample input requesting the RASSI module to calculate and diagonalize 
the spin-orbit Hamiltonian the ground and triplet excited state in water.}
\label{fig:rassi_input1}
\end{figure}
\begin{inputlisting}
 &RASSI
NROFjobiph= 2 1 1; 1; 1
Spinorbit 
Ejob
\end{inputlisting}

The first \file{JOBMIX} file contains the wave function for the ground state and
the second file the $^3B_2$ state discussed above. The keyword \keyword{Ejob}
makes the \program{RASSI} program use the CASPT2 energies which have been 
written on the \file{JOBMIX} files in the diagonal of the spin-orbit
Hamiltonian. The output of this calculation will give four spin-orbit states and
the corresponding transition properties, which can for example be used to
compute the radiative lifetime of the triplet state.

\subsection{RASSI - Basic and Most Common Keywords}
\begin{keywordlist}
\item[NROFjob] Number of input files, number of roots, and roots for each file
\item[EJOB/HDIAG] Read energies from input file / inline
\item[SPIN] Compute spin-orbit matrix elements for spin properties
\item[]

%--
\end{keywordlist}


\ifmanual
%! tut_casvb.tex $ this file belongs to the Molcas repository $*/
\subsection{CASVB --- A non-orthogonal MCSCF program} 
\label{TUT:sec:casvb}
\index{casvb}\index{Program!CASVB}
\program{CASVB} is a program for carrying out quite general types of 
non-orthogonal MCSCF calculations, offering, for example, all the advantages 
associated with working within a valence bond formalism.

{\bf Warning:} as for any general MCSCF program, one may experience convergence 
problems, ({\em e.g.,}\/ due to redundant parameters), and the non-orthogonal 
optimization of orbitals can furthermore give linear dependency problems. 
Several options in \program{CASVB} can help overcoming these difficulties.

This program can be used in two basic modes:
\begin{itemize}
\item[a)] fully variational optimization
\item[b)] representation of CASSCF wavefunctions using
   overlap- ({\em relatively inexpensive\/}) or energy-based criteria.
\end{itemize}

\program{CASVB} executes the following logical steps:
Setup of wavefunction information, starting guess generation, one, or several,
optimization steps, various types of analysis of the converged solution.

\subsection{\program{CASVB} input}
\index{casvb!Input}
\program{CASVB} attempts to define defaults for as many input quantities as 
possible, so that in the simplest case no input to the \program{CASVB} module 
is required.
Sample input for a CASVB calculation on the lowest singlet state of CH$_2$:
%%%To_extract{/doc/samples/tutorials/CASVB.CH2.input}
\begin{inputlisting}
 &GATEWAY
coord
3
ch2 molecule
C 0.000000  0.000000 0.000000
H 0.000000  0.892226 0.708554
H 0.000000 -0.892226 0.708554

group= x y; basis= sto-3g
 &SEWARD
 &SCF
 &RASSCF
nactel= 6 0 0; inactive= 1 0 0 0; ras2= 3 1 2 0
lumorb
 &CASVB
\end{inputlisting}
%%%To_extract

\subsection{\program{CASVB} output}
\index{casvb!Output}
The amount of output in \program{CASVB} depends heavily on the setting of the
\keyword{PRINT} levels. In case of problems with convergence behaviour it is
recommended to increase these from their rather terse default values.

In the following the main features of the output are outlined, exemplified by 
the job in the input above. Initially, all relevant information
from the previous \program{RASSCF} calculation is recovered from the 
\file{JOBIPH} interface file, after which the valence bond wavefunction 
information is summarized, as shown below. Since 
spatial configurations have not been specified explicitly in this example, a 
single covalent configuration is chosen as default. This gives 5 spin-adapted 
VB structures.

{\begin{footnotesize}
\begin{verbatim}
 Number of active electrons :   6
           active orbitals  :   6
           Total spin       : 0.0
           State symmetry   :   1

 Spatial VB configurations
 -------------------------
     Conf. =>   Orbitals
       1   =>    1  2  3  4  5  6

 Number of VB configurations :     1
           VB structures     :     5
           VB determinants   :    20
\end{verbatim}
\end{footnotesize}}

The output from the following optimization steps summarizes only the most 
relevant quantities and convergence information at the default print level. For 
the last optimization step, for example, The output below thus
states that the VB wavefunction was found by maximizing the overlap with a 
previously optimized CASSCF wavefunction (output by the \program{RASSCF} 
program), and that the spin adaptation was done using the Yamanuchi-Kotani 
scheme. Convergence was reached in 7 iterations.

{\begin{footnotesize}
\begin{verbatim}
 -- Starting optimization - step  3 --------

 Overlap-based optimization (Svb).

 Optimization algorithm:            dFletch
 Maximum number of iterations:           50
 Spin basis:                         Kotani

 -------------------------------------------
 Optimization entering local region.
 Converged ... maximum update to coefficient:  0.59051924E-06
 Final Svb :    0.9978782695
 Number of iterations used:   7
\end{verbatim}
\end{footnotesize}}

Finally in the output below the converged 
solution is printed; orbital coefficients (in terms of the active CASSCF MOs) 
and structure coefficients. The overlap between orbitals are generally of 
interest, and, as also the structures are non-orthogonal, the structure weights 
in the total wavefunction. The total VB wavefunction is not symmetry-adapted 
explicitly (although one may ensure the correct symmetry by imposing constraints
on orbitals and structure coefficients), so its components in the various 
irreducible representations can serve to check that it is physically plausible 
(a well behaved solution generally has just one non-vanishing component).

Next follows the one-electron density with natural-orbital analysis, again with 
quantities printed in the basis of the active CASSCF MOs.

{\begin{footnotesize}
\begin{verbatim}
 Orbital coefficients :
 ----------------------
           1           2           3           4           5           6
   1  0.43397359 -0.43397359 -0.79451779 -0.68987187 -0.79451780 -0.68987186
   2 -0.80889967  0.80889967 -0.05986171 -0.05516284 -0.05986171 -0.05516284
   3  0.00005587 -0.00005587  0.20401015 -0.20582094  0.20401016 -0.20582095
   4  0.39667145  0.39667145  0.00000000  0.00000000  0.00000000  0.00000000
   5 -0.00000001 -0.00000001 -0.53361427 -0.65931951  0.53361425  0.65931952
   6  0.00000000  0.00000000  0.19696124 -0.20968879 -0.19696124  0.20968879

 Overlap between orbitals :
 --------------------------
           1           2           3           4           5           6
   1  1.00000000 -0.68530352 -0.29636622 -0.25477647 -0.29636623 -0.25477647
   2 -0.68530352  1.00000000  0.29636622  0.25477647  0.29636623  0.25477646
   3 -0.29636622  0.29636622  1.00000000  0.81994979  0.35292419  0.19890631
   4 -0.25477647  0.25477647  0.81994979  1.00000000  0.19890634  0.04265679
   5 -0.29636623  0.29636623  0.35292419  0.19890634  1.00000000  0.81994978
   6 -0.25477647  0.25477646  0.19890631  0.04265679  0.81994978  1.00000000

 Structure coefficients :
 ------------------------
      0.00000000  0.00000001  0.09455957  0.00000000 -0.99551921
 
 Saving VB wavefunction to file VBWFN.
 
 Saving VB CI vector to file JOBIPH.
 
 Svb :          0.9978782695
 Evb :        -38.4265149062

 Chirgwin-Coulson weights of structures :
 ----------------------------------------
 VB spin+space (norm   1.00000000) :
      0.00000000  0.00000000 -0.00211737  0.00000000  1.00211737
 VB spin only  (norm   0.38213666) :
      0.00000000  0.00000000  0.00894151  0.00000000  0.99105849

 Symmetry contributions to total VB wavefunction :
 -------------------------------------------------
 Irreps 1 to 4 :  0.10000000E+01  0.15118834E-17  0.17653074E-17  0.49309519E-17

 Energies for components > 1d-10 :
 ---------------------------------
 Irreps 1 to 4 : -0.38426515E+02  0.00000000E+00  0.00000000E+00  0.00000000E+00

 One-electron density :
 ----------------------
           1           2           3           4           5           6
   1  1.98488829 -0.00021330  0.00011757  0.00000000  0.00000000  0.00000000
   2 -0.00021330  1.90209222 -0.00006927  0.00000000  0.00000000  0.00000000
   3  0.00011757 -0.00006927  0.02068155  0.00000000  0.00000000  0.00000000
   4  0.00000000  0.00000000  0.00000000  0.09447774  0.00000000  0.00000000
   5  0.00000000  0.00000000  0.00000000  0.00000000  1.97572540 -0.00030574
   6  0.00000000  0.00000000  0.00000000  0.00000000 -0.00030574  0.02213479

 Natural orbitals :
 ------------------
           1           2           3           4           5           6
   1 -0.99999668  0.00000000  0.00257629  0.00000000  0.00000000  0.00005985
   2  0.00257628  0.00000000  0.99999668  0.00000000  0.00000000 -0.00003681
   3 -0.00005995  0.00000000 -0.00003666  0.00000000 -0.00000001 -1.00000000
   4  0.00000000  0.00000000  0.00000000  1.00000000  0.00000001  0.00000000
   5  0.00000000  0.99999999  0.00000000  0.00000000  0.00015650  0.00000000
   6  0.00000000 -0.00015650  0.00000000 -0.00000001  0.99999999 -0.00000001

 Occupation numbers :
 --------------------
           1           2           3           4           5           6
   1  1.98488885  1.97572545  1.90209167  0.09447774  0.02213475  0.02068154
\end{verbatim}
\end{footnotesize}}

\subsection{Viewing and plotting VB orbitals}
\index{casvb!Plotting}
In many cases it can be helpful to view the shape of the converged valence bond 
orbitals. \molcas\ therefore provides two facilities for doing this. For the 
Molden program, an interface file is generated at the end of each 
\program{CASVB} run (see also Section~\ref{UG:sec:Molden}). Alternatively a 
\program{CASVB} run may be followed by \program{RASSCF} 
(Section~\ref{UG:sec:rasscf}) and \program{GRID\_IT} 
(Section~\ref{UG:sec:gridit}) with the \keyword{VB} specification, in order to 
generate necessary files for viewing with \program{LUSCUS}.

% tut_motra.tex $ this file belongs to the Molcas repository $*/
\section{MOTRA --- An Integral Transformation Program}
\label{TUT:sec:motra}
\index{Program!MOTRA}\index{MOTRA}
\index{Integrals!Integral transformation}
Integrals saved by the \program{SEWARD} module
are stored in the Atomic Orbital (AO) basis.  Some programs have their own
procedures to transform the integrals into the Molecular Orbital (MO) basis.
The \molcas\ \program{MOTRA} module performs this task for
Configuration Interaction (CI), Coupled- and Modified Coupled-Pair (CPF and
MCPF, respectively) and Coupled-Cluster (CC) calculations.

The sample input below contains the \program{motra} input
information for our continuing water calculation.  We firstly specify that the
\program{RASSCF} module interface file will be the source of the
orbitals using the keyword \keyword{JOBIph}.  The keyword
\keyword{FROZen} is used to specify the number of orbitals  in each
symmetry which will not be correlated in
subsequent calculations.  This can also be performed in the corresponding
\program{MRCI}, \program{CPF} or CC programs
but is more efficient to freeze them here.  
Virtual orbitals can be deleted using the \keyword{DELEte} keyword.

\begin{inputlisting}
 &MOTRA
JobIph
Frozen= 1 0 0 0
\end{inputlisting}

\subsection{\program{motra} Output}

The \program{motra} section of the output is short and self
explanatory.  The integral files produced by \program{SEWARD}, \file{ONEINT} 
and \file{ORDINT}, are used as input by the
\program{MOTRA} module which produces the transformed symbolic files
\file{TRAONE} and \file{TRAINT}, respectively. In our case, the files
are called \file{water.TraOne} and \file{water.TraInt}, respectively.
 

The \program{motra} module also requires input orbitals.
If the \keyword{LUMOrb} keyword is specified the orbitals are taken
from the \file{INPORB} file which can be any formated orbital
file such as \file{water.ScfOrb} or \file{water.RasOrb}.  The 
\keyword{JOBIph} keyword causes the \program{MOTRA} module to
read the required orbitals from the \file{JOBIPH} file.

\subsection{MOTRA - Basic and Most Common Keywords}
\begin{keywordlist}
\item[FROZEN] By symmetry: non-correlated orbitals (default: core)
\item[RFPErt] Previous reaction field introduced as a perturbation
\item[LUMORB] Input orbital file as ASCII (INPORB)
\item[JOBIPH] Input orbital file as binary (JOBOLD)
\item[]

%--
\end{keywordlist}


% tut_guga.tex $ this file belongs to the Molcas repository $*/
\section[GUGA --- CI Coupling Coefficients Program]
        {GUGA --- A Configuration Interaction Coupling Coefficients Program}
\label{TUT:sec:guga}
\index{GUGA}\index{Program!GUGA}
Several of the Configuration Interaction (CI) modules in \molcas\ use
the \program{guga} module to compute the CI coupling coefficients. 
We continue our water calculations using the input file shown in
the input below. The \keyword{TITLe} keyword behaves
in a similar fashion as described in previous modules.  
There are several compulsory keywords of the \program{guga} module. The
number of electrons to be correlated is specified using the 
\keyword{ELECtrons} keyword.  We are correlating the valence electrons.
The spin state is specified using the \keyword{SPIN} keyword.  
\index{GUGA!Input}

Sample input requesting the the GUGA module to calculate the coupling 
coefficients for neutral triplet water in C$_{2v}$ symmetry with six electrons 
in the active space:
\begin{inputlisting}
 &GUGA
Title= GUGA for C2v Water
Electrons= 8; Spin= 3
Inactive= 1 0 0 0; Active= 2 2 0 1
CIAll= 1
\end{inputlisting}

The keywords \keyword{CIALl} and \keyword{REFErence} are mutually
exclusive.  We specify \keyword{CIALl} which will calculate the
energy using all possible references functions that can be constructed 
using the input set of occupation numbers of the active orbitals regardless of 
the spin coupling (all configurations used to build the corresponding CASSCF
wave function).  Specific selected references  can be chosen using the 
\keyword{REFErence} keyword. Either the \keyword{ACTIve} or \keyword{INACtive} 
keyword should be used for a meaningful calculation.  The default for both 
keywords is zero for all symmetries.  These keywords function in a similar 
fashion to these in the \program{RASSCF} program module.  The \keyword{INACtive}
keyword specifies the orbitals that are fully occupied in each symmetry
in all the reference functions and the \keyword{ACTIve} keyword
specifies the orbitals that may have varying occupations in all references.  
The selection of \keyword{INACtive} orbitals in the input above 
is forcing the bonding $sp$ hybrid orbital to remain fully occupied in all 
reference states.

\subsection{\program{GUGA} Output}

The \program{GUGA} section of the output lists the possible configurations
in the active space. There are nine possible triplet configurations of
six electrons in five orbitals. Apart from the various types of orbital in each
symmetry the \program{GUGA} section of the output also gives the number of 
states that will coupled with various states. There are no input files for the 
\program{GUGA} module but the calculated coupling coefficients are stored in 
\file{CIGUGA}.


% tut_mrci.tex $ this file belongs to the Molcas repository $*/
\section{MRCI --- A Configuration Interaction Program}
\label{TUT:sec:mrci}
\index{Program!MRCI}\index{MRCI}\index{CI}
\index{ACPF}
Multi Reference Single and Doubles Configuration Interaction (MR-SDCI)
wave functions are produced by the \program{MRCI} program module in
the \molcas\ codes.  
The \keyword{SDCI} keyword requests an
ordinary Multi Reference Single and Doubles Configuration Interaction
calculation.  This is the default and is mutually exclusive with the
\keyword{ACPF} keyword which requests an Average Coupled Pair Function
calculation.  The final keyword, \keyword{ROOT}, specifies the number
of the CI root the calculation should compute.  The second CI root is
the first excited state and since the \program{GUGA} module has computed the
coupling coefficients for a triplet state, the \program{MRCI} module will
converge to the first excited triplet state.

\subsection{\program{mrci} Output}

The \program{mrci} section of the output lists the number of each type
of orbital in each symmetry including pre-frozen orbitals that were
frozen by the \program{guga} module.  There is a list of the
reference configurations with the inactive orbitals included.  An empty
orbital is listed as `{\tt 0}' and a doubly occupied as `{\tt 3}'.  The
spin of a singly occupied orbital by `{\tt 1}' (spin up) or `{\tt 2}'
(spin down).  The total
number of configuration state functions (CSFs) is listed below the reference
configurations.

Sample input requesting the the MRCI module to calculate the first 
excited MRCI energy for neutral triplet water in C$_{2v}$ symmetry with six 
electrons in the active space:
\begin{inputlisting}
 &MRCI
Title= MR-SDCI of 2nd CI root of C2v Water
SDCI; Root= 2
\end{inputlisting}

A listing of the possible CI roots is followed by the CI iteration and
convergence information.  The Davidson and ACPF corrections are included
along with the important CSFs in the CI wave function.  The molecular
orbitals are listed near the end of the output.

There are four input files to the \program{MRCI} module; \file{CIGUGA}
from \program{GUGA}, \file{TRAONE} and \file{TRAINT} from
\program{MOTRA} and \file{ONEINT} from \program{SEWARD}.  The orbitals
are saved in \file{CIORBnn} where $nn$ is the number of the CI root.

% tut_cpf.tex $ this file belongs to the Molcas repository $*/
\section{CPF --- A Coupled-Pair Functional Program}
\label{TUT:sec:cpf}
\index{CPF}\index{Program!CPF}
\index{MCPF}\index{ACPF}
The \program{CPF} program produces  Single and Doubles Configuration
Interaction (SDCI), Coupled-Pair Functional (CPF), Modified Coupled-Pair
Functional (MCPF), and Averaged Coupled-Pair Functional (ACPF) wave
functions (see CPF section of the user's guide) from one
reference configuration. The difference between the \program{MRCI} and
\program{CPF} codes is that the former can handle Configuration
Interaction (CI) and Averaged Coupled-Pair Functional (ACPF) calculations
with more than one reference configuration. For a closed-shell reference
the wave function can be generated with the \program{SCF} program. In 
open-shell cases the \program{RASSCF} has to be used.

The \keyword{TITLe} keyword behaviors in a similar fashion to the
other \molcas\ modules.  The \keyword{CPF} keyword requests an
Coupled-Pair Functional calculation.  
This is the default and is mutually exclusive with keywords
\keyword{MCPF}, \keyword{ACPF}, and \keyword{SDCI} which request different
type of calculations. The input below lists the input files
for the \program{guga} and \program{cpf} programs to obtain the MCPF
energy for the lowest triplet state of B$_2$ symmetry in the water molecule.
The \program{GUGA} module computes the coupling coefficients for a triplet 
state of the appropriate symmetry and the \program{CPF} module will
converge to the first excited triplet state. One orbital of the first
symmetry has been frozen in this case (core orbital) in the \program{MOTRA} 
step.

\subsection{\program{cpf} Output}

The \program{cpf} section of the output lists the number of each type
of orbital in each symmetry including pre-frozen orbitals that were
frozen by the \program{guga} module.  After some information concerning the
total number of internal configurations used and storage data, it appears
the single reference configuration in the \program{mrci} format: an empty
orbital is listed as `{\tt 0}' and a doubly occupied as `{\tt 3}'.  The
spin of a singly occupied orbital by `{\tt 1}' (spin up) or `{\tt 2}'
(spin down). The molecular orbitals are listed near the end of the output.

Sample input requested by the GUGA and CPF modules to calculate the ACPF energy for
the lowest B$_1$ triplet state of the water in C$_{2v}$ symmetry:
\begin{inputlisting}
 &GUGA
Title= H2O molecule. Triplet state.
Electrons= 8; Spin= 3
Inactive= 2 0 1 0; Active= 1 1 0 0
CiAll= 2

 &CPF
Title= MCPF of triplet state of C2v Water
MCPF
\end{inputlisting}

There are four input files to the \program{cpf} module; \file{CIGUGA}
from \program{GUGA}, \file{TRAONE} and \file{TRAINT} from
\program{MOTRA} and \file{ONEINT} from \program{SEWARD}.  The orbitals
are saved in \file{CPFORB}.

% ccsdt.tex $ this file belongs to the Molcas repository $*/
\section{CCSDT --- A Set of Coupled-Cluster Programs}
\label{TUT:sec:ccsdt}
\index{CCSD}\index{CCSD(T)}\index{Program!CCSDT}
\index{Program!CCSORT}\index{Program!CCSD}\index{Program!CCT3}

The \molcas\ program \program{CCSDT} 
computes Coupled-Cluster Singles Doubles, CCSD, and Coupled-Cluster Singles
Doubles and Non-iterative Triples Correction CCSD(T) wave functions
for restricted single reference
both closed- and open-shell systems. 

In addition to the \file{ONEINT} and \file{ORDINT} integral files
(in non-Cholesky calculations),
the \program{CCSDT} code requires the \file{JOBIPH} file containing the
reference wave function (remember that it is not possible to
compute open-shell systems with the \program{SCF} program) and
the transformed two-electron integrals produced by the \program{MOTRA}
module and stored in the \file{TRAINT} file. 


Previously to execute the \program{CCSDT} module, wave functions
and integrals have to be prepared. First, a RASSCF calculation has
to be run in such a way that the resulting wave function has one
single reference. In closed-shell situations this means to include
all the orbitals as inactive and set the number of active electrons to zero.
Keyword \keyword{OUTOrbitals} followed by the specification \keyword{CANOnical}
must be used in
the \program{RASSCF} input to activate the construction of canonical
orbitals and the calculation of the CI-vectors on the basis of
the canonical orbitals.
After that the \program{MOTRA} module has to
be run to transform the two-electron integrals using the molecular
orbitals provided by the \program{RASSCF} module. 
The files \file{JOBIPH} or \file{RASORB} from the
\program{RASSCF} calculation can be used directly by \program{MOTRA}
using the keywords \keyword{JOBIph} or \keyword{LUMOrb} in the \program{MOTRA} input.
Frozen or
deleted orbitals can be introduced in the transformation step
by the proper options in the \program{MOTRA} input. 

\subsection{\program{CCSDT} Outputs}

The section of the \molcas\ output corresponding to the CC program
is self explanatory. The default output simply contains
the wave function specifications from the previous RASSCF calculation,
the orbital specifications, the diagonal Fock matrix elements and orbital 
energies, the technical description of the calculation, the iterations leading to the CCSD energy,
and the five largest amplitudes of each type, which will help to evaluate
the calculation. If triples excitations have been required the description
of the employed method (from the three available) to compute perturbatively
the triple excited contributions to the CC energy, the value of the
correction, and the energy decomposition into spin parts will be available.

\subsection{Example of a CCSD(T) calculation}

Figure~\ref{fig:ccsdt_input} contains the input files required by the
\program{seward}, \program{scf}, 
\program{rasscf}, \program{motra} and \program{ccsdt}
programs to compute the ground state of the HF$^+$ cation.
molecule, which is a doublet of $\Sigma^+$ symmetry. A more detailed 
description of the different options included in the input of the
programs can be found in the CCSDT section of the user's guide.
This example describes how to calculate CCSD(T) energy for HF(+) cation.
This cation can be safely represented by the single determinant as a reference
function, so one can assume that CCSD(T) method will be suitable for its
description.

The calculation can be divided into few steps:
\begin{enumerate}
\item
Run \program{SEWARD} to generate AO integrals.
\item
Calculate the HF molecule at the one electron level using \program{SCF} to
prepare an estimate of MO for the \program{RASSCF} run.
\item
Calculate HF(+) cation by subtracting one electron from the orbital with
the first symmetry. There is  only one electron in one active orbital
so only one configuration is created. Hence, we obtain a simple single
determinant ROHF reference.
\item
Perform MO transformation exploiting \program{MOTRA} using MO coefficients
from the \program{RASSCF} run.
\item
Perform the Coupled Cluster calculation using \program{CCSDT} program. First,
the data produced by the programs \program{RASSCF} and \program{MOTRA} need
to be reorganized, then the CCSD calculation follows, with the chosen spin
adaptation being T2 DDVV. Finally, the noniterative triple excitation contribution
calculation is following, where the CCSD amplitudes are used.
\end{enumerate}

This is an open shell case, so it is suitable to choose CCSD(T) method
as it is defined by Watts {\em et al}. \cite{t3_watts}.
Since CCSD amplitudes produced by previous \program{CCSD} run are partly
spin adapted and denominators are produced from the corresponding diagonal
 Fock matrix elements,
final energy is sometimes referred as SA1 ${\rm CCSD(T)_{\it d}}$ (see
\cite{t3_neo}).

%A suitable shell script to run these calculations can be found at the end of
%section~\ref{UG:sec:cct3} of the user's guide.

\begin{figure}[h]
\caption{Sample input containing the files required by the SEWARD, SCF,
RASSCF, MOTRA, CCSORT, CCSD, and
CCT3 programs to compute the ground state of the HF$^+$ cation.}
\label{fig:ccsdt_input}
\end{figure}
%%%To_extract{/doc/samples/tutorials/CCSDT.HF.input}
\begin{inputlisting}
 &SEWARD &END
Title= HF molecule
Symmetry
X Y
Basis set
F.ANO-S-VDZ
F      0.00000   0.00000   1.73300
End of basis
Basis set
H.ANO-S-VDZ
H      0.00000   0.00000   0.00000
End of basis
End of input
 &SCF
 &RASSCF 
Title= HF(+) cation
OUTOrbitals= Canonical
Symmetry= 1; Spin= 2
nActEl= 1 0 0; Inactive= 2 1 1 0; Ras2= 1 0 0 0
LumOrb; OUTOrbitals= Canonical
 &MOTRA; JobIph; Frozen= 1 0 0 0
 &CCSDT 
Iterations= 50; Shift= 0.2,0.2; Accuracy= 1.0d-7
Denominators= 2; Extrapolation= 5,4
Adaptation= 1; Triples= 3; T3Denominators= 0
\end{inputlisting}
%%%To_extract

\program{RASSCF} calculates the HF ionized state by removing one electron
from the orbital in the first symmetry.
Do not forget to use keyword
\keyword{CANONICAL}.
In the \program{CCSDT} run, the number of iterations is limited to 50.
Denominators will be formed using orbital energies. (This corresponds to the
chosen spin adaptation.) Orbitals will be shifted by 0.2 au,
what will accelerate the convergence. However, final energy will not be
affected by the chosen type of denominators and orbital shifts. Required
accuracy is 1.0d-7 au. for the energy. T2 DDVV class of CCSD amplitudes will
be spin adapted.
To accelerate the convergence,
DIIS procedure is exploited. It will start after 5th iteration and
the last four iterations will be taken into account in each extrapolation step.

In the triples step the CCSD(T) procedure as defined
by Watts {\em et al}. \cite{t3_watts} will be performed.
Corresponding denominators will be produced using diagonal Fock matrix elements.

\subsection{CCSDT - Basic and Most Common Keywords}
\begin{keywordlist}
\item[CCSD] Coupled-cluster singles and doubles method
\item[CCT] CCSD plus a non iterative triples (T) calculation
\item[]

%--
\end{keywordlist} 


\else
\section{Other Multiconfigurational and Multireference Methods}
\molcas\ contains computational modules to perform
\begin{itemize}
\item Coupled Cluster calculations: CCSD, CCSD(T), etc.  (\program{CCSDT} and \program{CHCC})
\item Multireference singles and doubles Configuration Interaction calculations ( \program{mrci} )
\item SDCI - single and double Configuration Interaction calculations (\program{SDCI})
\item CASVB/MCSCF Valence Bond calculations (\program{CASVB})
\end{itemize}

The Users Guide and Tutorials should be consulted for relevant keywords to be used in these modules.
\fi
%% alaska.tex $ this file belongs to the Molcas repository $
\section{ALASKA --- A Program for Integral Derivatives}
\label{TUT:sec:alaska}
\index{ALASKA}\index{Program!ALASKA}

\program{ALASKA} computes the first derivatives of the one- and two-electron 
integrals with respect to the nuclear displacements. The derivatives are contracted
with the one- and two-electron densities to form the molecular gradients, which
will be used by the program \program{SLAPAF}. At present the \program{ALASKA}
module computes SCF/DFT and MCSCF gradients analytically, the rest are computed
numerically. The \program{ALASKA} module is automatically invoked when needed if
the user has not explicitly requested the module to be executed. We postpone the
discussion about \program{ALASKA} to section~\ref{TUT:sec:structure}.


% tut_optimiza.tex $ this file belongs to the Molcas repository $*/
\section{ALASKA and SLAPAF: A Molecular Structure Optimization}
\label{TUT:sec:structure}
\index{Optimization}\index{Geometry}

One of the most powerful functions of {\it ab initio} calculations is geometry 
predictions.  The minimum energy structure of a molecule for a given method and 
basis set is instructive especially when experiment is unable to determine the 
actual geometry. \molcas\ performs a geometry optimization with analytical 
gradients at the SCF or RASSCF level of calculation, and with numerical 
gradients at the CASPT2 level.

In order to perform geometry optimization an input file must contain
a loop, which includes several calls: calculation of integrals (\program{SEWARD}),
calculation of energy (\program{SCF}, \program{RASSCF}, \program{CASPT2}), calculation of gradients
(\program{ALASKA}), and calculation of the new geometry (\program{SLAPAF}). 

This is an example of such input
\begin{sourcelisting}
  &GATEWAY
   coord= file.xyz
   basis= ANO-S-MB
>> EXPORT MOLCAS_MAXITER=25
>> Do While  <<
  &SEWARD
  &SCF  
  &SLAPAF
>> EndDo << 
\end{sourcelisting}

The initial coordinates will be taken from xyz file file.xyz, and the geometry 
will be optimized at the SCF level in this case. After the wave function calculation,
calculation of gradients is required, although code \program{ALASKA} is automatically
called by \molcas. \program{SLAPAF} in this case required the calculation of an
energy minimum (no input). Other options are transition states (\keyword{TS}), minimum energy
paths (\keyword{MEP-search}), etc
The loop will be terminated if the geometry 
converges, or maximum number of iterations (MaxIter) will be reached (the
default value is 50).

There are several EMIL commands 
\ifmanual
(see sect~\ref{UG:sec:emil_commands})
\fi
, which can be 
used to control geometry optimization. For example, it is possible to execute 
some \molcas\ modules only once:
\begin{sourcelisting}
>> IF ( ITER = 1 )
* this part of the input will be executed only during the first iteration
>> ENDIF
\end{sourcelisting}

Program \program{SLAPAF} is tailored to use  analytical or numerical gradients produced
by \program{ALASKA} to relax the geometry of a molecule towards an energy
minimum (default, no input required then) or a transition state. The program is also used for
finding inter state crossings (ISC), conical intersections (CI),
to compute reaction paths, intrinsic reaction coordinate (IRC) paths, etc.
%Examples as how to use the \program{SLAPAF} code is displayed in following section~\ref{TUT:sec:structure}.

\subsection{SLAPAF - Basic and Most Common Keywords}
\begin{keywordlist}
\item[TS] Computing a transition state
\item[FindTS] Computing a transition state with a constraint
\item[MEP-search] Computing a steepest-descent minimum reaction path
\item[ITER] Number of iterations
\item[INTErnal] Definition of the internal coordinates
%\item[IRC ] Intrinsic reaction coordinate analysis of a TS
\item[]

%--
\end{keywordlist} 

\ifmanual
% mckinley.tex $ this file belongs to the Molcas repository $*/
\section{MCKINLEY --- A Program for Integral Second Derivatives}
\label{TUT:sec:mckinley}
\index{MCKINLEY}\index{Program!MCKINLEY}

\program{MCKINLEY} computes the analytic second derivatives of the one- and two-electron
integrals with respect to the nuclear positions at the SCF and CASSCF level of theory. 
The differentiated integrals
can be used by program \program{MCLR} to performs response calculations on
single and multiconfigurational SCF wave functions. One of the basic uses
of \program{MCKINLEY} and \program{MCLR} is to compute analytical hessians
(vibrational frequencies, IR intensities, etc).
Note that \program{MCKINLEY} for a normal frequency calculations will automatically
start the \program{MCLR} module!
For all other methods a numerical procedure is automatically
invoked by \program{MCKINLEY} to compute the vibrational frequencies.

\subsection{MCKINLEY - Basic and Most Common Keywords}
\begin{keywordlist}
\item[PERTurbation] Suboptions Geometry (for geometry optimizations) or Hessian (full Hessian)
\item[]

%--
\end{keywordlist}

% mclr.tex $ this file belongs to the Molcas repository $*/
\section{MCLR --- A Program for Linear Response Calculations}
\label{TUT:sec:mclr} 
\index{MCLR}\index{Program!MCLR}

\program{MCLR} computes response calculations on single and multiconfigurational
SCF wave functions. One of the basic uses of \program{MCKINLEY} and \program{MCLR}
is to compute analytical hessians (vibrational frequencies, IR intensities, etc).
\program{MCLR} can also calculate the Lagrangian multipliers for
a MCSCF state included in a state average optimization and construct the effective
densities required for analytical gradients of such a state.
The use of keyword \keyword{RLXRoot} in the \program{RASSCF} program is required.
In both cases the explicit request of executing the \program{MCLR} module is not
required and will be automatic.
We postpone further
discussion about \program{MCLR} to section~\ref{TUT:sec:structure}.

It follows an example of how to optimize an excited state from a previous
State-Average (SA) CASSCF calculation. 

%%%To_extract{/doc/samples/tutorials/MCLR.acrolein.input}
\begin{inputlisting}
 &GATEWAY
Title= acrolein minimum optimization in excited state 2
Coord=$MOLCAS/Coord/Acrolein.xyz
*$
Basis= sto-3g
Group=NoSym
>>> Do while
 &SEWARD
 &RASSCF
Title= acrolein
Spin= 1; nActEl= 6 0 0; Inactive= 12; Ras2= 5
CiRoot= 3 3 1
Rlxroot= 2
 &SLAPAF
>>> EndDo
\end{inputlisting}
%%%To_extract

%$
The root selected for optimization has been selected here with the keyword
\keyword{Rlxroot} in \program{RASSCF}, but it is also possible to select it
with keyword \keyword{SALA} in \program{MCLR}.

Now if follows an example as how to compute the analytical hessian for the lowest
state of each symmetry in a CASSCF calculation (SCF, DFT, and RASSCF analytical
hessians are also available).

%%%To_extract{/doc/samples/tutorials/MCLR.benzoquinone.input}
\begin{inputlisting}
 &GATEWAY
Title=p-benzoquinone anion. Casscf optimized geometry.
Coord = $MOLCAS/Coord/benzoquinone.xyz
Basis= sto-3g
Group= X Y Z
 &SEWARD
 &RASSCF
TITLE=p-benzoquinone anion. 2B3u state.
SYMMETRY=2; SPIN=2; NACTEL=9 0 0
INACTIVE=8  0  5  0  7  0  4  0
RAS2    =0  3  0  1  0  3  0  1

 &MCKINLEY; Perturbation=Hessian

\end{inputlisting}
%%%To_extract

The \program{MCLR} is automatically called after \program{MCKINLEY}
and it is not needed in the input.

\subsection{MCLR program - Basic and Most Common Keywords}
\begin{keywordlist}
\item[SALA] Root to relax in geometry optimizations
\item[ITER] Number of iterations
\item[]

%--
\end{keywordlist}


\fi
\ifmanual
% genano.tex $ this file belongs to the Molcas repository $
\section{GENANO --- A Program to Generate ANO Basis Sets}
\label{TUT:sec:genano}
\index{Program!GENANO}\index{GENANO}
\index{Basis set!Generation}
\index{Basis set!Atomic Natural Orbitals}

\program{GENANO} is a program for determining the contraction coefficients for
generally contracted basis sets. They are determined by diagonalizing a density 
matrix, using the eigenvectors (natural orbitals) as the contraction 
coefficients, resulting in basis sets of the ANO (Atomic Natural Orbitals) type.
The program can be used to generate any set of atomic or molecular basis 
functions. Only one or more wave functions (represented by formated orbital 
files) are needed to generate the average density matrix. These natural orbital 
files can be produced by any of the wave function generators, as it is described
in section~\ref{UG:sec:genano} of the user's guide. As an illustrative example, 
in the Advanced Examples there is an example of how to 
generate a set of molecular basis set describing Rydberg orbitals for the 
benzene molecule. The reader is referred to this example for more details.

The \program{GENANO} program requires several input files. First, one 
\file{ONEINT} file generated by the \program{SEWARD} module for each input wave 
function. The files must be linked as \file{ONE001}, \file{ONE002}, etc. If the 
wave functions correspond to the same system, the same \file{ONEINT} file must 
be linked with the corresponding names as many times as wave functions are 
going to be treated. Finally, the program needs one file for wave function 
containing the formated set of natural orbitals. The files must be linked as 
\file{NAT001}, \file{NAT002}, etc.

The input file for module \program{GENANO} contains basically three important
keywords. \keyword{CENTER} defines the atom label for which the basis set is to 
be generated. The label must match the label it has in the \program{SEWARD}.
\keyword{SETS} keyword indicates that the next line of input contains the 
number of sets to be used in the averaging procedure and \keyword{WEIGHTS} 
defines the relative weight of each one of the previous sets in the averaging 
procedure. Figure~\ref{fig:genano_input} lists the input file required by the
\program{GENANO} program for making a basis set for the oxygen atom. Three 
natural orbital files are expected, containing the natural orbitals for the 
neutral atom, the cation, and the anion.

\begin{figure}[h]
\caption{Sample input requesting the GENANO module to
average three sets of natural orbitals on the oxygen atom.}
\label{fig:genano_input}
\begin{inputlisting}
 &GENANO
Title= Oxygen atom basis set: O/O+/O-
Center= O
Sets= 3
Weights= 0.50 0.25 0.25
\end{inputlisting}
\end{figure}

As output files \program{GENANO} provides the file \file{ANO},
containing the contraction coefficient matrix organized such that each column 
correspond to one contracted basis function, and the file \file{FIG}, which 
contains a PostScript figure file of the obtained eigenvalues. The output of 
\program{GENANO} is self-explanatory.

\fi
\ifmanual
% ffpt.tex $ this file belongs to the Molcas repository $
\section{FFPT --- A Finite Field Perturbation Program}
\label{TUT:sec:ffpt}
\index{FFPT}\index{Program!FFPT}
\index{Properties!Finite-field PT}
\index{Properties!FFPT}
Many molecular properties of wave functions can be computed using the 
\program{FFPT} program module in \molcas.  It adds the requested operator to 
the integrals computed by the \program{seward} module.  This must be done 
before the \molcas\ module calculating the required wave function is requested 
so the \program{FFPT} module is best run directly after the \program{seward}
module.

The \keyword{TITLe} keyword behaviors in a similar fashion to other
\molcas\ modules.
The sample input below contains the \program{FFPT} input
requesting that the dipole moment operator be added to the integrals
using the \keyword{DIPOle} keyword.
The size and direction is specified using the \keyword{COMP} keyword
which accepts free format input.  We can compute the dipole of the
molecule by numerical determination of the gradient of the energy
curve determined for several values of the dipole operator. From the second
derivative we can obtain the polarizability component.

Sample input requesting the FFPT module to
include a dipole moment operator in the integral file:
\begin{inputlisting}
 &FFPT
Title= Finite Perturbation with a dipole in the x negative of strength 0.1 au
FFPT
Dipole
 Comp 
 X -0.1
\end{inputlisting}

\subsection{\program{ffpt} Output}

The \program{ffpt} section of the output is short and self
explanatory.  The \file{ONEINT} file is updated with the requested
operator.


% vibrot.tex $ this file belongs to the Molcas repository $
\section{VIBROT --- A Program for Vibration-Rotation on Diatomic Molecules}
\label{TUT:sec:vibrot}
\index{Program!VIBROT}\index{VIBROT}
\index{Diatomic molecules}\index{Properties!Spectroscopic}

The program \program{VIBROT} computes vibration-rotation spectra for diatomic
molecules. As input it uses a potential curve computed pointwise by any of
the wave function programs. It does not require other input file from any
of the \molcas\ programs, just its standard input file.

In the Advances Examples the reader will find an overview of the input and
output files required by \program{VIBROT} and the different uses of the
program on the calculation of the electronic states of the C$_2$ molecule.
The reader is referred to section~\ref{UG:sec:vibrot}
of the user's guide for a detailed description of the program.

% tut_rassi.tex $ this file belongs to the Molcas repository $*/
\section{SINGLE\_ANISO --- A Magnetism of Complexes Program}
\label{TUT:sec:single_aniso}
\index{Program!SINGLE\_ANISO}\index{SINGLE\_ANISO}

The program \program{SINGLE\_ANISO} calculates nonperturbatively the temperature- and field- dependent magnetic 
properties (Van Vleck susceptibility tensor and function, molar magnetization vector and function) and the 
pseudospin Hamiltonians for Zeeman interaction (the {\it g} tensor and higher rank tensorial components) and the 
zero-field splitting (the \textit{D} tensor and higher rank tensorial components) for arbitrary mononuclear complexes 
and fragments on the basis of ab initio spin-orbit calculations.
\program{SINGLE\_ANISO} requires as input file the \file{RUNFILE} containing all necessary 
{\it ab initio} information: spin orbit eigenstates, angular momentum matrix elements, the states been mixed 
by the spin-orbit coupling in \program{RASSI}, etc.  Usually, the \program{SINGLE\_ANISO} 
runs after \program{RASSI}.

For a proper spin-orbit calculation the relativistic basis sets should be used for the whole calcualtion. 
For \program{SEWARD}, the atomic mean-field (\keyword{AMFI}), Douglas-Kroll (\keyword{DOUG}) must be employed.
To ensure the computation of angular momentum integrals the \keyword{ANGMOM} should be also used, specifying the origin
of angular momentum integrals as the coordinates of the magnetic center of the molecule, i.e. the coordinates of the atom
where the unpaired electrons mainly reside.  For program \program{RASSI} the necessary keywords are: \keyword{SPIN},
since we need a spin-orbit coupling calculation, and \keyword{MEES}, to ensure the computation of angular momentum 
matrix elements in the basis of spin-free states (SFS).

\index{SINGLE\_ANISO!Input}
In the cases where spin-orbit coupling has a minor effect on the low-lying energy spectrum (most of the 
isotropic cases: Cr$^{3+}$, Gd$^{3+}$, etc.) the pseudospin is usually the same as the ground spin. For these cases
the \program{SINGLE\_ANISO} may run without specifying any keywords in the input file.

\begin{inputlisting}
 &SINGLE_ANISO
\end{inputlisting}

In the cases when spin-orbit coupling play an important role in the low-lying energy spectrum, i.e. in the cases of e.g. octahedral Co$^{2+}$,
 most of the lanthanide complexes, the pseudospin differs strongly from the spin of the ground state. In these cases, 
the dimension of the pseudospin can be found by analysing the spin-orbit energy spectrum obtained at \program{RASSI}. 
The pseudospin is best defined as a group of spin-orbit states close in energy. Once specified, these eigenstates are further used
by the \program{SINGLE\_ANISO} to build proper pseudospin eigenfunctions. As an example of an input for \program{SINGLE\_ANISO} 
requiring the computation of all magnetic properties (which is the default) and the computation of the {\it g} tensor for the ground 
Kramers doublet (i.e. pseudospin of a Kramers doublet is \textit{\~{S}}=1/2).

\begin{inputlisting}
 &SINGLE_ANISO
  MLTP
  1
  2
\end{inputlisting}

\program{SINGLE\_ANISO} has implemented pseudospins: \textit{\~{S}}=1/2, \textit{\~{S}}=1, ..., up to \textit{\~{S}}=7/2. The user can also ask for more pseudospins at the same time:

\begin{inputlisting}
 &SINGLE_ANISO
  MLTP
  3
  2 4 2
\end{inputlisting}
For the above input example, the \program{SINGLE\_ANISO} will compute the {\it g} tensor for the ground Kramers doublet
(spin-orbit states 1 and 2), the {\it g} tensor, ZFS tensor and coefficients of higher rank ITO for the pseudospin  
\textit{\~{S}}=3/2 (spin orbit functions 3-6), and the  {\it g} tensor for the third excited Kramers doublet (spin orbit functions 7 and 8).

\subsection{\program{SINGLE\_ANISO} Output}

\index{SINGLE\_ANISO!Output}

The \program{SINGLE\_ANISO} section of the \molcas\ output is divided in four parts. In the first part, the \textit{g} tensor and higher rank Zeeman tensors are computed. They are followed by \textit{D} tensor and higher rank ZFS tensors. The program also computes the angular moments in the direction of the main magnetic axes.

In the second part, the paramaters of the crystal field acting on the ground atomic multiplet of lanthanides are calculated.

In the third part, the powder magnetic susceptibility is printed, followed by the magnetic susceptibility tensors with and without intermolecular interaction included.

In the fourth part, magnetization vectors (if required) are printed, and then the powder molar magnetization calculated for the \keyword{TMAG} 
temperature.

The keywords \keyword{TINT} and \keyword{HINT} control the temperature and field intervals for computation of 
magnetic susceptibility and molar magnetization respectively. 
Computation of the magnetic properties at the experimental temperature and field points with the estimation of the standard deviation from experiment
is also possible via \keyword{TEXP}, defining the experimental temperature and measured magnetic susceptibility and 
\keyword{HEXP}, defining the experimental field and averaged molar magnetization.

\begin{inputlisting}
 &SINGLE_ANISO
 TITLE
 g tensor and magnetic susceptibility
 TYPE
 4
 MLTP
 2
 3 3
 TINT
 0.0 100 101 0.001
\end{inputlisting}

The above input requires computation of the parameters of two pseudospins \textit{\~{S}}=1: the ground (spin-orbit functions 1-3) 
and first excited (spin-orbit functions 4-6) and the magnetic susceptibility in 101 steps equally distributed in 
the temperature domain 0.0-100.0 K.

\subsection{SINGLE\_ANISO - Basic and Most Common Keywords}
\begin{keywordlist}
\item[MLTP] Specifies the number and dimension of the pseudospins Hamiltonians
\item[TMAG] Sets the temperature for the computation of molar magnetization
\item[MVEC] Number and radial coordinates of directions for which the magnetization vector will be computed
%--
\end{keywordlist}


% tut_rassi.tex $ this file belongs to the Molcas repository $*/
\section{POLY\_ANISO --- Semi - {\it ab initio} Electronic Structure and Magnetism of Polynuclear Complexes Program}
\label{TUT:sec:single_aniso}
\index{Program!POLY\_ANISO}\index{POLY\_ANISO}

The program \program{POLY\_ANISO} calculates nonperturbatively the temperature- and field- dependent magnetic
properties (Van Vleck susceptibility tensor and function, molar magnetization vector and function) and the
pseudospin Hamiltonians for Zeeman interaction (the {\it g} tensor and higher rank tensorial components) and the
zero-field splitting (the \textit{D} tensor and higher rank tensorial components) for arbitrary mononuclear complexes
and fragments on the basis of ab initio spin-orbit calculations.
\program{POLY\_ANISO} requires as input file the \file{RUNFILE} containing all necessary
{\it ab initio} information: spin orbit eigenstates, angular momentum matrix elements, the states been mixed
by the spin-orbit coupling in \program{RASSI}, etc.  Usually, the \program{POLY\_ANISO}
runs after \program{RASSI}.

For a proper spin-orbit calculation the relativistic basis sets should be used for the whole calcualtion.
For \program{SEWARD}, the atomic mean-field (\keyword{AMFI}), Douglas-Kroll (\keyword{DOUG}) must be employed.
To ensure the computation of angular momentum integrals the \keyword{ANGMOM} should be also used, specifying the origin
of angular momentum integrals as the coordinates of the magnetic center of the molecule, i.e. the coordinates of the atom
where the unpaired electrons mainly reside.  For program \program{RASSI} the necessary keywords are: \keyword{SPIN},
since we need a spin-orbit coupling calculation, and \keyword{MEES}, to ensure the computation of angular momentum
matrix elements in the basis of spin-free states (SFS).

\index{POLY\_ANISO!Input}
In the cases where spin-orbit coupling has a minor effect on the low-lying energy spectrum (most of the
isotropic cases: Cr$^{3+}$, Gd$^{3+}$, etc.) the pseudospin is usually the same as the ground spin. For these cases
the \program{POLY\_ANISO} may run without specifying any keywords in the input file.

\begin{inputlisting}
 &POLY_ANISO
\end{inputlisting}

In the cases when spin-orbit coupling play an important role in the low-lying energy spectrum, i.e. in the cases of e.g. octahedral Co$^{2+}$,
 most of the lanthanide complexes, the pseudospin differs strongly from the spin of the ground state. In these cases,
the dimension of the pseudospin can be found by analysing the spin-orbit energy spectrum obtained at \program{RASSI}.
The pseudospin is best defined as a group of spin-orbit states close in energy. Once specified, these eigenstates are further used
by the \program{POLY\_ANISO} to build proper pseudospin eigenfunctions. As an example of an input for \program{POLY\_ANISO}
requiring the computation of all magnetic properties (which is the default) and the computation of the {\it g} tensor for the ground
Kramers doublet (i.e. pseudospin of a Kramers doublet is \textit{\~{S}}=1/2).

\begin{inputlisting}
 &POLY_ANISO
  MLTP
  1
  2
\end{inputlisting}

\program{POLY\_ANISO} has implemented pseudospins: \textit{\~{S}}=1/2, \textit{\~{S}}=1, ..., up to \textit{\~{S}}=7/2. The user can also ask for more pseudospins at the same time:

\begin{inputlisting}
 &POLY_ANISO
  MLTP
  3
  2 4 2
\end{inputlisting}
For the above input example, the \program{POLY\_ANISO} will compute the {\it g} tensor for the ground Kramers doublet
(spin-orbit states 1 and 2), the {\it g} tensor, ZFS tensor and coefficients of higher rank ITO for the pseudospin
\textit{\~{S}}=3/2 (spin orbit functions 3-6), and the  {\it g} tensor for the third excited Kramers doublet (spin orbit functions 7 and 8).

\subsection{\program{POLY\_ANISO} Output}

\index{POLY\_ANISO!Output}

The \program{POLY\_ANISO} section of the \molcas\ output is divided in four parts. In the first part, the \textit{g} tensor and higher rank Zeeman tensors are computed. They are followed by \textit{D} tensor and higher rank ZFS tensors. The program also computes the angular moments in the direction of the main magnetic axes.

In the second part, the paramaters of the crystal field acting on the ground atomic multiplet of lanthanides are calculated.

In the third part, the powder magnetic susceptibility is printed, followed by the magnetic susceptibility tensors with and without intermolecular interaction included.

In the fourth part, magnetization vectors (if required) are printed, and then the powder molar magnetization calculated for the \keyword{TMAG}
temperature.

The keywords \keyword{TINT} and \keyword{HINT} control the temperature and field intervals for computation of
magnetic susceptibility and molar magnetization respectively.
Computation of the magnetic properties at the experimental temperature and field points with the estimation of the standard deviation from experiment
is also possible via \keyword{TEXP}, defining the experimental temperature and measured magnetic susceptibility and
\keyword{HEXP}, defining the experimental field and averaged molar magnetization.

\begin{inputlisting}
 &POLY_ANISO
 TITLE
 g tensor and magnetic susceptibility
 TYPE
 4
 MLTP
 2
 3 3
 TINT
 0.0 100 101 0.001
\end{inputlisting}

The above input requires computation of the parameters of two pseudospins \textit{\~{S}}=1: the ground (spin-orbit functions 1-3)
and first excited (spin-orbit functions 4-6) and the magnetic susceptibility in 101 steps equally distributed in
the temperature domain 0.0-100.0 K.

\subsection{POLY\_ANISO - Basic and Most Common Keywords}
\begin{keywordlist}
\item[MLTP] Specifies the number and dimension of the pseudospins Hamiltonians
\item[TMAG] Sets the temperature for the computation of molar magnetization
\item[MVEC] Number and radial coordinates of directions for which the magnetization vector will be computed
%--
\end{keywordlist}


\fi
% tut_grid_it.tex $ this file belongs to the Molcas repository $*/
\section{GRID\_IT: A Program for Orbital Visualization}
\label{TUT:sec:gridit}
\index{Program!Grid\_It@\program{Grid\_It}}\index{Grid\_It@\program{Grid\_It}}

\program{GRID\_IT} is an interface program for calculations of molecular
orbitals and density in a set of cartesian grid points. Calculated grid
can be visualized by separate program \program{LUSCUS} in 
the form of isosurfaces. 

\program{GRID\_IT} generates the regular grid and calculates amplitudes of 
molecular orbitals in this net. Keywords \keyword{Sparse},\keyword{Dense},
\keyword{Npoints} specify the density of the grid. And keywords \keyword{ORange} (occupation range),
\keyword{ERange} (energy range), \keyword{Select} allow to select some specific orbitals to draw.

As default \program{GRID\_IT} will use grid net with intermediate quality,
and choose orbitals near HOMO-LUMO region. Note, that using keyword
\keyword{All} - to calculate grids for all orbitals or \keyword{Dense} -
to calculate grid with very high quality you can produce a very huge
output file.


\label{TUT:sec:gridit_dependencies}
\program{GRID\_IT} requires the communication file \file{RUNFILE},
processed by \program{GATEWAY} and any formated \file{INPORB} file: \file{SCFORB},
\file{RASORB}, \file{PT2ORB}, generated by program \program{SCF}, \program{RASSCF},
or \program{CASPT2}, respectively. The output file \file{M2MSI} 
contains the graphical information.

Normally you do not need to specify any keywords for \program{GRID\_IT}:
the selection of grid size, as well as the selection of orbitals done automatically.

An input example for \program{GRID\_IT} is:

\begin{inputlisting}
 &GRID_IT 
Dense
* compute orbitals from 20 to 23 form symmetry 1 and orbital 4 from symmetry 2
SELECT
1:20-23,2:4
\end{inputlisting}

\program{GRID\_IT} can be run in a sequence of other computational codes
(if you need to run \program{GRID\_IT} several times, you have to rename 
grid file by using EMIL command, or by using keyword NAME)
\begin{inputlisting}
&GATEWAY
 ...
&SEWARD 
&SCF
&GRID_IT
NAME=scf
&RASSCF
&GRID_IT
NAME=ras
\end{inputlisting}
or, you can run \program{GRID\_IT} separately, when the calculation has finished.

\begin{inputlisting}
&GATEWAY
&GRID_IT
FILEORB=/home/joe/project/water/water.ScfOrb
\end{inputlisting}

\ifmanual

This is quite important to understand that the timing for \program{GRID\_IT}, and
the size of generated grid file depends dramatically on the targeting problem.
To get a printer quality pictures you have to use Dense grid, but in order to see the
shape of orbitals - low quality grids are much more preferable.

The following table illustrates this dependence:

$C_{24}$ molecule, 14 orbitals.

\begin{tabular}{|c|c|c|c|} \hline
Keywords & Time (sec) & filesize & picture quality \\
\hline
Dense, ASCII  & 188  & 473 Mb & best \\
Dense         & 117   & 328 Mb & best \\
Dense, Pack   & 117  & 41  Mb & below average \\
Default (no keywords)  & 3 & 9 Mb & average \\
Pack                   & 3 & 1.4 Mb & average \\
Sparse                 & 1.3 & 3 Mb & poor \\
Sparse, Pack           & 1.3 & 620 Kb & poor \\
\hline
\end{tabular}

\fi
\subsection{GRID\_IT - Basic and Most Common Keywords}
\begin{keywordlist}
\item[ASCII] Generate the \file{grid} file in ASCII (e.g. to transfer to another computer), 
can be only used in combitation with \keyword{NoLUSCUS}
\item[ALL]   Generate all orbitals
\item[SELECT] Select orbitals to compute
\item[]

%--
\end{keywordlist}


\ifmanual
% tut_molden.tex $ this file belongs to the Molcas repository $*/
\section{Writing MOLDEN input}
\label{TUT:sec:Molden}
\index{Tool!MOLDEN} \index{MOLDEN}

By default the \program{GUESSORB}, \program{SCF}, \program{MBPT2}, \program{RASSCF},
\program{SLAPAF}, \program{LOCALISATION}, and \program{MCLR} modules
generate input in Molden format. The \program{SCF}, \program{MBPT2}, \program{RASSCF},
and \program{LOCALISATION} modules generate input for molecular orbital
analysis, \program{SLAPAF} for geometry optimization analysis, minimum energy paths,
Saddle optimization paths and IRC TS analysis,
and the \program{MCLR} module generates input for
analysis of harmonic frequencies. Molden files can be visualized by \program{GV}
or by \program{Molden} (http://www.caos.kun.nl/\~{}schaft/molden/molden.html).

The generic name of the input file and the actual
name are different for the nodes as a reflection on the data generated
by each module. Hence, the actual names (generic name) for the Molden files in each module are
\begin{itemize}
\item GUESSORB module:
\$Project.guessorb.molden (MD\_GSS)
\item SCF module:
\$Project.scf.molden (MD\_SCF)
\item MBPT2 module:
\$Project.mp2.molden (MD\_MP2)
\item RASSCF module:
\$Project.rasscf.molden (MD\_CAS) for the state-averaged natural orbitals, and
\$Project.rasscf.x.molden (MD\_CAS.x) for the state-specific natural spin orbitals,
where x is the index of a CI root.
\item SLAPAF module:
\$Project.geo.molden (MD\_GEO) for geometry optimizations,
\$Project.mep.molden (MD\_MEP) for minimum energy paths,
\$Project.irc.molden (MD\_IRC) for IRC analysis of a TS, and
\$Project.saddle.molden (MD\_SADDLE) for Saddle method TS optimizations.
\item LOCALISATION module:
\$Project.local.molden (MD\_LOC)
\item MCLR module:
\$Project.freq.molden (MD\_FREQ)
\end{itemize}

\fi
\ifmanual
%! tut_errors.tex $ this file belongs to the Molcas repository $*/
\section{Most frequent error messages found in MOLCAS}

\index{Error messages}
\index{Error}

Due to the large number of systems where the \molcas\ package is
executed and the large number of options included in each of
the programs it is not possible to compile here all the possible
sources of errors and error messages occurring in the calculations.
The \molcas\ codes contain specific error message data basis where
the source of the error and the possible solution is suggested.
Unfortunately it is almost impossible to cover all the possibilities.
Here the user will find a compendium of the more usual errors
showing up in \molcas\ and the corresponding error messages.

Many of the error messages the user is going to obtain are specific
for the operative system or architecture being used. 
The most serious ones are in most of cases 
related with compiler problems, operative system incompatibilities,
etc. Therefore the meaning of this errors must be checked in the proper 
manuals or with the computer experts, and if they are characteristic
only of \molcas, with \molcas\ authors. The most common, however,
are simple mistakes related to lack of execution or reading
permission of the shell scripts, \molcas\ executable modules, etc.

In the following the most usual errors found in \molcas\ are listed.

\begin{itemize}


\index{Error!molcas undefined}
\item The shell is unable to find the command \command{molcas}.
      The message in this case is, for instance:

\begin{sourcelisting}
  molcas:  not found
\end{sourcelisting}

 The solution is to add into the PATH the location of molcas driver script.

\index{Error!MOLCAS undefined}

\item If the \molcas\ environment is not properly installed the
 first message showing up in the default error file is:

\begin{sourcelisting}
***
*** Error: Could not find molcas driver shell
*** Currently MOLCAS=
\end{sourcelisting}
 
 Typing a command \command{molcas}, you can check which molcas
 installation will be used. Check the value of the variable \variable{MOLCAS},
 and define it in order to point to the proper location of molcas installation.
 
\index{Error!ENV undefined}

\item Environment is not defined

 An attempt to run an executable without molcas driver scripts gives
 an error:
\begin{sourcelisting}
  Usage: molcas module_name input
\end{sourcelisting}



 \item A call for a program can find problems like the three following ones:

\begin{sourcelisting}
Program NNNN is not defined 
\end{sourcelisting}
 
An error means that requested module is missing or the package is not installed.

\index{Error!Input file not found}

\item When the input file required for a \molcas\ program is not
      available, the program will not start at all and no output
      will be printed, except in the default error file where the
      following error message will appear:

\begin{sourcelisting}
 Input file specified for run subcommand not found : seward
\end{sourcelisting}

\index{Error!RUNFILE}

\item All the codes communicate via file \file{RUNFILE}, if for a some reason
the file is missing or corrupted, you will get an error 

\begin{sourcelisting}
***    Record not found in runfile
\end{sourcelisting}

 The simple solution - restart seward to generate proper \file{RUNFILE}

\index{Error!ONEINT}
\index{Error!ORDINT}

\item All the codes need integral files generated by \program{SEWARD} in
      files \file{ONEINT} and \file{ORDINT}.
      Even the direct codes need the one-electron integrals stored
      in \file{ONEINT}. The most common problem is then that a program
      fails to read one of this files because \program{SEWARD} has not
      been executed or because the files are read in the wrong address.
      Some of the error messages found in those cases are listed here.
    
      In the \program{SCF} module, the first message will appear when
      the one-electron integral file is missing and the second when 
      the two-electron integral file is missing:

\begin{sourcelisting}
Two-electron integral file was not found!
 Try keyword DIRECT in SEWARD.
\end{sourcelisting}

\index{Error!Insufficient memory}
\item  \molcas\ use dynamical allocation of memory for temporary arrays.
 An error message 'Insufficient memory' means that requested value
 is too small - you have to specify MOLCAS\_MEM variable and restart your
 calculation. 
 
\index{Error!memory allocation}
 \item if user ask to allocate (via MOLCAS\_MEM) an amount of memory, 
 which is large than possible on this computer, the following error message
 will be printed.
 
\begin{sourcelisting}
MA error: MA_init: could not allocate 2097152152 bytes
The initialization of the memory manager failed ( iRc=  1 ).
\end{sourcelisting}
 

\index{Error!input error}

\item An improper input (e.g. the code expects to read more numbers, than 
user specified in input file) will terminate the code with errorcode 112.

\index{Error!Disk address problems}
\index{Error!I/O problems}

\item Input/Output (I/O) problems are common, normally due to insufficient
  disk space to store the two-electron integral files or some of the
  intermediate files used by the programs. The error message would depend
  on the operative system used. An example for the \program{SCF} is
  shown below:

\begin{sourcelisting}
 *******************************************************************************
 *******************************************************************************
 ***                                                                         ***
 ***                                                                         ***
 ***    Location: AixRd                                                      ***
 ***    File: ORDINT                                                         ***
 ***                                                                         ***
 ***                                                                         ***
 ***    Premature abort while reading buffer from disk:                      ***
 ***    Condition: rc != LenBuf                                              ***
 ***    Actual   :                0!=          262144                        ***
 ***                                                                         ***
 ***                                                                         ***
 *******************************************************************************
 *******************************************************************************
\end{sourcelisting}

 The error indicates that the file is corrupted, or there is a bug in the 
 code. 


\item Sometimes you might experience the following problem with GEO/HYPER
      run:

\begin{sourcelisting}
  Quaternion tested
  mat. size =     4x    1
   -0.500000000
    0.500000000
   -0.500000000
    0.000000000
 ###############################################################################
 ###############################################################################
 ###                                                                         ###
 ###                                                                         ###
 ###    Location: CheckQuater                                                ###
 ###                                                                         ###
 ###                                                                         ###
 ###                                                                         ###
 ###    Quaternion does not represent a rotation                             ###
 ###                                                                         ###
 ###                                                                         ###
 ###############################################################################
 ###############################################################################

\end{sourcelisting}

 The error indicates that you need to rearrange the Cartesian coordinates
 (atoms) of one or another fragment.

\end{itemize}
 
\clearpage
%\section{\molcasversion\ Flowchart}
%\label{TUT:sec:flow_all}
%\begin{figure}[hbt]
%\leavevmode
%\flowchart{all}
%\caption{Program module dependencies flowchart for MOLCAS. Shadow boxes represent optative modules to be installed independently.}
%\label{fig:flow_all}
%\end{figure}

%\clearpage
%%% Local Variables: 
%%% mode: latex
%%% TeX-master: t
%%% End: 

\fi
% expbas.tex $ this file belongs to the Molcas repository $*/

\section{Tools for selection of the active space} 
\label{TUT:sec:tools}

Selecting an active space is sometimes easy. For a small molecule,
an active space for the ground and the lowest valence excited states
is usually the valence orbitals, i.e. orbitals composed of atomic
orbitals belonging to the usual 'valence shells' (there are
some exceptions to this rule). Problems arise for medium or large
molecules, for higher excited states, and for molecules including
transition, lanthanide or actinide elements. A good wish list
of orbitals will give a CASSCF/CASPT2 calculation that demand
unrealistically large computer resources and time.
Compromises must be made. Any smaller selection of active orbitals
can in general affect your results, and the selection should be
based on the specific calculations: see~\ref{TUT:sec:hints} for
advise.

The following three tools may be help in the process:

\begin{description}
\item[localisation]  is a program that can take a (subrange of) orbitals
from an orbital file, and produce a new orbital file where these orbitals
have been transformed to become localized, while spanning the same space
as the original ones.
\item[expbas]  can take an orbital file using a smaller basis set, and
'expand' it into a new orbital file using a larger basis.
\item[LUSCUS]  (is of course also described elsewhere) is the orbital viewer.
\end{description}

It is of course best to have a good perception of the electronic structure
of the molecule, including all states of interest for the calculation.
If it is a larger system, where lots of ligands can be assumed not to
partake in non-dynamic correlation, it is a good idea to run some simple
exploratory calculations with a much smaller model system.
Check the literature for calculations on similar systems or model systems.

First of all, you need to know how many orbitals (in each symmetry) that
should be active. Their precise identity is also good to know, in order
to have a good set of starting orbitals, but we come to that later.
{\bf Necessary} active orbitals are: Any shells that may be open in any of the 
states or structures studied. Breaking a bond generally produces a
correlated bond orbital and a correlating antibonding orbital, that must
both be active (Since it is the {\bf number} of orbitals we are dealing
with as yet, you may as well think of the two radical orbitals that are 
produced by completely breaking the bond). 
You probably want to include one orbital for each aromatic carbon.
{\bf Valuable correlated} active orbitals are: Oxygen lone pair, CC
$\pi$ bonds.  {\bf Valuable correlating} active orbitals are: the 
antibonding $\pi^{\ast}$ CC orbitals, and one additional set of 
correlating $d$ orbitals for most transition elements (sometimes 
called the 'double d-shell effect').

The valuable correlated orbitals can be used as Ras-1 orbitals, and
correlating ones can be used as Ras-3 orbitals, if the active space
becomes too large for a casscf calculation.

Assuming we can decide on the number of active orbitals, the next task
is to prepare starting orbitals that enables CASSCF to converge, by
energy optimization, to the actual starting orbitals for your calculation.
Use a very small basis set to begin with: This will usually be one of the
minimal bases, e.g. ANO-S-MB. This is not just to save time: the small 
basis and the large energy spacings make it much easier to get well-defined 
correlating orbitals.

Performing the actual casscf (or rasscf) calculation may give you the
active space you want: Viewing the orbitals by \program{LUSCUS} may confirm this, but
very often the orbitals are too mixed up (compared to ones mental
picture of what constitutes the best orbitals).
Using  localisation program solves this problem. In order to localise 
without mixing up orbitals from different subspaces may require to
produce the new orbital file through several runs of the program;
however, for the present perpose, it may be best not to have so
very strict restrictions, for example: Allow mixing among a few
high inactive and the most occupied orbitals;  and also among the
weakly occupied and some virtual orbitals.

Running the localisation program, and viewing the localised orbitals,
is a great help since directly in \program{LUSCUS} one can redefine orbitals as
being inactive, or ras3 , or whatever, to produce a new orbital file.
The resulting annotated localised orbitals can be used in a new run.

Once a plausible active space has been found, use the expbas tool to
obtain starting orbitals using, e.g. ANO-VDZP basis, or whatever is 
to be used in the bulk of the production run.

It is also a good idea to, at this point, 'waste' a few resources on
a single-point calculation for a few more states than you are really
interested in, and maybe look at properties, etc. There may be
experimental spectra to compare with.

And please have a look at the following 
section, 'Some practical HINTS'~\ref{TUT:sec:hints}.


% hints.tex $ this file belongs to the Molcas repository $*/

\section{Some practical HINTS} 
\label{TUT:sec:hints}

This section contains a collection of practical hints 
and advices how to 
use \molcas\ in solving quantum chemistry problems.


%---------------------------------------------------------------------- SLIDE -
\subsection{GATEWAY/SEWARD program:} 
\begin{itemize}
\item Try the Cholesky approximation (or RI)! 
It saves disk space and possibly time. 
\item Think about basis set. ANO-like basis sets have many advantages,
but they are "marginal". 
\item Try to avoid inline basis sets, use the library.
\item Remember that the quality of basis set should match quality of
computational method. 
\item Use ANO-RCC even for atoms in the 2nd row.
\item Be extremely careful when computing anions.
Remember that special situations requires special basis sets.
\item Use minimal or small basis set for understanding chemical problem.
You always can use expbas later..
\end{itemize}


%---------------------------------------------------------------------- SLIDE -
\subsection{SCF program}
\begin{itemize}
\item HF orbitals are in many cases good starting orbitals, 
but quite often GuessOrb can be used instead.
\item Very large basis set together with HF can lead to linear dependences.
\item Remember! Hartree-Fock method allows multiple solutions (even for trivial
systems)
\item Be reasonable selecting convergence thresholds
\item UHF convergence is much poor comparing to RHF. 
\item Sometime you have to use Scramble keyword to break the symmetry.
\end{itemize}

\begin{itemize}
\item DFT convergence can't be better that HF convergence. Think about starting
orbitals for DFT.
\item Remember that DFT is a powerful method but 
it is still single configurational method. Don't use it beyond 
it's limits.
\item Choose your favorite functional, and stay with your decision
\item Note that MOLCAS is not the best DFT code available
\end{itemize}

%------------------------------------------------------------------------------

%---------------------------------------------------------------------- SLIDE -
\subsection{RASSCF program:}
\begin{itemize}
\item  MCSCF are multi-solution methods that heavily depend on 
 the starting orbitals and 
level of calculation (roots). 
\item On convergence ALWAYS (ALWAYS, ALWAYS, etc) check the orbitals 
(LUSCUS, molden, CMOcorr, etc). MCSCF methods will lead to different solutions 
for active spaces of 
different nature. Use your chemical intuition and 
let the calculation guide you. 
\item Analyze carefully the CI coefficients and natural occupation 
numbers together with 
the orbitals (average orbitals are fine in general for that) 
in order to understand the 
nature of the states .
\end{itemize}

%------------------------------------------------------------------------------

%---------------------------------------------------------------------- SLIDE -
\begin{itemize}
\item You get average orbitals, and orbitals for individual roots, which 
you can visualized by \program{LUSCUS} or molden
etc, contain the natural orbitals of the different roots. 
\item Try increasing the number of SA-CASSCF roots to locate more excited states. They can 
be low-lying solutions at the CASPT2 level. In high symmetry cases you may also need 
to consider roots that have high energy at the initial steps and can become lower roots in 
the converged calculation. 
\item It is NOT advisable to play games with weights for the different roots. Roots with equal 
weights make your calculation more clear and reproducible. 
\end{itemize}

%------------------------------------------------------------------------------

%---------------------------------------------------------------------- SLIDE -
\begin{itemize}
\item MOLCAS can handle only $D_2h$ subgroups. Molecules with other 
symmetry ($C_{3v}$, $D_{4d}$, $T_d$, $O_h$) have a problem. 
Especially if you use approximations, like CD.
\item Work in a symmetry point group that allows degenerate states to belong to the 
same irreducible representation (e.g. $C_2$ for linear molecules). Try $C_1$ too. 
\item Working in a too high symmetry might prevent you from obtaining less 
symmetric lowest-lying localized solutions (e.g. $Ni^{2+}$) 
\item Start with clean symmetric orbitals (GUESSORB). Sometime (for example
 for a radical), an orbital
 for positively charged system can be more symmetric.
\item use if needed, CLEANUP and SUPSYM, or for linear molecules: the 
LINEAR keyword. 
\end{itemize} 


%------------------------------------------------------------------------------

%---------------------------------------------------------------------- SLIDE -
\begin{itemize}
\item use it! RASSCF is a simple way to increase an active space.
\item Balance RAS1/3 and RAS2 subspaces. Try to change orbitals between 
these subspaces.
\item Removing RAS2 space completely is not a good idea.
\item Note that RAS calculations have a slower convergence, and demand more
resources.
\end{itemize}

%------------------------------------------------------------------------------

%---------------------------------------------------------------------- SLIDE -
\begin{itemize}
\item Increase LEVShift parameter in cases of slow or difficult convergence. 
\item Sometimes RASSCF is very sensitive when is close to convergence.
Try restarting the calculation from the previous JOBIPH file
\item Try to restart from orbitals (or JOBIPH) instead of starting from scratch.
\end{itemize}

%------------------------------------------------------------------------------
%---------------------------------------------------------------------- SLIDE -
\subsection{Selection of active spaces: }
\begin{itemize}
\item Always compare calculations with the same active space size (and nature if possible). 
\item Ask yourself first which is your goal. The selection of the active space depends on that. 
\item If you made a selection once, try to reuse orbitals! Especially for a set of
calculations with different geometries
\item In ground state calculations many orbitals can have an occupation close to 2 and 0 and 
might rotate with others in the inactive (secondary) space. It might be wise to skip them 
\item For low-lying excited states and few roots you might leave inactive quite a number of 
orbitals. Check with RASSCF for instance.
\end{itemize}


%------------------------------------------------------------------------------

%---------------------------------------------------------------------- SLIDE -
\begin{itemize}
\item SCF orbital energies sometimes help to choose the orbitals by using the energy order 
criterion, but you must learn to see the problems (like lone pair orbitals having too low 
energies at the SCF level). 
\item You typically will need correlating orbitals, that is, if you have a $\pi$ orbital you need a $\pi^*$, 
the same for $\sigma$,$\sigma^*$, but not for lone pairs. 
\item CASSCF/RASSCF geometry optimizations are the worst case. If you miss orbitals you 
might end up in a totally wrong geometry (e.g. breaking a bond usually requires the 
bonding and antibonding orbitals in the space). 
\item 
Organic (1st row atoms) molecules usually require open shell orbitals, 
$\pi$, $\pi^*$, and lone 
pairs. If 2nd row atoms are added ($S$, $P$, $Si$, etc) $s$ orbitals enter in action (s bonds are 
longer). $CH$ bonds can often be left be inactive. 


%---------------------------------------------------------------------- SLIDE -
\item Rydberg states require additional diffuse basis sets and specific orbitals in the active 
space. Use basis sets centered in the charge center of the positive ion 
(consult the manual). 
\item Transition metal chemistry (1st row) sometimes requires a double $d$ shell description 
in the active space. 
\item Lanthanides have a quite inert $4f$ shell that must be active together with 5d, 6s (6p). 
Actinides $5f$, $6d$, $7s$. 
%
\end{itemize}
%------------------------------------------------------------------------------

\begin{itemize}
\item {\bf use expbas!} start from minimal basis set, decide the active
space, and expand the basis to "normal". With small basis set you can 
clearly identify orbitals. 

\item {\bf use localization!} Especially for virtual orbitals.

\item {\bf expand active space by adding RAS1/3} - give the system a freedom, and see how it 
reacts.
\end{itemize}

%------------------------------------------------------------------------------

%---------------------------------------------------------------------- SLIDE -
\subsection{CASPT2 program:}
\begin{itemize}
\item The new IPEA = 0.25 zeroth Hamiltonian is the default. 
It particularly improves open shell cases. But there are some cases where
IPEA=0 gives better correlation with experiment.
\item Energy differences between different states or situations are only reliable between 
calculations with the same active space size and similar reference weights in CASPT2. 
\item An intruder state (low reference weight in the CASPT2 state) might be informing you 
that your active space lacks an important orbital. Check the list of large perturbative 
contributions (small denominators combined with large RHS values; check the output) 
and also the occupation number of the CASPT2 orbitals. 
\item For weakly interacting intruder states cases try the IMAGINARY level shift parameter. 
%0.05 or 0.1 au is typically enough. It is wise to compute a series: 0.0, 0.05, 0.10, 0.15
%and check that the result is converged. Then, take the lowest value that solves your 
%problem. Beware of using too large level shifts (not larger than IMAGINARY 0.20). 
Don't use the level shift to reach agreement with experiment! 
\end{itemize}

%------------------------------------------------------------------------------

%---------------------------------------------------------------------- SLIDE -
\begin{itemize}
\item For heavy valence-Rydberg mixing cases or for closely degenerated CASPT2 states, 
use MS-CASPT2. 
\item If the MS-CASPT2 description differs a lot from the CASPT2 one, 
try to check the 
calculation by increasing the active space (introducing angular correlation if possible) 
until the result is converged. The "true" solution is typically between both cases 
(CASPT2 and MS-CASPT2). If you are suspicious about the MS-CASPT2 result, 
better keep the CASPT2 one. It has worked out generally well so far. 
\end{itemize}

%------------------------------------------------------------------------------

%---------------------------------------------------------------------- SLIDE -
\subsection{RASSI program:}
\begin{itemize}
\item Remember that the program shows first the interaction among the input states and later this description might change. 
%(because the states order and nature change) in a 
%second part of the output. In general, 
ALWAYS check the changing order of states. 
\item For spin-orbit coupling calculations don't forget to include the CASPT2 energies as input 
(EJOB or HDIAG keywords) because the results depend on the energy gap. In other cases 
having the CASPT2 energies as input will help you to get the right oscillator strength and 
Einstein coefficient in the final table. 
\item If you have degenerate states be sure that the CASPT2 energies are degenerate. If they 
are not (which is common) average the energies for the degenerate set (the two 
components of E symmetry for example). 
\item Remember that the spin-orbit coupled results (e.g. TDM) depend on the number of interacting singlet and triplet states included in RASSI. 
\end{itemize}

%------------------------------------------------------------------------------


%---------------------------------------------------------------------- SLIDE -
\subsection{Geometry optimization}
\begin{itemize}
\item Not all methods have analytical derivatives.
\item Default thresholds in slapaf are typically too tight. Do not waste computer time!
\item Use constrained optimization
\item For minima on flat hypersurfaces, such in loosely bound fragments, or in slow convergence 
cases you might have to decrease the CUTOFF threshold in ALASKA
\item Be careful with the bond angle definition if you are close to a linear bond. 
You may have to switch to the LAngle definition 
\item Don't forget that CASSCF does not include dynamical correlation. In some cases you better 
change to DFT or numerical CASPT2 optimizations or, if this is not feasible, may be 
preferable to run RASSCF optimizations 
\end{itemize}


%------------------------------------------------------------------------------

%---------------------------------------------------------------------- SLIDE -
\begin{itemize}
\item Poor active spaces may lead you to symmetry broken wrong solutions (e.g. a $C_s$ minimum 
for water below the true $C_{2v}$ one) 
\item Poor geometry convergence might be reduced or at least controlled by reducing the initial 
trust radius with the MAXSTEP keyword or/and by doing the optimization in Cartesian 
coordinates (CARTESIAN) 
\item In order to obtain localized solutions it might be a good idea to feed the program with a 
slightly distorted geometry that helps the method to reach the non symmetric solutions. 
Other possibilities are to use an electric field, to add a charge far from the system or use a  solvent cavity. In all cases you break symmetry and allow less symmetric situations. 
\end{itemize}

%------------------------------------------------------------------------------

%---------------------------------------------------------------------- SLIDE -
\begin{itemize}
\item Linearly interpolated internal coordinates geometries may be a good starting point to locate 
a transition state. Use also the useful FindTS command. Sometimes can be wise to compute 
a MEP from the TS to prove that it is relevant for the studied reaction path. Try also the new 
Saddle approach! 
\item When locating a CASSCF surface crossing (MECP) ALWAYS compute CASPT2 energies 
at that point. The gap between the states can be large at that level. In severe cases you might have to make a scan with CASPT2 to find a better region for the crossing. 
\item Remember that (so far) MOLCAS does not search for true conical intersections (CIs) but 
minimum energy crossing points (MECP) because it lacks NACMEs. Note however that 
typically computed minimum energy CIs (MECIs) may not be photochemically relevant 
if they are not easily accessible. Barriers have to be computed. Use MEPs!! 
\end{itemize}

%------------------------------------------------------------------------------

%---------------------------------------------------------------------- SLIDE -
\begin{itemize}
\item Numerical hessians and optimizations may lead you to bad solutions when different 
electronic states are too close. As you move your calculation from the equilibrium geometry 
some of the points may belong to other state and corrupt your result. This might be the case 
for numerical CASPT2 crossing search. Use then MS-CASPT2 search. 
\item Remember that SA-RASSCF analytical gradients and SA-CASSCF analytical hessians are 
not implemented. 
\item Be careful with the change of roots and nature along a geometry optimization or MEP. 
For example, you start with the state in root 3 (at the CASSCF level) and reach a region 
of crossing root 3 and root 2. You may need to change to root 2 for your state. 
Not an easy solution (so far). 
\end{itemize}

%------------------------------------------------------------------------------

%---------------------------------------------------------------------- SLIDE -
\subsection{Solvent effects}
\begin{itemize}
\item Some effects of the solvent are very specific, such as hydrogen bonds, and require to 
include explicit solvent molecules. Try adding a first solvent shell (optimized with 
molecular mechanics for instance) and then a cavity, for instance with PCM. 
\item Too small cavity sizes can lead you to unphysical solutions, 
even if they seem to match experiment. 
\item Remember using NonEquilibrium (final state) and RFRoot (SA-CASSCF) 
when required 
\item QM/MM is a much powerful strategy, but it requires experience and knowledge 
of the field 
\end{itemize}



