% hints.tex $ this file belongs to the Molcas repository $*/

\section{Some practical HINTS} 
\label{TUT:sec:hints}

This section contains a collection of practical hints 
and advices how to 
use \molcas\ in solving quantum chemistry problems.


%---------------------------------------------------------------------- SLIDE -
\subsection{GATEWAY/SEWARD program:} 
\begin{itemize}
\item Try the Cholesky approximation (or RI)! 
It saves disk space and possibly time. 
\item Think about basis set. ANO-like basis sets have many advantages,
but they are "marginal". 
\item Try to avoid inline basis sets, use the library.
\item Remember that the quality of basis set should match quality of
computational method. 
\item Use ANO-RCC even for atoms in the 2nd row.
\item Be extremely careful when computing anions.
Remember that special situations requires special basis sets.
\item Use minimal or small basis set for understanding chemical problem.
You always can use expbas later..
\end{itemize}


%---------------------------------------------------------------------- SLIDE -
\subsection{SCF program}
\begin{itemize}
\item HF orbitals are in many cases good starting orbitals, 
but quite often GuessOrb can be used instead.
\item Very large basis set together with HF can lead to linear dependences.
\item Remember! Hartree-Fock method allows multiple solutions (even for trivial
systems)
\item Be reasonable selecting convergence thresholds
\item UHF convergence is much poor comparing to RHF. 
\item Sometime you have to use Scramble keyword to break the symmetry.
\end{itemize}

\begin{itemize}
\item DFT convergence can't be better that HF convergence. Think about starting
orbitals for DFT.
\item Remember that DFT is a powerful method but 
it is still single configurational method. Don't use it beyond 
it's limits.
\item Choose your favorite functional, and stay with your decision
\item Note that MOLCAS is not the best DFT code available
\end{itemize}

%------------------------------------------------------------------------------

%---------------------------------------------------------------------- SLIDE -
\subsection{RASSCF program:}
\begin{itemize}
\item  MCSCF are multi-solution methods that heavily depend on 
 the starting orbitals and 
level of calculation (roots). 
\item On convergence ALWAYS (ALWAYS, ALWAYS, etc) check the orbitals 
(LUSCUS, molden, CMOcorr, etc). MCSCF methods will lead to different solutions 
for active spaces of 
different nature. Use your chemical intuition and 
let the calculation guide you. 
\item Analyze carefully the CI coefficients and natural occupation 
numbers together with 
the orbitals (average orbitals are fine in general for that) 
in order to understand the 
nature of the states .
\end{itemize}

%------------------------------------------------------------------------------

%---------------------------------------------------------------------- SLIDE -
\begin{itemize}
\item You get average orbitals, and orbitals for individual roots, which 
you can visualized by \program{LUSCUS} or molden
etc, contain the natural orbitals of the different roots. 
\item Try increasing the number of SA-CASSCF roots to locate more excited states. They can 
be low-lying solutions at the CASPT2 level. In high symmetry cases you may also need 
to consider roots that have high energy at the initial steps and can become lower roots in 
the converged calculation. 
\item It is NOT advisable to play games with weights for the different roots. Roots with equal 
weights make your calculation more clear and reproducible. 
\end{itemize}

%------------------------------------------------------------------------------

%---------------------------------------------------------------------- SLIDE -
\begin{itemize}
\item MOLCAS can handle only $D_2h$ subgroups. Molecules with other 
symmetry ($C_{3v}$, $D_{4d}$, $T_d$, $O_h$) have a problem. 
Especially if you use approximations, like CD.
\item Work in a symmetry point group that allows degenerate states to belong to the 
same irreducible representation (e.g. $C_2$ for linear molecules). Try $C_1$ too. 
\item Working in a too high symmetry might prevent you from obtaining less 
symmetric lowest-lying localized solutions (e.g. $Ni^{2+}$) 
\item Start with clean symmetric orbitals (GUESSORB). Sometime (for example
 for a radical), an orbital
 for positively charged system can be more symmetric.
\item use if needed, CLEANUP and SUPSYM, or for linear molecules: the 
LINEAR keyword. 
\end{itemize} 


%------------------------------------------------------------------------------

%---------------------------------------------------------------------- SLIDE -
\begin{itemize}
\item use it! RASSCF is a simple way to increase an active space.
\item Balance RAS1/3 and RAS2 subspaces. Try to change orbitals between 
these subspaces.
\item Removing RAS2 space completely is not a good idea.
\item Note that RAS calculations have a slower convergence, and demand more
resources.
\end{itemize}

%------------------------------------------------------------------------------

%---------------------------------------------------------------------- SLIDE -
\begin{itemize}
\item Increase LEVShift parameter in cases of slow or difficult convergence. 
\item Sometimes RASSCF is very sensitive when is close to convergence.
Try restarting the calculation from the previous JOBIPH file
\item Try to restart from orbitals (or JOBIPH) instead of starting from scratch.
\end{itemize}

%------------------------------------------------------------------------------
%---------------------------------------------------------------------- SLIDE -
\subsection{Selection of active spaces: }
\begin{itemize}
\item Always compare calculations with the same active space size (and nature if possible). 
\item Ask yourself first which is your goal. The selection of the active space depends on that. 
\item If you made a selection once, try to reuse orbitals! Especially for a set of
calculations with different geometries
\item In ground state calculations many orbitals can have an occupation close to 2 and 0 and 
might rotate with others in the inactive (secondary) space. It might be wise to skip them 
\item For low-lying excited states and few roots you might leave inactive quite a number of 
orbitals. Check with RASSCF for instance.
\end{itemize}


%------------------------------------------------------------------------------

%---------------------------------------------------------------------- SLIDE -
\begin{itemize}
\item SCF orbital energies sometimes help to choose the orbitals by using the energy order 
criterion, but you must learn to see the problems (like lone pair orbitals having too low 
energies at the SCF level). 
\item You typically will need correlating orbitals, that is, if you have a $\pi$ orbital you need a $\pi^*$, 
the same for $\sigma$,$\sigma^*$, but not for lone pairs. 
\item CASSCF/RASSCF geometry optimizations are the worst case. If you miss orbitals you 
might end up in a totally wrong geometry (e.g. breaking a bond usually requires the 
bonding and antibonding orbitals in the space). 
\item 
Organic (1st row atoms) molecules usually require open shell orbitals, 
$\pi$, $\pi^*$, and lone 
pairs. If 2nd row atoms are added ($S$, $P$, $Si$, etc) $s$ orbitals enter in action (s bonds are 
longer). $CH$ bonds can often be left be inactive. 


%---------------------------------------------------------------------- SLIDE -
\item Rydberg states require additional diffuse basis sets and specific orbitals in the active 
space. Use basis sets centered in the charge center of the positive ion 
(consult the manual). 
\item Transition metal chemistry (1st row) sometimes requires a double $d$ shell description 
in the active space. 
\item Lanthanides have a quite inert $4f$ shell that must be active together with 5d, 6s (6p). 
Actinides $5f$, $6d$, $7s$. 
%
\end{itemize}
%------------------------------------------------------------------------------

\begin{itemize}
\item {\bf use expbas!} start from minimal basis set, decide the active
space, and expand the basis to "normal". With small basis set you can 
clearly identify orbitals. 

\item {\bf use localization!} Especially for virtual orbitals.

\item {\bf expand active space by adding RAS1/3} - give the system a freedom, and see how it 
reacts.
\end{itemize}

%------------------------------------------------------------------------------

%---------------------------------------------------------------------- SLIDE -
\subsection{CASPT2 program:}
\begin{itemize}
\item The new IPEA = 0.25 zeroth Hamiltonian is the default. 
It particularly improves open shell cases. But there are some cases where
IPEA=0 gives better correlation with experiment.
\item Energy differences between different states or situations are only reliable between 
calculations with the same active space size and similar reference weights in CASPT2. 
\item An intruder state (low reference weight in the CASPT2 state) might be informing you 
that your active space lacks an important orbital. Check the list of large perturbative 
contributions (small denominators combined with large RHS values; check the output) 
and also the occupation number of the CASPT2 orbitals. 
\item For weakly interacting intruder states cases try the IMAGINARY level shift parameter. 
%0.05 or 0.1 au is typically enough. It is wise to compute a series: 0.0, 0.05, 0.10, 0.15
%and check that the result is converged. Then, take the lowest value that solves your 
%problem. Beware of using too large level shifts (not larger than IMAGINARY 0.20). 
Don't use the level shift to reach agreement with experiment! 
\end{itemize}

%------------------------------------------------------------------------------

%---------------------------------------------------------------------- SLIDE -
\begin{itemize}
\item For heavy valence-Rydberg mixing cases or for closely degenerated CASPT2 states, 
use MS-CASPT2. 
\item If the MS-CASPT2 description differs a lot from the CASPT2 one, 
try to check the 
calculation by increasing the active space (introducing angular correlation if possible) 
until the result is converged. The "true" solution is typically between both cases 
(CASPT2 and MS-CASPT2). If you are suspicious about the MS-CASPT2 result, 
better keep the CASPT2 one. It has worked out generally well so far. 
\end{itemize}

%------------------------------------------------------------------------------

%---------------------------------------------------------------------- SLIDE -
\subsection{RASSI program:}
\begin{itemize}
\item Remember that the program shows first the interaction among the input states and later this description might change. 
%(because the states order and nature change) in a 
%second part of the output. In general, 
ALWAYS check the changing order of states. 
\item For spin-orbit coupling calculations don't forget to include the CASPT2 energies as input 
(EJOB or HDIAG keywords) because the results depend on the energy gap. In other cases 
having the CASPT2 energies as input will help you to get the right oscillator strength and 
Einstein coefficient in the final table. 
\item If you have degenerate states be sure that the CASPT2 energies are degenerate. If they 
are not (which is common) average the energies for the degenerate set (the two 
components of E symmetry for example). 
\item Remember that the spin-orbit coupled results (e.g. TDM) depend on the number of interacting singlet and triplet states included in RASSI. 
\end{itemize}

%------------------------------------------------------------------------------


%---------------------------------------------------------------------- SLIDE -
\subsection{Geometry optimization}
\begin{itemize}
\item Not all methods have analytical derivatives.
\item Default thresholds in slapaf are typically too tight. Do not waste computer time!
\item Use constrained optimization
\item For minima on flat hypersurfaces, such in loosely bound fragments, or in slow convergence 
cases you might have to decrease the CUTOFF threshold in ALASKA
\item Be careful with the bond angle definition if you are close to a linear bond. 
You may have to switch to the LAngle definition 
\item Don't forget that CASSCF does not include dynamical correlation. In some cases you better 
change to DFT or numerical CASPT2 optimizations or, if this is not feasible, may be 
preferable to run RASSCF optimizations 
\end{itemize}


%------------------------------------------------------------------------------

%---------------------------------------------------------------------- SLIDE -
\begin{itemize}
\item Poor active spaces may lead you to symmetry broken wrong solutions (e.g. a $C_s$ minimum 
for water below the true $C_{2v}$ one) 
\item Poor geometry convergence might be reduced or at least controlled by reducing the initial 
trust radius with the MAXSTEP keyword or/and by doing the optimization in Cartesian 
coordinates (CARTESIAN) 
\item In order to obtain localized solutions it might be a good idea to feed the program with a 
slightly distorted geometry that helps the method to reach the non symmetric solutions. 
Other possibilities are to use an electric field, to add a charge far from the system or use a  solvent cavity. In all cases you break symmetry and allow less symmetric situations. 
\end{itemize}

%------------------------------------------------------------------------------

%---------------------------------------------------------------------- SLIDE -
\begin{itemize}
\item Linearly interpolated internal coordinates geometries may be a good starting point to locate 
a transition state. Use also the useful FindTS command. Sometimes can be wise to compute 
a MEP from the TS to prove that it is relevant for the studied reaction path. Try also the new 
Saddle approach! 
\item When locating a CASSCF surface crossing (MECP) ALWAYS compute CASPT2 energies 
at that point. The gap between the states can be large at that level. In severe cases you might have to make a scan with CASPT2 to find a better region for the crossing. 
\item Remember that (so far) MOLCAS does not search for true conical intersections (CIs) but 
minimum energy crossing points (MECP) because it lacks NACMEs. Note however that 
typically computed minimum energy CIs (MECIs) may not be photochemically relevant 
if they are not easily accessible. Barriers have to be computed. Use MEPs!! 
\end{itemize}

%------------------------------------------------------------------------------

%---------------------------------------------------------------------- SLIDE -
\begin{itemize}
\item Numerical hessians and optimizations may lead you to bad solutions when different 
electronic states are too close. As you move your calculation from the equilibrium geometry 
some of the points may belong to other state and corrupt your result. This might be the case 
for numerical CASPT2 crossing search. Use then MS-CASPT2 search. 
\item Remember that SA-RASSCF analytical gradients and SA-CASSCF analytical hessians are 
not implemented. 
\item Be careful with the change of roots and nature along a geometry optimization or MEP. 
For example, you start with the state in root 3 (at the CASSCF level) and reach a region 
of crossing root 3 and root 2. You may need to change to root 2 for your state. 
Not an easy solution (so far). 
\end{itemize}

%------------------------------------------------------------------------------

%---------------------------------------------------------------------- SLIDE -
\subsection{Solvent effects}
\begin{itemize}
\item Some effects of the solvent are very specific, such as hydrogen bonds, and require to 
include explicit solvent molecules. Try adding a first solvent shell (optimized with 
molecular mechanics for instance) and then a cavity, for instance with PCM. 
\item Too small cavity sizes can lead you to unphysical solutions, 
even if they seem to match experiment. 
\item Remember using NonEquilibrium (final state) and RFRoot (SA-CASSCF) 
when required 
\item QM/MM is a much powerful strategy, but it requires experience and knowledge 
of the field 
\end{itemize}


