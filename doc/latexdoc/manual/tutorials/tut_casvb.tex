%! tut_casvb.tex $ this file belongs to the Molcas repository $*/
\subsection{CASVB --- A non-orthogonal MCSCF program} 
\label{TUT:sec:casvb}
\index{casvb}\index{Program!CASVB}
\program{CASVB} is a program for carrying out quite general types of 
non-orthogonal MCSCF calculations, offering, for example, all the advantages 
associated with working within a valence bond formalism.

{\bf Warning:} as for any general MCSCF program, one may experience convergence 
problems, ({\em e.g.,}\/ due to redundant parameters), and the non-orthogonal 
optimization of orbitals can furthermore give linear dependency problems. 
Several options in \program{CASVB} can help overcoming these difficulties.

This program can be used in two basic modes:
\begin{itemize}
\item[a)] fully variational optimization
\item[b)] representation of CASSCF wavefunctions using
   overlap- ({\em relatively inexpensive\/}) or energy-based criteria.
\end{itemize}

\program{CASVB} executes the following logical steps:
Setup of wavefunction information, starting guess generation, one, or several,
optimization steps, various types of analysis of the converged solution.

\subsection{\program{CASVB} input}
\index{casvb!Input}
\program{CASVB} attempts to define defaults for as many input quantities as 
possible, so that in the simplest case no input to the \program{CASVB} module 
is required.
Sample input for a CASVB calculation on the lowest singlet state of CH$_2$:
%%%To_extract{/doc/samples/tutorials/CASVB.CH2.input}
\begin{inputlisting}
 &GATEWAY
coord
3
ch2 molecule
C 0.000000  0.000000 0.000000
H 0.000000  0.892226 0.708554
H 0.000000 -0.892226 0.708554

group= x y; basis= sto-3g
 &SEWARD
 &SCF
 &RASSCF
nactel= 6 0 0; inactive= 1 0 0 0; ras2= 3 1 2 0
lumorb
 &CASVB
\end{inputlisting}
%%%To_extract

\subsection{\program{CASVB} output}
\index{casvb!Output}
The amount of output in \program{CASVB} depends heavily on the setting of the
\keyword{PRINT} levels. In case of problems with convergence behaviour it is
recommended to increase these from their rather terse default values.

In the following the main features of the output are outlined, exemplified by 
the job in the input above. Initially, all relevant information
from the previous \program{RASSCF} calculation is recovered from the 
\file{JOBIPH} interface file, after which the valence bond wavefunction 
information is summarized, as shown below. Since 
spatial configurations have not been specified explicitly in this example, a 
single covalent configuration is chosen as default. This gives 5 spin-adapted 
VB structures.

{\begin{footnotesize}
\begin{verbatim}
 Number of active electrons :   6
           active orbitals  :   6
           Total spin       : 0.0
           State symmetry   :   1

 Spatial VB configurations
 -------------------------
     Conf. =>   Orbitals
       1   =>    1  2  3  4  5  6

 Number of VB configurations :     1
           VB structures     :     5
           VB determinants   :    20
\end{verbatim}
\end{footnotesize}}

The output from the following optimization steps summarizes only the most 
relevant quantities and convergence information at the default print level. For 
the last optimization step, for example, The output below thus
states that the VB wavefunction was found by maximizing the overlap with a 
previously optimized CASSCF wavefunction (output by the \program{RASSCF} 
program), and that the spin adaptation was done using the Yamanuchi-Kotani 
scheme. Convergence was reached in 7 iterations.

{\begin{footnotesize}
\begin{verbatim}
 -- Starting optimization - step  3 --------

 Overlap-based optimization (Svb).

 Optimization algorithm:            dFletch
 Maximum number of iterations:           50
 Spin basis:                         Kotani

 -------------------------------------------
 Optimization entering local region.
 Converged ... maximum update to coefficient:  0.59051924E-06
 Final Svb :    0.9978782695
 Number of iterations used:   7
\end{verbatim}
\end{footnotesize}}

Finally in the output below the converged 
solution is printed; orbital coefficients (in terms of the active CASSCF MOs) 
and structure coefficients. The overlap between orbitals are generally of 
interest, and, as also the structures are non-orthogonal, the structure weights 
in the total wavefunction. The total VB wavefunction is not symmetry-adapted 
explicitly (although one may ensure the correct symmetry by imposing constraints
on orbitals and structure coefficients), so its components in the various 
irreducible representations can serve to check that it is physically plausible 
(a well behaved solution generally has just one non-vanishing component).

Next follows the one-electron density with natural-orbital analysis, again with 
quantities printed in the basis of the active CASSCF MOs.

{\begin{footnotesize}
\begin{verbatim}
 Orbital coefficients :
 ----------------------
           1           2           3           4           5           6
   1  0.43397359 -0.43397359 -0.79451779 -0.68987187 -0.79451780 -0.68987186
   2 -0.80889967  0.80889967 -0.05986171 -0.05516284 -0.05986171 -0.05516284
   3  0.00005587 -0.00005587  0.20401015 -0.20582094  0.20401016 -0.20582095
   4  0.39667145  0.39667145  0.00000000  0.00000000  0.00000000  0.00000000
   5 -0.00000001 -0.00000001 -0.53361427 -0.65931951  0.53361425  0.65931952
   6  0.00000000  0.00000000  0.19696124 -0.20968879 -0.19696124  0.20968879

 Overlap between orbitals :
 --------------------------
           1           2           3           4           5           6
   1  1.00000000 -0.68530352 -0.29636622 -0.25477647 -0.29636623 -0.25477647
   2 -0.68530352  1.00000000  0.29636622  0.25477647  0.29636623  0.25477646
   3 -0.29636622  0.29636622  1.00000000  0.81994979  0.35292419  0.19890631
   4 -0.25477647  0.25477647  0.81994979  1.00000000  0.19890634  0.04265679
   5 -0.29636623  0.29636623  0.35292419  0.19890634  1.00000000  0.81994978
   6 -0.25477647  0.25477646  0.19890631  0.04265679  0.81994978  1.00000000

 Structure coefficients :
 ------------------------
      0.00000000  0.00000001  0.09455957  0.00000000 -0.99551921
 
 Saving VB wavefunction to file VBWFN.
 
 Saving VB CI vector to file JOBIPH.
 
 Svb :          0.9978782695
 Evb :        -38.4265149062

 Chirgwin-Coulson weights of structures :
 ----------------------------------------
 VB spin+space (norm   1.00000000) :
      0.00000000  0.00000000 -0.00211737  0.00000000  1.00211737
 VB spin only  (norm   0.38213666) :
      0.00000000  0.00000000  0.00894151  0.00000000  0.99105849

 Symmetry contributions to total VB wavefunction :
 -------------------------------------------------
 Irreps 1 to 4 :  0.10000000E+01  0.15118834E-17  0.17653074E-17  0.49309519E-17

 Energies for components > 1d-10 :
 ---------------------------------
 Irreps 1 to 4 : -0.38426515E+02  0.00000000E+00  0.00000000E+00  0.00000000E+00

 One-electron density :
 ----------------------
           1           2           3           4           5           6
   1  1.98488829 -0.00021330  0.00011757  0.00000000  0.00000000  0.00000000
   2 -0.00021330  1.90209222 -0.00006927  0.00000000  0.00000000  0.00000000
   3  0.00011757 -0.00006927  0.02068155  0.00000000  0.00000000  0.00000000
   4  0.00000000  0.00000000  0.00000000  0.09447774  0.00000000  0.00000000
   5  0.00000000  0.00000000  0.00000000  0.00000000  1.97572540 -0.00030574
   6  0.00000000  0.00000000  0.00000000  0.00000000 -0.00030574  0.02213479

 Natural orbitals :
 ------------------
           1           2           3           4           5           6
   1 -0.99999668  0.00000000  0.00257629  0.00000000  0.00000000  0.00005985
   2  0.00257628  0.00000000  0.99999668  0.00000000  0.00000000 -0.00003681
   3 -0.00005995  0.00000000 -0.00003666  0.00000000 -0.00000001 -1.00000000
   4  0.00000000  0.00000000  0.00000000  1.00000000  0.00000001  0.00000000
   5  0.00000000  0.99999999  0.00000000  0.00000000  0.00015650  0.00000000
   6  0.00000000 -0.00015650  0.00000000 -0.00000001  0.99999999 -0.00000001

 Occupation numbers :
 --------------------
           1           2           3           4           5           6
   1  1.98488885  1.97572545  1.90209167  0.09447774  0.02213475  0.02068154
\end{verbatim}
\end{footnotesize}}

\subsection{Viewing and plotting VB orbitals}
\index{casvb!Plotting}
In many cases it can be helpful to view the shape of the converged valence bond 
orbitals. \molcas\ therefore provides two facilities for doing this. For the 
Molden program, an interface file is generated at the end of each 
\program{CASVB} run (see also Section~\ref{UG:sec:Molden}). Alternatively a 
\program{CASVB} run may be followed by \program{RASSCF} 
(Section~\ref{UG:sec:rasscf}) and \program{GRID\_IT} 
(Section~\ref{UG:sec:gridit}) with the \keyword{VB} specification, in order to 
generate necessary files for viewing with \program{LUSCUS}.
